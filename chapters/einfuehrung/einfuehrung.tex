\chapter{Einführung}
In diesem ersten Kapitel wird der aktuelle Zustand bezüglich \acs{WuNDa} und seinen externen Benutzern beschrieben. Danach wird das Problem erläutert, das sich durch die Rationalisierung ergibt und es wird ein Lösungsansatz vorgestellt.

\section{Überblick über den aktuellen Zustand}
Bei dem \ac{WuNDa} handelt es sich um ein Softwaresystem, das es der wuppertaler Stadtverwaltung ermöglicht auf geographisch-räumliche Informationen zuzugreifen \autocite[vgl.][]{cismet-wunda}. Durch diesen Schwerpunkt findet \ac{WuNDa} häufig Anwendung im Ressort 102, dem Ressort für Vermessung, Katasteramt und Geodaten.

Weiterhin stellt die Stadtverwaltung einen Online-Zugriff auf \ac{WuNDa} bereit, so dass auch Benutzer, die keine Beamte der Stadtverwaltung sind, auf dieses System zugreifen können. Diese Gruppe von Benutzern wird im Folgenden externe Benutzer genannt.
Zu diesen externen Benutzer gehören u.\,a. die Polizei, Sparkasse, Notare und insbesondere \acp{ÖbVI}.

Die, für diese Arbeit wichtigste, Funktionalität des Online-Zugriffs ist die sogenannte Selbstentnahme von Produkten. Dabei können externe Benutzer amtliche Auszüge aus dem Liegenschaftskataster selbst beziehen und so den Weg zum Katasteramt sparen. Zudem wird auf dem Katasteramt die ansonsten benötigte Bearbeitungszeit eingespart \autocite[vgl.][]{wupp-wunda}.

Die Selbstentnahme unterstützt neben den kostenfreien Produkten auch kostenpflichtige Produkte. 
Dies zeigt sich am Beispiel der \acp{ÖbVI}. Diese können bestimmte Produkte, die im Rahmen von der Anfertigung von Amtlichen Lageplänen und der Durchführung hoheitlicher Vermessungen benötigt werden, kostenfrei selbst beziehen.
Besteht allerdings Bedarf nach anderen Produkten so sind diese kostenpflichtig \autocite[vgl.][]{wupp-wunda-oebvi}. Ausführlichere Informationen hierzu befinden sich im \autoref{subsec:gebuehren}.
Dadurch wird es erforderlich, dass jedes bezogene Produkt protokolliert wird.
Somit wird jedem externen Benutzer eine Abrechnung gestellt, dessen Abrechnungsturnus er individuell bestimmen kann.
In jeder Abrechnung befindet sich eine Auflistung der Produkte, die während einem Abrechnungsturnus gedownloadet wurden, sowie die anfallenden Gebühren.
Die Abrechnung wird manuell von einem Beamten des Katasteramts erstellt, indem er sämtliche Produktprotokolle mit Hilfe eines Tabellenkalkulationsprogramm filtert und ausschließlich die gewünschten Protokolle für einen bestimmten externen Benutzer und dessen Abrechnungsturnus übrig bleiben.
Durch dieses Verfahren ist das Erstellen der Abrechnungen arbeitsaufwändig und fehleranfällig \autocite[vgl.][]{sander-abrechnung}.


%\begin{itemize}
%	\item Wunda, ein System zum Verwalten von Katasterdaten
%	\item Auf die Katasterdaten greifen viele Benutzer zu
%	\item Dies sind Beamte des Katatsteramtes
%	\item Aber auch externe Benutzer
%	\item Ext Benutzer sind u.a. Sparkasse, Polizei, \ac{ÖbVI}
%	\item Diese 3 brauchen Zugriff auf Wunda aus diversen Gründen...
%	\item \ac{ÖbVI} brauchen den Zugriff da sie Leute sind die das Land vermessen
%	\item Sie greifen Daten ab, geben Daten rein und verarbeiten Daten
%	\item Sie sind dazu rechtlich befähigt
%	\item Um die Daten zu erhalten sie die ext Benutzer auf die Beamten des Katatseramtes angewiesen
%	\item Die erste Aufgaben war das Herausgeben der Daten in Form von Produkten, dies wurde allerdings bereits geändert. Direkt Zugriff der ext. Benutzer.
%	\item Die zweite Aufgabe ist die Berechnung der angefallenen Kosten durch die bezogenen Produkte
%	\item Jedes Produkt hat nämlich einen bestimmten Preis, der vom Verwendungszweck etc... abhängt
%	\item Die Kostenermittlung wird manuell von den Beamten durchgeführt mit Excel. ???
%\end{itemize}

\section{Rationalisierung} \label{sec:rationalisierung}
Die Selbstentnahme wurde einerseits wegen des Erlasses zur Online-Bereitstellung vom 2.2.2011 für die \acp{ÖbVI} eingeführt \autocite[vgl.][]{wupp-wunda-oebvi}, andererseits wegen eines Rationalisierungsplans, der zwischen dem Ressort 102 und dem Wuppertaler Stadtdirektor vereinbart wurde.
Dieser Plan schreibt vor, dass bis Ende 2014 etwa zweieinhalb  Vollzeitkräfte wegfallen und nicht wieder besetzt werden.

Der geplanten Rationalisierung steht allerdings im Weg, dass die Methode zur Erstellung der Abrechnungen arbeits- und zeitintensiv ist.
Nichtsdestoweniger werden erhebliche Anstrengungen unternommen, um die Gruppe der externen Benutzer zu vergrößern.
Durch das Wachsen dieser Gruppe, wächst allerdings auch der Aufwand der mit dem Erstellen der Abrechnungen einhergeht, folglich wird dieser noch zeitintensiver.

Die Problemstellung lautet deshalb wie folgt: die Rationalisierung kann nur dann vollständig durchgeführt werden, falls die Methode zur Erstellung der Abrechnungen so verändert werden kann, dass sie weniger zeitaufwändig ist \autocite[vgl.][]{sander-abrechnung}. 

%\begin{itemize}
%	\item Das Amt will Beamten abbauen. Stichwort Rationalisierung
%	\item Die Rationalisierung kann nicht durchgeführt werden solange die ext. Benutzer auf die Beamten angewiesen sind.
%	\item Die Ermittlung der Kosten ist zeitintensiv
%	\item Problem das gelöst werden muss
%\end{itemize}

\section{Lösungsansatz}

Der, in dieser Arbeit verfolgte, Lösungsansatz ist es eine Methode zu finden, die die Erstellung der Abrechnungen möglichst automatisiert und somit den Arbeitsaufwand für die Beamten zu minimalisieren. 

Der naheliegendste Ansatz ist es \acs{WuNDa} für diese Automatisierung zu erweitern, da über dieses System bereits die Selbstentnahme bereitgestellt wird und die dadurch anfallenden Gebühren protokolliert werden.
Dies bedeutet dass für das Erreichen des Ziels mindestens eine Übersicht und Auswahl für die einzelnen protokollierten Produkte zur Verfügung stehen muss, weiterhin muss ein automatisches Erstellen der Abrechnungen dieser Produkte möglich sein.
Hierbei handelt es sich lediglich um minimale Anforderungen, um diese vollständig erfassen zu können wird im nächsten Kapitel die bereits vorgestellte Arbeit der externen Benutzer und die beschriebene Rationalisierung beleuchtet und in Verbindung gebracht.
Danach wird in \autoref{ch:spezi} detaillierter auf die benötigten und zusätzlich gewünschten Erweiterungen von \ac{WuNDa} eingegangen.
Anschließend wird in \autoref{ch:realisierung} die Umsetzung der Erweiterungen in \ac{WuNDa} beschrieben.

\section{Arbeitsabläufe von Projektentwicklern}
Der Abschluss dieser Arbeit besteht aus \autoref{ch:aussicht}, hier wird der Arbeitsablauf eines Projektentwicklers analysiert um mögliche benötigte Funktionalitäten in \ac{WuNDa} ausfindig zu machen.
Bei dieser Kundengruppe handelt es sich noch nicht um externe Benutzer, dies soll sich in näher Zukunft jedoch ändern, so dass auch Projektentwickler \ac{WuNDa} benutzen und somit das Katasteramt entlasten.


%\begin{itemize}
%	\item Lösung: Rechnung auto. erstellen
%	\item Lösung über Wunda ist möglich
%	\item ext. Zugriff ist bereits gegeben und Kosten werden bereits protokolliert
%	\item D.h. Daten müssen in gewünschte Form zusammengefasst werden
%	\item Dass dies tatsächlich Zeit spart wird im nächsten Kapitel erläutert
%\end{itemize}




