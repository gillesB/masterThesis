\chapter{Einführung}
\section{Überblick über den aktuellen Zustand}
\begin{itemize}
	\item Wunda, ein System zum Verwalten von Katasterdaten
	\item Auf die Katasterdaten greifen viele Benutzer zu
	\item Dies sind Beamte des Katatsteramtes
	\item Aber auch externe Benutzer
	\item Ext Benutzer sind u.a. Sparkasse, Polizei, \ac{ÖbVI}
	\item Diese 3 brauchen Zugriff auf Wunda aus diversen Gründen...
	\item \ac{ÖbVI} brauchen den Zugriff da sie Leute sind die das Land vermessen
	\item Sie greifen Daten ab, geben Daten rein und verarbeiten Daten
	\item Sie sind dazu rechtlich befähigt
	\item Um die Daten zu erhalten sie die ext Benutzer auf die Beamten des Katatseramtes angewiesen
	\item Die erste Aufgaben war das Herausgeben der Daten in Form von Produkten, dies wurde allerdings bereits geändert. Direkt Zugriff der ext. Benutzer.
	\item Die zweite Aufgabe ist die Berechnung der angefallenen Kosten durch die bezogenen Produkte
	\item Jedes Produkt hat nämlich einen bestimmten Preis, der vom Verwendungszweck etc... abhängt
	\item Die Kostenermittlung wird manuell von den Beamten durchgeführt mit Excel. ???
\end{itemize}

\section{Rationalisierung}
\begin{itemize}
	\item Das Amt will Beamten abbauen. Stichwort Rationalisierung
	\item Die Rationalisierung kann nicht durchgeführt werden solange die ext. Benutzer auf die Beamten angewiesen sind.
	\item Die Ermittlung der Kosten ist zeitintensiv
	\item Problem das gelöst werden muss
\end{itemize}

\section{Lösungsansatz}
\begin{itemize}
	\item Lösung: Rechnung auto. erstellen
	\item Lösung über Wunda ist möglic
	\item ext. Zugriff ist bereits gegeben und Kosten werden bereits protokolliert
	\item D.h. Daten müssen in gewünschte Form zusammengefasst werden
	\item Dass dies tatsächlich Zeit spart wird im nächsten Kapitel erläutert
\end{itemize}