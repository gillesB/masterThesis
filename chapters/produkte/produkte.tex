\chapter{Beleuchtung der Arbeitsweise von externen Benutzer}

\begin{itemize}
	\item Um einen besseren Einblick in der Arbeit der externen Benutzer zu erhalten werden die einzelnen Produkte aufgelistet
	\item Danach folgt ein Beweis dass ein tatsächlicher Arbeitsaufwand durch viele Rechnungen besteht.
\end{itemize}

\section{Auflistung der Produkte}
\begin{itemize}
	\item Die Produkte sind in verschiedene Kategorien eingeteilt
	\item Die Kategorien sind wichtig weil...
	\item Tabelle
\end{itemize}

\begin{longtable}{|p{0.3\textwidth}|p{0.7\textwidth}|}
	\hline 
	Produktname & Beschreibung \\ 
	\hline 
	\endhead
	Flurstücksnachweis
	&
	Im „Flurstücksnachweis“ werden alle flurstücksrelevanten Angaben beschrieben. Dies sind
	neben der „katastertechnischen Bezeichnung des Flurstücks“ „Gebietszugehörigkeit“, „Lage“,
	„Fläche“, „Tatsächliche Nutzung“, „Klassifizierung nach Straßen- oder Wasserrecht“, „Gebäude“, „Hinweise zum Flurstück“, „Buchungsart“ und „Buchung“. \autocite{adv-alkis-erlaeuterung} \\ 
	\hline
	Flurstücks- und Eigentumsnachweis (NRW)
	&
	foo \\
	\hline
	Flurstücks- und Eigentumsnachweis (kommunal)
	&
	Einzelobjektrenderer ALKIS-Flurstücke / Produkte / PDF-Produkte / Flurstücks- und Eigentumsnachweis (kommunal) und Aggregierender Renderer ALKIS-Flurstücke / PDF-Produkte / Flurstücks- und Eigentumsnachweis (kommunal) \\
	\hline
	Bestandsnachweis (NRW)
	&
	foo \\
	\hline
	Bestandsnachweis (kommunal)
	&
	Der „Bestandsnachweis“ enthält alle Grundstücke, die auf einem Buchungsblatt gebucht sind. \\
	\hline
	Grundstücksnachweis (NRW)
	&
	Im „Grundstücksnachweis“ wird das unter einer laufenden Nummer im Buchungsblatt geführte Grundstück beschrieben. Neben den im Grundbuch gebuchten Grundstücken können
	dies auch von der Buchungspflicht befreite Grundstücke (§ 3 Abs. 2 GBO) und Grundstücke
	sein, die noch nicht im Grundbuch gebucht sind (Verzeichnis der Bodenordnung ersetzt den
	Grundbuchnachweis). \autocite{adv-alkis-erlaeuterung}\\
	\hline
	Flurkarte (NRW)
	&
	Der „Auszug aus dem Liegenschaftskataster - Flurkarte NRW“  ist eine landesweit einheitliche Ausgabe aus dem Grunddatenbestand. Er enthält alle Flurstücke, Gebäude und Nutzungsarten ohne weitere topographische Inhalte. Die Flurkarte NRW ist zum Beispiel für Beleihungszwecke ausreichend. \autocite{wupp-alkis} \\
	\hline
	Stadtgrundkarte (kommunal)
	&
	Umfangreiche, das Stadtbild prägende Inhalte, die über den Grunddatenbestand hinausgehen - vorwiegend topographischer Art - sind zusätzlich in dem „Auszug aus dem Liegenschaftskataster mit kommunalen Ergänzungen - Stadtgrundkarte enthalten“.
	Dies sind zum Beispiel Fahrbahnbegrenzungen, Böschungen, Mauern, Bahngleise und Zäune, aber auch Balkone oder die Geschosszahl von Gebäuden. 
			
	Für alle Bürger und Fachkunden, die diese gewohnten Inhalte auch weiterhin benötigen, beispielsweise für Planungen, ist diese Ausgabe die richtige Wahl, denn sie entspricht weitgehend der bisherigen Wuppertaler Liegenschaftskarte. \autocite{wupp-alkis} \\
	\hline
	Stadtgrundkarte
	&
	Stadtgrundkarte ohne kommunale Ergänzungen. \\
	\hline
	Stadtgrundkarte mit Höhenlinien
	&
	Stadtgrundkarte mit zusätzlichen Höhenlinien. \\
	\hline
	Schätzungskarte (NRW)
	&
	Die Schätzungskarte ist „eine auf der Grundlage der Flurkarte hergestellte Einzelkarte, in der zusätzlich die Schätzungsergebnisse der Reichsbodenschätzung eingetragen sind.“  \autocite{gg-schaetzungskarte}
			
	Die Kennzeichnung aller landwirtschaftlich nutzbaren Bodenflächen nach der Bodenbeschaffenheit durch Einteilung in Klassen (Bestandsaufnahme) und die Feststellung ihrer Ertragsfähigkeit aufgrund der natürlichen Ertragsbedingungen durch Ermittlung von Wertzahlen (eigentliche Schätzung). Gestartet im Jahr 1934. Auch heute noch wertvolle Datenquelle in ökologischen und Landwirtschaftsanwendungen von GIS. \autocite{gg-reichsbodenschaetzung} \\
			
	\hline
	Amtliche Basiskarte (NRW)
	&
	Die Amtliche Basiskarte NRW (ABK NRW oder nur ABK) ist eine Übersichtskarte, die eine Verbindung zwischen der großmaßstäbigen Liegenschaftskarte und der Topographischen Karte 1:25000 (TK25) herstellt. (wikipedia) \\
	\hline
	Punktliste (TEXT)
	&
	ALKIS-Lagefestpunkte als Textdokument.
			
	Die Lagefestpunkte bilden in ihrer Gesamtheit das Lagefestpunktfeld mit einer Punktdichte je nach Bundesland von 1 Festpunkt auf 1 - 5 $km^2$. Das Lagefestpunktfeld dient heute noch überwiegend als Grundlage für alle amtlichen Vermessungen, insbesondere der Katastervermessung und der topographischen Landesaufnahme. \autocite{adv-lagefestpunkte}\\
	\hline
	Punktnachweis (PDF)
	&
	ALKIS-Lagefestpunkte als PDF-Dokument. \\
	\hline
	\ac{DGK}
	&
	Die \ac{DGK} ist eine aus der Liegenschaftskarte/Stadtgrundkarte abgeleitete topografische Karte, welche die Deutsche Grundkarte (DGK5 im Maßstab 1:5.000) abgelöst hat. \autocite{wupp-digitale-grundkarte} \\
	\hline
	Digitale Grundkarte mit Höhenlinien
	&
	Eine \ac{DGK} mit zusätzlichen Höhenlinien \\
	\hline
	Orthofoto
	&
	Orthofotos sind vereinfacht gesagt Luftbild und Karte in einem: Sie bieten kartenähnliche Genauigkeit und Maßstäbigkeit auf der Grundlage einer fotorealistischen Abbildung der Erdoberfläche.  \autocite{wupp-orthofoto}\\
	\hline
	Orthofoto mit Katasterdarstellung
	&
	Orthofoto mit zusätzlicher Katasterdarstellung. \\
	\hline
	Nivellement-Punkt
	&
	Darstellung eines einzlnen \acf{NivP}. \\
	\hline
	NivP-Übersicht
	&
	Übersicht über \acp{NivP}. \acp{NivP} werden auch Höhenfestpunkte genannt. Auf der Grundlage topographischer Karten 1:25 000 geben NivP-Übersichten den großräumigen Überblick auf die Nummerierung und die Lage der NivP im Gelände an \autocite{adv-nivp} \\
	\hline
	AP-Karte
	&
	Formular zur Erhebung von ALKIS-Punkten (für \acp{AP} und sonstige Vermessungspunkte) \\
	\hline
	AP-Übersicht
	&
	\aclp{AP} sind Lagefestpunkte, die das \acs{TP}-Netz unterster Ordnung (im Regelfall 4. Ordnung) verdichten und – gemeinsam mit diesen Trigonometrischen Punkten – einen koordinatenmäßigen Anschluss von Messungen an das übergeordnete Bezugssystem der Landesvermessung ermöglichen. \autocite{wiki-aufnahmepunkt} \\
	\hline
	\ac{PNÜ}
	&
	Eine Karte, die die Punktnummer eines Flurstückes zeigt.\\
	\hline
	Vermessungsriss
	&
	Darstellung eines einzlnen Vermessungsrisses. \\
	\hline
	Dokument Liegenschaftskatasterakte
	&
	foo \\
	\hline
	\acs{NAS}-Daten (mit Eigentümern)
	&
	Vollständiger \acs{NAS}-Datensatz, wird als \acs{XML}-Datei ausgegeben. \\
	\hline
	\acs{NAS}-Daten (ohne Eigentümer)
	&
	\acs{NAS}-Datensatz ohne Eigentümer, wird als \acs{XML}-Datei ausgegeben. \\
	\hline
	\acs{NAS}-Daten (nur Punkte)
	&
	\acs{NAS} Datensatz nur Punkte, wird als \acs{XML}-Datei ausgegeben. \\
	\hline
	\acs{DXF}-Daten (Stadtgrundkarte kommunal)
	&
	Die Stadtgrundkarte (kommunal) als \acs{DXF}-Datei.  \\
	\hline
	GEOTIFF-Daten (Stadtgrundkarte kommunal)
	&
	Die Stadtgrundkarte (kommunal) als GEOTIFF-Datei. \\
	\hline
			
\end{longtable} 

\section{Beleuchtung typischer Workflows der externen Benutzer}
\begin{itemize}
	\item Um einen noch besseren Einblick in die Arbeit zu erhalten werden typische Arbeitsschritte der externen Benutzer aufgezeigt
	\item typische Workflows aufzeichnen...
	\item Aufzeigen dass Arbeitsaufwand vorhanden ist und Rationalisierung tatsächlich nicht sofort durchgeführt werden kann
\end{itemize}

\section{Aktuelle Unterstützung der Workflows durch das System}
\begin{itemize}
	\item fachlicher IST-Zustand
	\item Der Arbeitsschritt xy und andere bereits durch das System unterstützt in dem der ext. Benutzer die Produkte aus seinem Büro heraus beziehen kann.
	\item Nach einem bestimmten Abrechnungsturnus erhält er die manuell erstellte Rechnung.
	\item Möglich da: Protokollierung der bezogenen Produkte funktioniert bereits
	\item Da sich die benötigten Änderungen vor allem auf das Erstellen der Rechnungen bezieht ändert sich für den ext. Benutzer nicht so viel.
\end{itemize}









