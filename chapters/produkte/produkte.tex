\chapter{Beleuchtung der Arbeitsweise von externen Benutzer}

In diesem Kapitel soll aufgezeigt werden, warum die geplante Rationalisierung mit dem aktuellen Stand nicht durchzuführen ist.

Um dies zu erreichen müssen in einem ersten Schritt einige fachliche Erläuterungen erfolgen. So werden einige Begriffe definiert und die Gebühren zu den Produkten erläutert. Daraufhin  werden die Produkte aufgelistet, die von den externen Benutzern bezogen werden können.

Im Anschluss werden einige Workflows von externen Benutzern beleuchtet, woraufhin überprüft wird ob und wie \ac{WuNDa} diese unterstützt. Demnach wird der aktuelle Stand von \ac{WuNDa} bezüglich der bestehenden Problematik beschrieben.

Im letzten Teil des Kapitels wird mittels der zuvor gewonnen Erkenntnisse verdeutlicht, dass \ac{WuNDa} dem Problem entgegenwirkend erweitert werden kann. 

%\begin{itemize}
%	\item Um einen besseren Einblick in der Arbeit der externen Benutzer zu erhalten werden die einzelnen Produkte aufgelistet
%	\item Danach folgt ein Beweis dass ein tatsächlicher Arbeitsaufwand durch viele Rechnungen besteht.
%\end{itemize}

\section{Fachliche Erläuterungen}
In diesem Abschnitt werden einige fachliche Begriffe und Zusammenhänge erläutert, die für das Verständnis dieses Kapitel von Bedeutung sind.

\subsection{ÖbVI}
Damit hoheitliche Vermessungen nicht nur durch Verwaltungen durchgeführt werden können, kann der Staat private Vermessungsingenieure mit hoheitlichen Aufgaben beleihen.
Solche Freiberufler werden \acfp{ÖbVI} genannt und sind berechtigt Katastervermessungen durchzuführen. Dies sind Vermessungen, die der Einrichtung und Fortführung des Liegenschaftskatasters und der Feststellung oder Abmarkung von Grundstücksgrenzen dienen \autocite[vgl.][]{bdvi-oebvi}.
Weiterhin dürfen sie Grenzzeichen setzen oder entfernen \autocite[vgl.][]{wolff-oebvi}. 

\subsection{ALKIS und sein Datenmodell}
Das \acf{ALKIS} ersetzt das \acf{ALB} und die \acf{ALK}, indem es die Daten beider Systeme vereint. Die \acs{AdV} hat \acs{ALKIS} entwickelt und organisiert die Pflege des Datenmodells \autocite[vgl.][]{adv-alkis}.

Das Datenmodell von ALKIS ist stufenweise aufgebaut. Die oberste Stufe ist das ALKIS-Fachschema der AdV, dieses Fachschema bildet eine Grundlage mit der sämtliche Objekte, Attribute und Attributwerte aller Katasterbehörden Deutschlands abgebildet werden können.
Die folgenden Stufen sind Reduktionen dieses Fachschema, d.h. sie benutzen nur einen Teil dieser Grundlage. Weiterhin ist die jeweils folgende Stufe eine Reduktion der vorhergegangenen Stufe. Eine Übersicht der einzelnen Stufen kann der \myfig{fig:prod-stufen} entnommen werden. 

\begin{figure}[htbp]
	\centering
	\incgraph{width=0.8\textwidth}{./img/prod-stufen.png}
	\caption{Erweiterungen der Datenbestände}
	\label{fig:prod-stufen}
\end{figure}

In \ac{NRW} wurde das Maximalprofil NRW festgelegt, da hier einige Objekte aus dem ALKIS-Fachschema der AdV nicht benötigt werden. Alle Informationen die im Maximalprofil NRW geführt werden gehören zum amtlichen Liegenschaftskataster.
Weiterhin dürfen in NRW nur diese Informationen innerhalb ALKIS geführt werden. Jedes Katasteramt darf frei wählen welche Informationen es aus dieser Stufe führen will, vorausgesetzt diese Informationen gehören nicht zu einer unteren Stufe.
Die nächste Stufe bildet der Grunddatenbestand NRW dieser ist landesweit gültig und muss von jedem Katasteramt in NRW geführt werden.
Die unterste Stufe bildet der Grunddatenbestand AdV, dieser ist bundesweit gültig und muss von jedem Katasteramt in Deutschland geführt werden.

Weiterhin haben die Katasterämter die Möglichkeit eine eigene Stufe zu diesem Stufenkonzept hinzuzufügen, diese gehört im Gegensatz zu den vorherigen Stufen nicht mehr zum amtlichen Liegenschaftskataster \autocite[vgl.][1-4]{bezk-grunddaten}. In Wuppertal ist dies \acs{ALWIS}.

\subsubsection{ALWIS}
Um sämtliche Datenbestände, die für die Katasterbenutzung benötigt werden, abzubilden wurde in Wuppertal eine weitere Schicht eingeführt, die \acf{ALWIS} genannt wird.
Diese umfasst sämtliche Informationen, die in ALKIS zur Verfügung stehen und wurde um die fehlenden Informationen erweitert.
Eine der Erweiterungen ist das Vermessungsregister.

\subsection{Gebühren der Produkte} \label{subsec:gebuehren}
In diesem Abschnitt wird das Zustandekommen der Gebühren der Produkte kurz beschrieben.
Da die genaue Ermittlung der Gebühren nicht Bestandteil dieser Arbeit ist, wird hier nur auf das Nötigste eingegangen. 

Die Gebühren der einzelnen Produkte wird durch eine Grundgebühr bestimmt, auf die verschiedene Rabatte angewendet werden können.
Ein Rabatt kann z.B. ein Mengenrabatt sein oder ein Rabatt der durch eine Reglung gegeben werden muss. 

Eine Reglung die insbesondere die \acp{ÖbVI} betrifft ist in der \ac{VermWertGebO} festgelegt.
Laut dieser muss die Selbstentnahme der Produkte, die zur Durchführung bestimmter Amtshandlungen benötigt werden, kostenfrei sein.
Zu diesen Handlungen gelten die Anfertigung von Amtlichen Lageplänen und die Durchführung hoheitlicher Vermessungen \autocite[vgl.][]{wupp-wunda-oebvi}.
Die Produkte die diese Bedingungen erfüllen werden im folgenden vermessungsrelevant genannt. Der Rabatt der bei diesen Produkten gegeben wird, wird "`zweckabhängiger Rabatt"' genannt.

Weitere Informationen können dem Dokument "`Entgeltordnung Ressort 102"' \autocite{wupp-entgelt} entnommen werden.

\section{Auflistung der Produkte} \label{sec:produktliste}
In diesem Abschnitt werden die Produkte aufgelistet, welche die externen Benutzer selbst beziehen können. Diese Produkte sind in vier Produktklassen eingeteilt:
\begin{itemize}
	\item ALKIS-Standardausgaben
	\item Standardausgaben WUP-kommunal
	\item WuNDa-Berichte
	\item Daten Liegenschaftskataster
\end{itemize}
Diese Produktklassen unterteilen die Produkte logisch und geben an von welcher Stelle diese definiert wurden.

%\begin{itemize}
%	\item Die Produkte sind in verschiedene Produktklassen eingeteilt
%	\item Die Produktklassen sind wichtig weil...
%	\item Tabelle
%\end{itemize}
\subsection{ALKIS-Standardausgaben}

In der Produktklasse ALKIS-Standardausgaben wird zwischen landesweit einheitliche Standardausgaben und kommunalen Produkten unterschieden.
Bei den Standardausgaben handelt es sich um Auszüge aus dem Liegenschaftskataster deren Inhalt und Form landesweit festgelegt ist.
Diese Ausgaben beinhalten nur Daten die im Grunddatenbestand NRW definiert sind \autocite[vgl.][8]{bezk-grunddaten}.

Bei den kommunalen Produkten handelt es sich um Auszüge aus dem Liegenschaftskataster, allerdings dürfen auch Daten verwendet werden, die nicht im Grunddatenbestand NRW definiert sind. Ein Beispiel eines kommunalen Produktes wäre eine modifizierte Standardausgabe, die mit Daten aus dem Maximalprofil NRW erweitert wurde. Weiterhin muss die Bezeichnung bzw. Beschriftung deutlich gekennzeichnet sein, so dass eine Verwechslung mit den Standardausgaben ausgeschlossen ist \autocite[vgl.][9]{bezk-grunddaten}. Dies wird durch die Anhänge \textit{(NRW)} bzw. \textit{(kommunal)} garantiert.

Die Produkte der Produktklasse ALKIS-Standardausgaben können der Tabelle \ref{tab-alkis-standard-start} \vpagerefrange{tab-alkis-standard-start}{tab-alkis-standard-end} entnommen werden.

\begin{longtable}{|p{0.3\textwidth}|p{0.642\textwidth}|}
	\caption{Produkte der Produktklasse ALKIS-Standardausgaben} \label{tab-alkis-standard-start} \\
	\hline 
	\head{Produktname} & \head{Beschreibung} \tabularnewline
	\hline 
	\endfirsthead	
		\caption{Produkte der Produktklasse ALKIS-Standardausgaben}  \\
			\hline
	\head{Produktname} & \head{Beschreibung} \tabularnewline
	\hline 
	\endhead	


	Flurstücksnachweis
	&
	\enquote{Im "`Flurstücksnachweis"' werden alle flurstücksrelevanten Angaben beschrieben.
	Dies sind neben der "`katastertechnischen Bezeichnung des Flurstücks"' "`Gebietszugehörigkeit"', "`Lage"',
	"`Fläche"', "`Tatsächliche Nutzung"', "`Klassifizierung nach Straßen- oder Wasserrecht"', "`Gebäude"', "`Hinweise zum Flurstück"', "`Buchungsart"' und "`Buchung"'.} \autocite[269]{adv-alkis-erlaeuterung}
	\\ 
	\hline
	Flurstücks- und Eigentumsnachweis (NRW)
	&
	\enquote{Der "`Flurstücks- und Eigentumsnachweis"' enthält alle Angaben des "`Flurstücksnachweises"'
	und darüber hinaus die "`Eigentümer"' ("`Name"', "`Adresse"',) und die "`Angaben zu den Berechtigten.} \autocite[269]{adv-alkis-erlaeuterung} Ein Beispielprodukt ist unter  \vref{fig:eigentumsnachweis_nrw} zu finden. \\
	\hline
	Flurstücks- und Eigentumsnachweis (kommunal)
	&
	Erweiterung des Flurstücks- und Eigentumsnachweis (NRW). Beim Beispielprodukt \ref{fig:eigentumsnachweis_kom} ist zu erkennen, dass mehr Informationen zu den Eigentümer angegeben sind. \\
	\hline
	Bestandsnachweis (NRW)
	&
	\enquote{Der "`Bestandsnachweis"' enthält alle Grundstücke, die auf einem Buchungsblatt gebucht sind.} \autocite[269]{adv-alkis-erlaeuterung} Ein Beispielprodukt ist unter  \vref{fig:bestandsnachweis_nrw} zu finden. \\
	\hline
	Bestandsnachweis (kommunal)
	&
	Der Bestandsnachweis (kommunal) enthält die gleichen Erweiterungen wie dies beim Flurstücks- und Eigentumsnachweis der Fall ist. \\
	\hline
	Grundstücksnachweis (NRW)
	&
	\enquote{Im "`Grundstücksnachweis"' wird das unter einer laufenden Nummer im Buchungsblatt geführte Grundstück beschrieben. Neben den im Grundbuch gebuchten Grundstücken können
	dies auch von der Buchungspflicht befreite Grundstücke (§ 3 Abs. 2 GBO) und Grundstücke
	sein, die noch nicht im Grundbuch gebucht sind (Verzeichnis der Bodenordnung ersetzt den
	Grundbuchnachweis).} \autocite[269]{adv-alkis-erlaeuterung}\\
	\hline
	Flurkarte (NRW)
	&
	\enquote{Der "`Auszug aus dem Liegenschaftskataster - Flurkarte NRW"'  ist eine landesweit einheitliche Ausgabe aus dem Grunddatenbestand. Er enthält alle Flurstücke, Gebäude und Nutzungsarten ohne weitere topographische Inhalte.} \autocite{wupp-alkis} In \acs{WuNDa} wird die Flurkarte auch Liegenschaftskarte genannt, ein Beispieldokument kann unter \ref{fig:flurkarte} gefunden werden.\\
	\hline
	Stadtgrundkarte (kommunal)
	&
Die Stadtgrundkarte (kommunal) ist eine umfangreiche Grundkarte, die, das Stadtbild prägende, Inhalte wiedergibt. Zusätzlich zur Stadtgrundkarte werden topographische Informationen wie zum Beispiel 
Fahrbahnbegrenzungen, Böschungen, Mauern, Bahngleise und Zäune, aber auch Balkone oder die Geschosszahl von Gebäuden.	
 \autocite[vgl.][]{wupp-alkis} Ein Beispieldokment kann unter \vref{fig:stadtgrundkarte} gefunden werden. \\
	\hline			
	Stadtgrundkarte
	&
	Stadtgrundkarte ohne kommunale Ergänzungen. Wird nicht vertrieben. \\
	\hline
	Schätzungskarte (NRW)
	&
	Die Schätzungskarte basiert auf der Flurkarte und zeigt zusätzlich Schätzungsergebnisse einer Bodenschätzung \autocite[vgl.][]{gg-schaetzungskarte}. 
	
	Bei einer Bodenschätzung werden landwirtschaftlich genutzte Böden auf ihre Fruchtbarkeit hin geschätzt, so dass auf dieser Basis eine Besteuerung vorgenommen werden kann \autocite[vgl.][]{sachsen-boden}.   \\
						
	\hline
	Amtliche Basiskarte (NRW)
	&
	\enquote{Die Amtliche Basiskarte NRW (ABK NRW oder nur ABK) ist eine Übersichtskarte, die eine Verbindung zwischen der großmaßstäbigen Liegenschaftskarte und der Topographischen Karte 1:25000 (TK25) herstellt.} \todo{quelle:(wikipedia)} \\
	\hline
	Punktnachweis (PDF)
	&
	Beim Punktnachweis handelt es sich um ein Dokument das Informationen zu verschiedenen Vermessungspunkte enthält wie etwa \ac{TP}, \ac{NivP} und \ac{AP} \autocite[vgl.][]{innen-punkte}. Ein Beispiel eine Punktnachweises kann \ref{fig:punktnachweis} entnommen werden.	 \\
	\hline
	Punktliste (TEXT)
	&
	Die Punktliste enthält Informationen zu Vermessungspunkten, ähnlich wie der Punktnachweis, es handelt sich jedoch um einfache Textdatei. \\
	\hline
\end{longtable} 
\label{tab-alkis-standard-end}
	
\subsection{Standardausgaben WUP-kommunal}

Die ALKIS-Standardausgaben decken nicht alle Produkte ab, die benötigt werden. Aus diesem Grund musste die Stadtverwaltung Wuppertal weitere Produkte selbst definieren.
Bei diesen Eigendefinition werden keine ALKIS-Objekte dargestellt und somit handelt es sich nicht um Auszüge aus dem Liegenschaftskataster.
Der Inhalt der Produkte wird aus Rasterdatenbeständen bezogen und zeigen demnach Geodaten mit Bild.
Obwohl es sich nicht um ALKIS-Standardausgaben handelt sind einige dieser Produkte vermessungsrelevant und somit in bestimmten Fällen für den ÖbVI gebührenfrei.

Die Produkte der Produktklasse Standardausgaben WUP-kommunal können der Tabelle \vref{tab-wupp} entnommen werden.

%\begin{itemize}
%\item von der Stadtverwaltung selbst definiert
%\item keine ALKIS-Standardausgaben
%\item Inhalt sind keine ALKIS-Objekte
%\item Ausgaben von speziellen eigens definierten Rasterdatenbeständen
%\item Werden trotzdem benötigt (auch wenn kein Alkis)
%\item da teils vermessungsrelevant, teils gebührenfrei für ÖbVI
%\item teils eigene Preisvorschriften
%\item siehe Mail für mehr Details
%\end{itemize}

\begin{longtable}{|p{0.3\textwidth}|p{0.642\textwidth}|}
	\caption{Produkte der Produktklasse Standardausgaben WUP-kommunal} \label{tab-wupp} \\
	\hline 
	\head{Produktname}  & \head{Beschreibung} \tabularnewline
	\hline 
	\endfirsthead	
	\caption{Produkte der Produktklasse Standardausgaben WUP-kommunal}\\
	\hline 
	\head{Produktname}  & \head{Beschreibung} \tabularnewline
	\hline 
	\endhead
	\ac{DGK}
	&
	\enquote{Die \ac{DGK} ist eine aus der Liegenschaftskarte/Stadtgrundkarte abgeleitete topografische Karte, welche die Deutsche Grundkarte (DGK5 im Maßstab 1:5.000) abgelöst hat.} \autocite{wupp-digitale-grundkarte} \\
	\hline
	Digitale Grundkarte mit Höhenlinien
	&
	Eine \ac{DGK} mit zusätzlichen Höhenlinien \\
	\hline
	Orthofoto
	&
	\enquote{Orthofotos sind vereinfacht gesagt Luftbild und Karte in einem: Sie bieten kartenähnliche Genauigkeit und Maßstäbigkeit auf der Grundlage einer fotorealistischen Abbildung der Erdoberfläche.}  \autocite{wupp-orthofoto}\\
	\hline
	Orthofoto mit Katasterdarstellung
	&
	Orthofoto mit zusätzlicher Katasterdarstellung. \\
	\hline
	NivP-Übersicht
	&
	Übersicht über \acp{NivP}. \acp{NivP} werden auch Höhenfestpunkte genannt. \enquote{Auf der Grundlage topographischer Karten 1:25000 geben NivP-Übersichten den großräumigen Überblick auf die Nummerierung und die Lage der NivP im Gelände an.} \autocite{adv-nivp} \\
	\hline
	AP-Übersicht
	&
	\enquote{\aclp{AP} sind Lagefestpunkte, die das \acs{TP}-Netz unterster Ordnung (im Regelfall 4. Ordnung) verdichten und -- gemeinsam mit diesen Trigonometrischen Punkten -- einen koordinatenmäßigen Anschluss von Messungen an das übergeordnete Bezugssystem der Landesvermessung ermöglichen.} \autocite{wiki-aufnahmepunkt} \\
	\hline
	\ac{PNÜ}
	&
	Eine Karte, die die Punktnummer eines Flurstückes zeigt.\\
	\hline
	Stadtgrundkarte mit Höhenlinien
	&
	Stadtgrundkarte mit zusätzlichen Höhenlinien. \\
	\hline
\end{longtable} 	

\subsection{WuNDa-Berichte}
Die WuNDa-Berichte sind ebenfalls eigens definierte Produkte. Im Gegensatz zu den Standardausgaben WUP-kommunal enthalten sie allerdings keine Geodaten mit Bild, sondern sind Berichte die ein einzlnes Objekt beschreiben. Weiterhin sind sie ebenfalls nicht Bestandteil von ALKIS und auf Grund ihres Inhaltes vermessungsrelevant.

Die Produkte der Produktklasse Standardausgaben WuNDa-Berichte können der Tabelle \ref{tab-wunda-berichte} entnommen werden.
\begin{longtable}{|p{0.3\textwidth}|p{0.642\textwidth}|}	
	\caption{Produkte der Produktklasse WuNDa-Berichte} \label{tab-wunda-berichte} \\
	\hline 
	\head{Produktname}  & \head{Beschreibung} \tabularnewline
	\hline 
	\endfirsthead
	\caption{Produkte der Produktklasse WuNDa-Berichte} \\
	\hline 
	\head{Produktname}  & \head{Beschreibung} \tabularnewline
	\hline 
	\endhead
	AP-Karte
	&
	Formular zur Erhebung von ALKIS-Punkten (für \acp{AP} und sonstige Vermessungspunkte) \\
	\hline
	Nivellement-Punkt
	&
	Darstellung eines einzlnen \acf{NivP}. \\
	\hline
	Vermessungsriss
	&
	Darstellung eines einzlnen Vermessungsrisses. \\
	\hline
	Dokument Liegenschaftskatasterakte
	&
	\todo{Was ist das?} \\
	\hline
\end{longtable} 
	
\subsection{Daten Liegenschaftskataster}

Die Produkte der Produktklasse Daten Liegenschaftskataster sind einzelne Dateien, die einen Teil des Liegenschaftskatasters abbilden. Die folgenden Dateiformate können ausgewählt werden:
\begin{description}
\item[NAS] Die \ac{NAS} ist das Datenaustauschformat von Alkis und umfasst neben den Fachobjekten auch Operationen zur Haltung von Bestandsdaten. \ac{NAS} basiert auf \ac{XML}, \ac{GML} und \ac{WFS} \autocite[vgl.][]{sachsen-nas}.

\item[DXF] Das \ac{DXF} ist ein Dateiformat das von dem Unternehmen Autodesk entwickelt wurde und in \ac{CAD}-Applikationen (z.B. AutoCAD) als vektorbasierte Bilddatei Verwendung findet.
\ac{DXF} wurde mit dem Hintergrund entwickelt, dass es möglich sein soll AutoCAD Dokumente mit Programmen öffnen zu können, die nicht von Autodesk entwickelt wurden. Aus diesem Grund ist \ac{DXF} eine ASCII-Textdatei \autocite[vgl.][]{fileinfo-dxf}.

\item[GEOTIFF] \enquote{Ein GeoTIFF ist eine spezielle Form eines TIFF-Bildes, also ein Dateiformat zur Speicherung von Bilddaten (Dateinamenserweiterung .geotiff, oft auch nur .tif). Da das TIF-Format eine verlustfreie Speicherung zulässt, eignet es sich gut zur Verarbeitung von geographischen Daten, da es bei Satelliten- und Luftbildern bzw. anderen Rasterdaten oft auf hohe Abbildungsgenauigkeit ankommt."'} \autocite{wiki-geotiff} Weiterhin sind die \enquote{Bildinformationen [.] in jedem Programm darstellbar, welches den normalen TIFF-Standard unterstützt.} \autocite{wiki-geotiff}
\end{description}

Die Produkte der Produktklasse Standardausgaben WuNDa-Berichte können der Tabelle \ref{tab-daten-kataster} entnommen werden.
\begin{longtable}{|p{0.3\textwidth}|p{0.642\textwidth}|}
	\caption{Produkte der Produktklasse Daten Liegenschaftskataster} \label{tab-daten-kataster} \\
	\hline 
	\head{Produktname}  & \head{Beschreibung} \tabularnewline
	\hline 
	\endhead	
				
	\acs{NAS}-Daten (mit Eigentümern)
	&
	Vollständiger \acs{NAS}-Datensatz, wird als \acs{XML}-Datei ausgegeben. \\
	\hline
	\acs{NAS}-Daten (ohne Eigentümer)
	&
	\acs{NAS}-Datensatz ohne Eigentümer, wird als \acs{XML}-Datei ausgegeben. \\
	\hline
	\acs{NAS}-Daten (nur Punkte)
	&
	\acs{NAS} Datensatz nur Punkte, wird als \acs{XML}-Datei ausgegeben. \\
	\hline
	\acs{DXF}-Daten (Stadtgrundkarte kommunal)
	&
	Die Stadtgrundkarte (kommunal) als \acs{DXF}-Datei.  \\
	\hline
	GEOTIFF-Daten (Stadtgrundkarte kommunal)
	&
	Die Stadtgrundkarte (kommunal) als GEOTIFF-Datei. \\
	\hline	
\end{longtable} 

\section{Beleuchtung typischer Workflows der externen Benutzer}
\begin{itemize}
	\item Um einen noch besseren Einblick in die Arbeit zu erhalten werden typische Arbeitsschritte der externen Benutzer aufgezeigt
	\item typische Workflows aufzeichnen...
	\item Aufzeigen dass Arbeitsaufwand vorhanden ist und Rationalisierung tatsächlich nicht sofort durchgeführt werden kann
\end{itemize}

\subsection{Teilungsvermessung}
Ein solcher Workflow ist die Teilungsvermessung, die durch einen \ac{ÖbVI} durchgeführt wird. Durch die Teilungsvermessung wird ein neues Grundstück aus einem abgetrennten Teil eines bestehenden Grundstücks erstellt. Die folgende Beschreibung wurde nach \autocite{klein-vermessung} und \autocite{jungemann-alltag} erstellt.

Die Teilungsvermessung kann aus Sicht des \ac{ÖbVI} in drei Teile zerlegt werden:
\begin{itemize}
	\item Messungsvorbereitung
	\item Örtliche Vermessungsarbeiten
	\item Auswertung
\end{itemize}

Auf jeden dieser drei Schritte wird im folgenden eingegangen.

\subsubsection{Messungsvorbereitung}

Nachdem der \ac{ÖbVI} die Auftragserteilung zur einer Teilungsvermessung durch den Eigentümer, Erwerber oder Bauträger erhalten hat, bezieht er die erforderlichen Katasterunterlagen vom Katasteramt und die Teilungsgenehmigung von der für das Baurecht zuständigen Baugenehmigungsbehörde.

Die zur Vermessung neu zu erstellenden Punkte müssen beim Katasteramt reserviert werden, so dass diese später eine eindeutige Kennung besitzen. Dieser Vorgang wird Punktnummernreservierung genannt. Weiterhin müssen die folgenden Katasterunterlagen bezogen werden:
\begin{itemize}
	\item Eigentümerangaben
	\item Flurkarte
	\item NAS-Bestandsdaten
	\item Fortführungsrisse
	\item AP-Karten
	\item AP-Übersichten
	\item Punktnummernübersichten
\end{itemize}
Sobald sämtliche Unterlagen vorhanden sind kann die eigentliche Vorbereitung der Teilungsvermessung beginnen. So müssen die zuvor erhaltenen NAS"=Bestandsdaten in ein CAD-System sowie in ein Berechnungsprogramm importiert werden.
\todo{folgendes Fachchinesisch verstehen}
\begin{itemize}
	\item Einlesen in Tachymeter und GPS
	\item Übernahme CAD- und Berechnungsauftrag auf Außendienstrechner
	\item CODE-Liste für die automationsgerechte Erfassung der Vermessungspunkte im Außendienst
\end{itemize}

%\begin{itemize}
%\item nach der Auftragserteilung
%\item erforderlichen Katasterunterlagen und Teilungsgenehmigung beim Kat.Amt und für das Baurecht zuständigen Baugenehmigungsbehörde
%\item Produkte sind: ...
%\item alles zusammen, vorbereitung
%\item Importieren der NAS Bestandsdaten CAD und Berechnungsprogramm
%\item Ausgabe von Koordinatendateien für den Außendienst
%\begin{itemize}
%\item Einlesen in Tachymeter und GPS
%\item Übernahme CAD- und Berechnungsauftrag auf Außendienstrechner
%\item CODE-Liste für die automationsgerechte Erfassung der Vermessungspunkte im Außendienst
%\end{itemize}
%\item dann erfolgt örtliche Vermessung
%\end{itemize}

\subsubsection{Örtliche Vermessungsarbeiten}

Die erste Aufgabe an Ort und Stelle ist das Untersuchen der alten Grenzen und das Aufsuchen der bereits vorhandenen Abmarkungen (Grenzsteine etc.).
Danach kann der Lageanschluss durchgeführt werden. Beim Lageanschluss handelt es sich um einen \enquote{Anschluss einer Vermessung an das amtliche Bezugskoordinatensystem, [dieser ]wird durch differenzielle Beobachtung von Navigationssatelliten realisiert} \autocite{bier-lage}.
Diese Beobachtung erfolgt über \acs{SAPOS}, dieses steht für \acl{SAPOS} und ermöglicht eine deutschlandweite Positionsbestimmung mittels Satelliten \autocite[vgl.][2]{sapos-prospekt}.
Die beim Katasteramt angeforderten AP-Karten und AP-Übersichten werden nur für Kontrollmessungen benötigt oder falls der Lageanschluss mittels \acs{SAPOS} nicht möglich ist. Dies ist z.B. der Fall falls noch keine Koordinaten der Grenzpunkte vorliegen und die alten Vorgängervermessungen rekonstruiert werden müssen \autocite[vgl.][]{wolff-gps}. 
Danach kann der Aufmaß der neuen Vermessungspunkte \todo{mit Codierung} erfolgen und 
die neue Grenze kann, wie gewünscht, durch neue Abmarkungen gekennzeichnet werden um den neuen Grenzverlauf festzulegen. Zu Schluss werden die Rohmessdaten zur weiteren Verarbeitung ausgegeben.

%\begin{itemize}
%\item untersuchen der alten Grenzen
%\item Lageanschluss über SAPOS (Normalfall) (AP-Karten und AP-Übersichten werden normalerweise nur für Kontrollpunkte benötigt, ggfls. für die örtliche Grenzuntersuchung.)
%\item aufsuchen der Abmarkungen
%\item neue Abmarkungen für neuen Grenzverlauf setzen
%\item Aufmaß der Vermessungspunkte mit Codierung
%\item Ausgabe der Rohmessdaten zur Weiterverarbeitung
%\end{itemize}

\subsubsection{Auswertung} \label{subsubsec:auswertung}

Nach der örtlichen Arbeit folgt die Auswertung der ermittelten Rohmessdaten. Hierzu gehört es die Protokolle z.B. die Risse anzufertigen und die Berechnungen z.B. der neuen Flächen durchzuführen. 

Der erste Schritt hierfür ist das Einlesen der Rohmessdaten. \todo{wohin?} Danach wird die Dokumentation gem. Anlage 6 erstellt. \todo{Welche Anlage?} Nach diesen zwei Schritten kann der Fortführungsriss und die Skizze zur Grenzniederschrift automatisch erzeugt wird. \todo{Magie?} Danach folgen Berechnungen:
\begin{itemize}
	\item Hierarchische Berechnung der Polaraufnahme
	\item Flächenhafte Ausgleichung
	\item VP Liste
\end{itemize}

Nachdem Fertigstellen der Berechnungen müssen die Daten aufbereitet werden ehe diese zurück an das Katasteramt gegeben werden können. An dieser Stelle muss zwischen der vorgesehen ALKIS-Vorgehensweise und der Zurzeit benutzten Vorgehensweise in Wuppertal unterschieden werden. Nach der Abgabe der Informationen ist die Teilungsvermessung abgeschlossen. \todo{Gehören die beiden Vorgehensweisen wirklich hierhin?}

\paragraph{ALKIS}

Laut \autocite{jungemann-alltag} ist mit ALKIS vorgesehen dass die Punktendaten in einer Erhebungsdatei der Form NAS-ERH Stufe 1 zurückgegeben werden können.
Bei NAS-Erhebungsdaten (NAS-ERH) Stufe 1 handelt es sich um einen Mindestumfang an Informationen der zu Vermessungspunkten bekannt sein muss \autocite[vgl.][]{bezk-nas-erh}.

Ehe die Datei erstellt werden kann muss darauf geachtet werden, dass jedes Katasteramt in \ac{NRW} andere ALKIS-Datenbestände benutzen kann. Dies kann an der Attributart "`abmarkung\_Marke"' zur Objektart AX\_Grenzpunkt verdeutlicht werden. Für diese Attributart gibt es im  Grunddatenbestand NRW drei mögliche Werte:
\begin{description}
\item[1000] Marke, allgemein
\item[9500] Ohne Marke
\item[9998] nach Quellenlage nicht zu spezifizieren 
\end{description}
Das Maximalprofil NRW lässt allerdings weitere Werte zu, so z.B. "`Stein, Grenzstein"' (1110) oder "`Rohr mit Schutzkappe"' (1201) \autocite[vgl.][]{bezk-schluessel}. Kennt das Katasteramt die benutzten Werte nicht so müssen diese umgeschlüsselt werden, dies bedeutet dass im Zweifelsfall einer der drei Werte des Grunddatenbestandes NRW benutzt wird. 

\todo{Automatisierte Umsetzung der bürospezifischen Codierung in ALKIS Attribute ???}

Nach dem Abschließen der beiden Schritte kann die Erhebungsdatei nach NAS-ERH Stufe 1  erstellt werden und an das Katasteramt weitergegeben werden. 

\paragraph{Wuppertal}
Im Augenblick ist es nur möglich die NAS-Dateien der Stadt Wuppertal zu beziehen. Demnach funktioniert das, oben beschriebene, Zurückgeben der NAS-Erhebungsdaten nicht. Aus diesem Grund erstellt der \ac{ÖbVI} Dokumente zu seinen ausgewerteten Daten und den ermittelten Punkten und gibt diese an das Katasteramt weiter. Dort müssen diese manuell von Beamten mittels des 3A Editors eingepflegt werden. Bei dem 3A Editor handelt es sich um eine Anwendung von AED-SICAD, mit der es möglich ist ALKIS-Daten fortzuführen \autocite[vgl.][]{sicad-3a}.


%\begin{itemize}
%\item Im Büro werden nun die Protokolle (Risse) und Berechnungen (neue Flächen) angefertigt
%\item Einlesen der Rohmessdaten
%\item Erstellung der Dokumentation gem. Anlage 6
%\item Automatisierte Erzeugung von Fortführungsriss und Skizze zur Grenzniederschrift
%\item Berechnung
%\begin{itemize}
%\item Hierarchische Berechnung der Polaraufnahme
%\item Flächenhafte Ausgleichung
%\item VP Liste
%\end{itemize}
%\item Mindestumfang der Erhebungsdaten für neue Punkte in der NAS-ERH Stufe 1
%\item Unterschiedliche ALKIS-Datenbestände bei den Katasterämtern in NRW am Beispiel der Attributart abmarkung\_Marke zur Objektart AX\_Grenzpunkt
%\item Die Messung wird nun dem Katasteramt eingereicht und in die öffentlichen Karten und Verzeichnisse übernommen.
%\end{itemize}

\section{Aktuelle Unterstützung der Workflows durch das System}
\todo{Bisher nur ein Workflow beschrieben. Deshalb stimmt das untere?}

In diesem Abschnitt wird beschrieben wie \ac{WuNDa} die oben beschriebenen Workflows prinzipiell unterstützt.
Um nicht auf sämtliche Details der einzelnen Workflows einzugehen, werden diese in drei Phasen abstrahiert:
\begin{itemize}
 \item Beschaffung
 \item Verarbeitung
 \item Fortführung
\end{itemize}
Weiterhin wird sich vor allem auf die Schritte konzentriert die schlussendlich zum Erstellen der Abrechnungen führen. Somit wird hier nicht beschrieben wie jedes einzelne Produkt in in \ac{WuNDa} erstellt werden kann.

\subsection{Beschaffung}
\ac{WuNDa} unterstützt die externen Benutzer bei der Beschaffung der, für seinen Arbeitsschritt benötigten, Produkten. Durch die Selbstentnahme der Produkte spart er sich das Anfragen der Produkte beim Katasteramt, die vorher die eigentliche Beschaffung übernehmen musste. Mittels der Selbstentnahme können die Produkte die im Abschnitt \ref{sec:produktliste} aufgelistet sind bezogen werden.
Die Benutzer können je nach Produkt verschiedene Einstellungen treffen. So können Sie etwa einen Bereich in der Karte oder Objekte auswählen, die in dem jeweiligen Produkt angezeigt werden.

\begin{figure}[htbp]
	\centering
	\incgraph{width=0.9\textwidth}{./rimg/alkis-print-dialog.png}
	\caption{ALKIS-Druckdialog}
	\label{fig:alkis-druck}
\end{figure}

Die Selbstentnahme soll am sogenannten ALKIS-Druckdialog (siehe \myfig{fig:alkis-druck}) exemplarisch vorgeführt werden.
Nach dem Öffnen des Dialogs kann der Benutzer die Produktklasse auswählen in diesem Fall ist dies "`Gdb-NRW-Amtlich"' welche eine Teilmenge der ALKIS-Standardausgaben darstellt. 
Je nach dem welche Produktklasse ausgewählt wurde, ändern sich auch die Auswahlmöglichkeiten der Produkte. In diesem Fall wurde eine farbige Liegenschaftskarte ausgewählt.
Eine weitere wichtige Eingabe ist das Flurstück-Feld, hier können ein oder mehrere Flurstücke per drag-and-drop reingezogen werden, diese werden dann später im Produkt dargestellt.
Weitere Eingabefelder sind das Format (z.B. DINA4), den  Maßstab (hier 1:500), die Auftragsnummer und ein Kommentarfeld. Die Option "`Drehwinkel vorschlagen"' beeinflusst die Rotation des später angezeigten Druckbereiches.

\begin{figure}[htbp]
	\centering
	\incgraph{width=0.9\textwidth}{./rimg/alkis-auswahl.png}
	\caption{ALKIS-Druckdialog}
	\label{fig:alkis-auswahl}
\end{figure}

Bestätigt der Benutzer mit "`Ok"' seine Eingabe so gelangt er zur Karte in \ac{WuNDa}, wobei der Druckbereich angezeigt wird (siehe \myfig{fig:alkis-auswahl}).
Dieser Druckbereich befindet sich über den zuvor ausgewählten Flurstücken und kann verschoben, in der Größe verändert und rotiert werden.
Mit einem Doppelklick auf den Druckbereich gelangt er zum Downloadprotokoll (siehe \myfig{fig:alkis-protocol}).  

\begin{figure}[htbp]
	\centering
	\incgraph{width=0.9\textwidth}{./rimg/alkis-download-protocol.png}
	\caption{ALKIS-Druckdialog}
	\label{fig:alkis-protocol}
\end{figure}

Das Downloadprotokoll hat mehrere Aufgaben, so 
\begin{itemize}
\item nimmt es weitere Eingaben entgegen,
\item es zeigt die fällige Gebühr des Downloads an,
\item und der Benutzer kann die Gebühr bestätigen.
\end{itemize}

Die Eingaben bestehen aus dem Pflichtfeld der Geschäftsbuchnummer, dem optionalen Feld der Projektbezeichnung und der Auswahl des Verwendungszwecks.
Für die Abrechnungen ist der Verwendungszweck besonders relevant, da dieser verschiedene Rabatte nach sich ziehen kann. Mehr Informationen zum Zustandekommen der Rabatte kann dem Abschnitt \ref{subsec:gebuehren} entnommen werden.

Im Berechnungsfeld kann der Benutzer nachvollziehen wie die Gebühr zu seinem Download zustande kommt. Hier wird das gewählte Produkt und dessen Grundpreis angegeben, woraufhin der Rabatt abgezogen wird, um schließlich die fällige Gebühr anzuzeigen.

Die auswählbare Verwendungszwecke eines \ac{ÖbVI} sind u.a. "`Vermessungsunterlagen (amtlicher Lageplan TS3)"' und "`Vermessungsunterlagen (hoheitliche Vermessung TS4)"'. In diesem Beispiel ist der Benutzer ein \ac{ÖbVI} der ein vermessungsrelevantes Dokument bezieht, somit fällt, nach Abzug der Rabatte, eine Gebühr von \EUR{0} an.

Mit dem Bestätigten des Dialoges akzeptiert der Benutzer die Gebühr und der Download des Produktes wird gestartet. Nach dem Abschluss des Downloads kann der Benutzer die heruntergeladene Datei wie gewohnt benutzen. Im Falle der Liegenschaftskarte wird eine PDF-Datei heruntergeladen, ein Beispiel hiervon ist auf Seite \pageref{fig:flurkarte} zu finden.

Zeitgleich mit dem Starten des Downloads wird dieser protokolliert. Dies bedeutet dass relevante Informationen zu dem Download abgespeichert werden. Der Abspeichern erfolgt in jedem Fall, also auch wenn die Gebühr \EUR{0} beträgt oder der Download fehlschlägt. Insbesondere der letzte Fall bereitet beim Erstellen der Abrechnung Probleme, wie dies dem Abschnitt \vref{subsec:erstellen_abrechnung} zu entnehmen ist, da diese Protokolle storniert werden müssen.
 
Kurzum der externe Benutzer erhält durch die Selbstentnahme die Möglichkeit die benötigten Produkte, von seinem Rechner aus, selbst zu beziehen und entlastet das Katasteramt in dem Sinne, dass die Bearbeitungszeit zum Erstellen der Produkte eingespart wird.

\subsection{Verarbeitung}
Bei der Verarbeitung der Produkte oder sonstigen weiteren Tätigkeiten unterstützt \ac{WuNDa} den externen Benutzer nicht.

\subsection{Fortführung}
Auch bei der Fortführung unterstützt \ac{WuNDa} den externen Benutzer nicht, obwohl dies ähnlich wie in Abschnitt \ref{subsubsec:auswertung} beschrieben denkbar wäre.

Welche Erweiterungen von \ac{WuNDa} denkbar und sinnvoll sind wird im Kapitel \todo{Ref} ausführlich beschrieben. Zu Schluss dieses Kapitels wird auf die Problemstellung aus Abschnitt \ref{sec:rationalisierung} eingegangen, demnach wird detaillierter beschrieben warum die geplante Rationalisierung nicht durchgeführt werden kann.

\begin{itemize}
	\item fachlicher IST-Zustand
	\item Der Arbeitsschritt xy und andere bereits durch das System unterstützt in dem der ext. Benutzer die Produkte aus seinem Büro heraus beziehen kann.
	\item Nach einem bestimmten Abrechnungsturnus erhält er die manuell erstellte Rechnung.
	\item Möglich da: Protokollierung der bezogenen Produkte funktioniert bereits
	\item Da sich die benötigten Änderungen vor allem auf das Erstellen der Rechnungen bezieht ändert sich für den ext. Benutzer nicht so viel.
\end{itemize}

\section{Hohe Bearbeitungszeit}

Bislang wurde angegeben, dass die geplante Rationalisierung auf Grund der hohen Bearbeitungszeit die durch die Selbstentnahme entsteht nicht durchgeführt werden kann. Um dies zu verdeutlichen wird im nächsten Abschnitt die Erstellung der Abrechnung beschrieben. 

\subsection{Manuelles Erstellen der Abrechnung} \label{subsec:erstellen_abrechnung}
Die bisherige Erstellung der Abrechnungen an die externen Benutzer beschrieben. Die Abrechnungen werden von einem Beamten des Katasteramts nach jedem Abrechnungsturnus (z.B. Quartal, jährlich) manuell erstellt, wobei jeder externe Benutzer zu einen anderen Abrechnungsturnus gehören kann.
Weiterhin muss für jeden Benutzer eine eigene Abrechnung erstellt werden.

Zum Erstellen der einzelnen Abrechnungen müssen die Protokolle der Downloads aus der Datenbank bezogen werden um weiterverarbeitet zu werden.
Dies geschieht indem die Protokolle sämtlicher Kunden nach dem gewünschten Zeitraum gefiltert werden, um dann in eine Excel-Datei importiert zu werden.
Demnach befindet sich in einer Spalte dieser Datei Informationen wie z.B. der Benutzername, der Download-Name und die zu bezahlende Gebühr.

Im folgenden Arbeitsschritt muss der Beamte die einzelnen Protokolle für jeden Benutzer von dieser Excel-Datei in eine jeweils eigene Excel-Datei kopieren. Eine solche Excel-Datei repräsentiert die Abrechnung an die einzelnen Kunden des Katasteramtes.

Diese Vorgehensweise hat einige Nachteile. Ein Nachteil ist, dass durch den manuellen Kopiervorgang eine sehr lange Bearbeitungszeit anfällt. Diese wird in Zukunft noch verlängert, da ein Ausbau der externen Benutzer geplant ist, woraufhin die Größe der Anfangsdatei anwächst.

Ein weiterer Nachteil ist, dass die Protokolle nur auf Login-Ebene angelegt werden, d.h. es wird nur der Name gespeichert, mit dem sich der externe Benutzer in \ac{WuNDa} anmeldet.
Im Workflow \todo{xy hat man gesehen, dass die Polizei mehrere externe Benutzer besitzt}, die Abrechnung soll allerdings an den eigentlichen Kunden, also die Polizei, gestellt werden und nicht an die einzelnen Benutzer. Demnach muss sich der Beamte merken welche Benutzer zu welchem Kunden gehören um die entsprechende Abrechnung zu erstellen. Dies ist zwar machbar, verkompliziert den Arbeitsschritt ungemein.

Ein weiterer Nachteil ist, dass keine Bearbeitungen der Protokolle selbst vorgenommen werden kann. Somit werden bereits abgerechnete Downloads nicht als abgerechnet markiert. Weiterhin können keine Änderungen z.B. am Auftragsnamen des Protokolls vorgenommen werden und das Protokoll kann nicht als storniert markiert werden. Solche Bearbeitungen müssen vom Beamten selbst notiert und beim Erstellen der Abrechnungen berücksichtigt werden.

Kurzum die Vorgehensweise ist durch die manuelle Bearbeitung und den aufgeführten Nachteilen sehr aufwändig und fehleranfällig. Dieses Problem wird durch die Vergrößerung der Anzahl der externen Benutzer noch verschärft.

Aus diesen Gründen soll \ac{WuNDa} um eine Verwaltung der Downloadprotokolle, eine automatisierte Erstellung der Abrechnungen, sowie anderen verwandten Funktionalitäten erweitert werden, die den externen Benutzer und dem Katasteramt Zeitersparnisse bringen sollen. Die gewünschten Erweiterungen werden im nächsten Kapitel spezifiziert.


%Bisheriges Erstellen der Abrechnung.
%\begin{itemize}
%	\item für jeden Kunden in dem entsprechenden Abrechnungsturnus (z.B. Quartal)
%	\item DB -> Excel (vorgefiltert nach Zeit, kein Kundenfilter)
%	\item Kopieren von großer Excel zur kleinen
%	\item Abrechnung nicht gekennzeichnet
%	\item alles manuell
%	\item sehr lange Bearbeitungszeit
%	\item nur Login-Ebene, bis jetzt 1 Login = 1 Kunde
%	\item bis jetzt machbar, Kundenzahl steigt aber, d.h. Anfangs Excel wird immer größer
%\end{itemize}
%
%Fehlanfälligkeit:
%\begin{itemize}
%	\item Stornierungen, Änderungen der Buchungen muss vom Kunden gemeldet werden
%	\item kommt wegen fehlerhaftem Download häufiger vor
%	\item die muss notiert werden und Buchungen werden bei Abrechnungen ignoriert
%\end{itemize}









