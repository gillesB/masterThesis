\chapter{Beleuchtung der Arbeitsweise von externen Benutzer}

In diesem Kapitel wird die Arbeitsweise der verschiedenen externen Benutzer beleuchtet, um einen besseren Einblick in deren Arbeit zu erhalten.
Dies soll aufzeigen wie die externen Kunden bei ihren Workflows durch \ac{WuNDa} unterstüzt werden.

Um dies zu erreichen müssen in einem ersten Schritt einige fachliche Erläuterungen erfolgen. So werden u.a. einige Begriffe definiert und die Gebühren zu den Produkten erläutert. Daraufhin  werden die Produkte aufgelistet, die von den externen Benutzern bezogen werden können.

Danach werden einige Workflows von externen Benutzern beleuchtet, woraufhin überprüft wird ob und wie \ac{WuNDa} diese unterstützt.

%\begin{itemize}
%	\item Um einen besseren Einblick in der Arbeit der externen Benutzer zu erhalten werden die einzelnen Produkte aufgelistet
%	\item Danach folgt ein Beweis dass ein tatsächlicher Arbeitsaufwand durch viele Rechnungen besteht.
%\end{itemize}

\section{Fachliche Erläuterungen}
\begin{itemize}
\item Gebühren
\item was ist alkis
\item ÖbVI
\end{itemize}

\subsection{ÖbVI}
Damit hoheitliche Vermessungen nicht nur durch Verwaltungen durchgeführt werden können, kann der Staat private Vermessungsingenieure mit hoheitlichen Aufgaben beleihen. Solche Freiberufler werden \acf{ÖbVI} genannt und sind berechtigt Katastervermessungen durchzuführen. Weiterhin sind sie berechtigt, Tatbestände, die durch vermessungstechnische Ermittlungen am Grund und Boden festgestellt wurden, mit öffentlichem Glauben zu beurkunden. \autocite[vgl.][]{bdvi-oebvi} \todo{Letzter Satz ist Zitat}

\subsection{ALKIS und sein Datenmodell}
Das \acf{ALKIS} ersetzt das \acf{ALB} und die \acf{ALK}, indem es die Daten beider Systeme vereint. Die \acs{AdV} hat \acs{ALKIS} entwickelt und organisiert die Pflege des Datenmodells \autocite[vgl.][]{adv-alkis}.

Das Datenmodell von ALKIS ist stufenweise aufgebaut. Die oberste Stufe ist das ALKIS-Fachschema der AdV, dieses Fachschema bildet eine Grundlage mit der sämtliche Objekte, Attribute und Attributwerte aller Katasterbehörden Deutschlands abgebildet werden können.
Die folgenden Stufen sind Reduktionen dieses Fachschema, d.h. sie benutzen nur einen Teil dieser Grundlage. Weiterhin ist die jeweils folgende Stufe eine Reduktion der vorhergegangenen Stufe. Eine Übersicht der einzelnen Stufen kann der \myfig{fig:prod-stufen} entnommen werden. 

\begin{figure}[htbp]
	\centering
	\incgraph{width=0.8\textwidth}{./img/prod-stufen.png}
	\caption{Erweiterungen der Datenbestände}
	\label{fig:prod-stufen}
\end{figure}

In \ac{NRW} wurde das Maximalprofil NRW festgelegt, da hier einige Objekte nicht benötigt werden. Alle Informationen die im Maximalprofil NRW geführt werden gehören zum amtlichen Liegenschaftskataster.
Weiterhin dürfen in NRW nur diese Informationen innerhalb ALKIS geführt werden. Jedes Katasteramt darf frei wählen welche Informationen es aus dieser Stufe führen will, vorausgesetzt diese Informationen gehören nicht zu einer unteren Stufe.
Die nächste Stufe bildet der Grunddatenbestand NRW dieser ist landesweit gültig und muss von jedem Katasteramt in NRW geführt werden.
Die unterste Stufe bildet der Grunddatenbestand AdV, dieser ist bundesweit gültig und muss von jedem Katasteramt in Deutschland geführt werden.

Weiterhin haben die Katasterämter die Möglichkeit eine eigene Stufe zu diesem Stufenkonzept hinzuzufügen, diese gehört im Gegensatz zu den vorherigen Stufen nicht mehr zum amtlichen Liegenschaftskataster \autocite[vgl.][1-4]{bezk-grunddaten}. In Wuppertal ist dies ALWIS. \todo{Umformulieren und erweitern.}

\section{Auflistung der Produkte}
In diesem Abschnitt werden die Produkte aufgelistet, welche die externen Benutzer selbst beziehen können. Diese Produkte sind in vier Produktklassen eingeteilt:
\begin{itemize}
	\item ALKIS-Standardausgaben
	\item Standardausgaben WUP-kommunal
	\item WuNDa-Berichte
	\item Daten Liegenschaftskataster
\end{itemize}
Diese Produktklassen unterteilen die Produkte logisch und geben an von welcher Stelle diese definiert wurden. \todo{ausbauen und stimmt das?}

%\begin{itemize}
%	\item Die Produkte sind in verschiedene Produktklassen eingeteilt
%	\item Die Produktklassen sind wichtig weil...
%	\item Tabelle
%\end{itemize}
\subsection{ALKIS-Standardausgaben}

Die Produkte dieser Produktklasse sind in \acs{ALKIS} landesweit standardisiert und beschrieben oder es sind kommunale Erweiterungen davon.

In der Produktklasse wird weiterhin zwischen landesweit einheitliche Standardausgaben und kommunalen Produkten unterschieden. Bei den Standardausgaben handelt es sich um Auszüge aus dem Liegenschaftskataster deren Inhalt und Form landesweit festgelegt ist. Diese Ausgaben beinhalten nur Daten die im Grunddatenbestand NRW definiert sind \autocite[vgl.][8]{bezk-grunddaten}.

Bei den kommunalen Produkten handelt es sich um Auszüge aus dem Liegenschaftskataster, allerdings dürfen auch Daten verwendet werden, die nicht im Grunddatenbestand NRW definiert sind. Ein Beispiel eines kommunalen Produktes wäre eine modifizierte Standardausgabe, die mit Daten aus dem Maximalprofil NRW erweitert wurde. Weiterhin muss die Bezeichnung bzw. Beschriftung deutlich gekennzeichnet sein, so dass eine Verwechslung mit den Standardausgaben ausgeschlossen ist \autocite[vgl.][9]{bezk-grunddaten}. Dies wird durch die Anhänge \textit{(NRW)} bzw. \textit{(kommunal)} garantiert.

Die Produkte der Produktklasse ALKIS-Standardausgaben können der Tabelle \ref{tab-alkis-standard-start} \vpagerefrange{tab-alkis-standard-start}{tab-alkis-standard-end} entnommen werden.

\begin{longtable}{|p{0.3\textwidth}|p{0.642\textwidth}|}
	\caption{Produkte der Produktklasse ALKIS-Standardausgaben} \label{tab-alkis-standard-start} \\
	\hline 
	\head{Produktname} & \head{Beschreibung} \tabularnewline
	\hline 
	\endfirsthead	
		\caption{Produkte der Produktklasse ALKIS-Standardausgaben}  \\
			\hline
	\head{Produktname} & \head{Beschreibung} \tabularnewline
	\hline 
	\endhead	


	Flurstücksnachweis
	&
	"`Im "`Flurstücksnachweis"' werden alle flurstücksrelevanten Angaben beschrieben.
	Dies sind neben der "`katastertechnischen Bezeichnung des Flurstücks"' "`Gebietszugehörigkeit"', "`Lage"',
	"`Fläche"', "`Tatsächliche Nutzung"', "`Klassifizierung nach Straßen- oder Wasserrecht"', "`Gebäude"', "`Hinweise zum Flurstück"', "`Buchungsart"' und "`Buchung"'."' \autocite[269]{adv-alkis-erlaeuterung}
	\\ 
	\hline
	Flurstücks- und Eigentumsnachweis (NRW)
	&
	"`Der "`Flurstücks- und Eigentumsnachweis"' enthält alle Angaben des "`Flurstücksnachweises"'
	und darüber hinaus die "`Eigentümer"' ("`Name"', "`Adresse"',) und die "`Angaben zu den Berechtigten."' \autocite[269]{adv-alkis-erlaeuterung} \\
	\hline
	Flurstücks- und Eigentumsnachweis (kommunal)
	&
	Erweiterung des Flurstücks- und Eigentumsnachweis (NRW)\\
	\hline
	Bestandsnachweis (NRW)
	&
	"`Der "`Bestandsnachweis"' enthält alle Grundstücke, die auf einem Buchungsblatt gebucht sind."' \autocite[269]{adv-alkis-erlaeuterung}  \\
	\hline
	Bestandsnachweis (kommunal)
	&
	Erweiterung vom Bestandsnachweis (NRW) \\
	\hline
	Grundstücksnachweis (NRW)
	&
	"`Im "`Grundstücksnachweis"' wird das unter einer laufenden Nummer im Buchungsblatt geführte Grundstück beschrieben. Neben den im Grundbuch gebuchten Grundstücken können
	dies auch von der Buchungspflicht befreite Grundstücke (§ 3 Abs. 2 GBO) und Grundstücke
	sein, die noch nicht im Grundbuch gebucht sind (Verzeichnis der Bodenordnung ersetzt den
	Grundbuchnachweis)."' \autocite[269]{adv-alkis-erlaeuterung}\\
	\hline
	Flurkarte (NRW)
	&
	"`Der "`Auszug aus dem Liegenschaftskataster - Flurkarte NRW"'  ist eine landesweit einheitliche Ausgabe aus dem Grunddatenbestand. Er enthält alle Flurstücke, Gebäude und Nutzungsarten ohne weitere topographische Inhalte. Die Flurkarte NRW ist zum Beispiel für Beleihungszwecke ausreichend."' \autocite{wupp-alkis} \\
	\hline
	Stadtgrundkarte (kommunal)
	&
	"`Umfangreiche, das Stadtbild prägende Inhalte, die über den Grunddatenbestand hinausgehen - vorwiegend topographischer Art - sind zusätzlich in dem "`Auszug aus dem Liegenschaftskataster mit kommunalen Ergänzungen - Stadtgrundkarte enthalten"'.
	Dies sind zum Beispiel Fahrbahnbegrenzungen, Böschungen, Mauern, Bahngleise und Zäune, aber auch Balkone oder die Geschosszahl von Gebäuden. 
						
	Für alle Bürger und Fachkunden, die diese gewohnten Inhalte auch weiterhin benötigen, beispielsweise für Planungen, ist diese Ausgabe die richtige Wahl, denn sie entspricht weitgehend der bisherigen Wuppertaler Liegenschaftskarte."' \autocite{wupp-alkis} \\
	\hline			
	Stadtgrundkarte
	&
	Stadtgrundkarte ohne kommunale Ergänzungen. Wird nicht vertrieben. \\
	\hline
	Schätzungskarte (NRW)
	&
	Die Schätzungskarte ist "`eine auf der Grundlage der Flurkarte hergestellte Einzelkarte, in der zusätzlich die Schätzungsergebnisse der Reichsbodenschätzung eingetragen sind."'  \autocite{gg-schaetzungskarte}
						
	"`Die Kennzeichnung aller landwirtschaftlich nutzbaren Bodenflächen nach der Bodenbeschaffenheit durch Einteilung in Klassen (Bestandsaufnahme) und die Feststellung ihrer Ertragsfähigkeit aufgrund der natürlichen Ertragsbedingungen durch Ermittlung von Wertzahlen (eigentliche Schätzung). Gestartet im Jahr 1934. Auch heute noch wertvolle Datenquelle in ökologischen und Landwirtschaftsanwendungen von GIS."' \autocite{gg-reichsbodenschaetzung} \\
						
	\hline
	Amtliche Basiskarte (NRW)
	&
	"`Die Amtliche Basiskarte NRW (ABK NRW oder nur ABK) ist eine Übersichtskarte, die eine Verbindung zwischen der großmaßstäbigen Liegenschaftskarte und der Topographischen Karte 1:25000 (TK25) herstellt."' \todo{quelle:(wikipedia)} \\
	\hline
	Punktliste (TEXT)
	&
	ALKIS-Lagefestpunkte als Textdokument.
						
	"`Die Lagefestpunkte bilden in ihrer Gesamtheit das Lagefestpunktfeld mit einer Punktdichte je nach Bundesland von 1 Festpunkt auf 1 - 5 $km^2$. Das Lagefestpunktfeld dient heute noch überwiegend als Grundlage für alle amtlichen Vermessungen, insbesondere der Katastervermessung und der topographischen Landesaufnahme."' \autocite{adv-lagefestpunkte}\\
	\hline
	Punktnachweis (PDF)
	&
	ALKIS-Lagefestpunkte als PDF-Dokument. \\
	\hline
\end{longtable} 
\label{tab-alkis-standard-end}
	
\subsection{Standardausgaben WUP-kommunal}

Die ALKIS-Standardausgaben decken nicht alle Produkte ab, die benötigt werden. Aus diesem Grund musste die Stadtverwaltung Wuppertal weitere Produkte selbst definieren.
Bei diesen Eigendefinition werden keine ALKIS-Objekte dargestellt und somit handelt es sich nicht um Auszüge aus dem Liegenschaftskataster.
Der Inhalt der Produkte wird aus Rasterdatenbeständen bezogen und zeigen demnach Geodaten mit Bild.
Obwohl es sich nicht um ALKIS-Standardausgaben handelt sind einige dieser Produkte vermessungsrelevant und somit in bestimmten Situation für den ÖbVI gebührenfrei.
\todo{vermessungsrelevant definieren und erklären. Dann braucht das mit dem "für ÖbVI gebührenfrei" nicht immer erwähnt zu werden. Ausserdem ist dies vollständiger.}

Die Produkte der Produktklasse Standardausgaben WUP-kommunal können der Tabelle \ref{tab-wupp} entnommen werden.

%\begin{itemize}
%\item von der Stadtverwaltung selbst definiert
%\item keine ALKIS-Standardausgaben
%\item Inhalt sind keine ALKIS-Objekte
%\item Ausgaben von speziellen eigens definierten Rasterdatenbeständen
%\item Werden trotzdem benötigt (auch wenn kein Alkis)
%\item da teils vermessungsrelevant, teils gebührenfrei für ÖbVI
%\item teils eigene Preisvorschriften
%\item siehe Mail für mehr Details
%\end{itemize}

\begin{longtable}{|p{0.3\textwidth}|p{0.642\textwidth}|}
	\caption{Produkte der Produktklasse Standardausgaben WUP-kommunal} \label{tab-wupp} \\
	\hline 
	\head{Produktname}  & \head{Beschreibung} \tabularnewline
	\hline 
	\endfirsthead	
	\caption{Produkte der Produktklasse Standardausgaben WUP-kommunal}\\
	\hline 
	\head{Produktname}  & \head{Beschreibung} \tabularnewline
	\hline 
	\endhead
	\ac{DGK}
	&
	"`Die \ac{DGK} ist eine aus der Liegenschaftskarte/Stadtgrundkarte abgeleitete topografische Karte, welche die Deutsche Grundkarte (DGK5 im Maßstab 1:5.000) abgelöst hat."' \autocite{wupp-digitale-grundkarte} \\
	\hline
	Digitale Grundkarte mit Höhenlinien
	&
	Eine \ac{DGK} mit zusätzlichen Höhenlinien \\
	\hline
	Orthofoto
	&
	"`Orthofotos sind vereinfacht gesagt Luftbild und Karte in einem: Sie bieten kartenähnliche Genauigkeit und Maßstäbigkeit auf der Grundlage einer fotorealistischen Abbildung der Erdoberfläche."'  \autocite{wupp-orthofoto}\\
	\hline
	Orthofoto mit Katasterdarstellung
	&
	Orthofoto mit zusätzlicher Katasterdarstellung. \\
	\hline
	NivP-Übersicht
	&
	Übersicht über \acp{NivP}. \acp{NivP} werden auch Höhenfestpunkte genannt. "`Auf der Grundlage topographischer Karten 1:25000 geben NivP-Übersichten den großräumigen Überblick auf die Nummerierung und die Lage der NivP im Gelände an."' \autocite{adv-nivp} \\
	\hline
	AP-Übersicht
	&
	"`\aclp{AP} sind Lagefestpunkte, die das \acs{TP}-Netz unterster Ordnung (im Regelfall 4. Ordnung) verdichten und -- gemeinsam mit diesen Trigonometrischen Punkten -- einen koordinatenmäßigen Anschluss von Messungen an das übergeordnete Bezugssystem der Landesvermessung ermöglichen."' \autocite{wiki-aufnahmepunkt} \\
	\hline
	\ac{PNÜ}
	&
	Eine Karte, die die Punktnummer eines Flurstückes zeigt.\\
	\hline
	Stadtgrundkarte mit Höhenlinien
	&
	Stadtgrundkarte mit zusätzlichen Höhenlinien. \\
	\hline
\end{longtable} 	

\subsection{WuNDa-Berichte}
Die WuNDa-Berichte sind ebenfalls eigens definierte Produkte. Im Gegensatz zu den Standardausgaben WUP-kommunal enthalten sie allerdings keine Geodaten mit Bild, sondern sind Berichte die ein einzlnes Objekt beschreiben. Weiterhin sind sie ebenfalls nicht Bestandteil von ALKIS und auf Grund ihres Inhaltes vermessungsrelevant.

Die Produkte der Produktklasse Standardausgaben WuNDa-Berichte können der Tabelle \ref{tab-wunda-berichte} entnommen werden.
\begin{longtable}{|p{0.3\textwidth}|p{0.642\textwidth}|}	
	\caption{Produkte der Produktklasse WuNDa-Berichte} \label{tab-wunda-berichte} \\
	\hline 
	\head{Produktname}  & \head{Beschreibung} \tabularnewline
	\hline 
	\endfirsthead
	\caption{Produkte der Produktklasse WuNDa-Berichte} \\
	\hline 
	\head{Produktname}  & \head{Beschreibung} \tabularnewline
	\hline 
	\endhead
	AP-Karte
	&
	Formular zur Erhebung von ALKIS-Punkten (für \acp{AP} und sonstige Vermessungspunkte) \\
	\hline
	Nivellement-Punkt
	&
	Darstellung eines einzlnen \acf{NivP}. \\
	\hline
	Vermessungsriss
	&
	Darstellung eines einzlnen Vermessungsrisses. \\
	\hline
	Dokument Liegenschaftskatasterakte
	&
	\todo{Was ist das?} \\
	\hline
\end{longtable} 
	
\subsection{Daten Liegenschaftskataster}

Die Produkte der Produktklasse Daten Liegenschaftskataster sind einzelne Dateien, die einen Teil des Liegenschaftskatasters abbilden. Die folgenden Dateiformate können ausgewählt werden:
\begin{description}
\item[NAS] Die \ac{NAS} ist das Datenaustauschformat von Alkis und umfasst neben den Fachobjekten auch Operationen zur Haltung von Bestandsdaten. \ac{NAS} basiert auf \ac{XML}, \ac{GML} und \ac{WFS} \autocite[vgl.][]{sachsen-nas}.

\item[DXF] Das \ac{DXF} ist ein Dateiformat das von dem Unternehmen Autodesk entwickelt wurde und in \ac{CAD}-Applikationen (z.B. AutoCAD) als vektorbasierte Bilddatei Verwendung findet.
\ac{DXF} wurde mit dem Hintergrund entwickelt, dass es möglich sein soll AutoCAD Dokumente mit Programmen öffnen zu können, die nicht von Autodesk entwickelt wurden. Aus diesem Grund ist \ac{DXF} eine ASCII-Textdatei \autocite[vgl.][]{fileinfo-dxf}.

\item[GEOTIFF] "`Ein GeoTIFF ist eine spezielle Form eines TIFF-Bildes, also ein Dateiformat zur Speicherung von Bilddaten (Dateinamenserweiterung .geotiff, oft auch nur .tif). Da das TIF-Format eine verlustfreie Speicherung zulässt, eignet es sich gut zur Verarbeitung von geographischen Daten, da es bei Satelliten- und Luftbildern bzw. anderen Rasterdaten oft auf hohe Abbildungsgenauigkeit ankommt."' \autocite{wiki-geotiff} Weiterhin sind die "`Bildinformationen [.] in jedem Programm darstellbar, welches den normalen TIFF-Standard unterstützt."' \autocite{wiki-geotiff}
\end{description}

Die Produkte der Produktklasse Standardausgaben WuNDa-Berichte können der Tabelle \ref{tab-daten-kataster} entnommen werden.
\begin{longtable}{|p{0.3\textwidth}|p{0.642\textwidth}|}
	\caption{Produkte der Produktklasse Daten Liegenschaftskataster} \label{tab-daten-kataster} \\
	\hline 
	\head{Produktname}  & \head{Beschreibung} \tabularnewline
	\hline 
	\endhead	
				
	\acs{NAS}-Daten (mit Eigentümern)
	&
	Vollständiger \acs{NAS}-Datensatz, wird als \acs{XML}-Datei ausgegeben. \\
	\hline
	\acs{NAS}-Daten (ohne Eigentümer)
	&
	\acs{NAS}-Datensatz ohne Eigentümer, wird als \acs{XML}-Datei ausgegeben. \\
	\hline
	\acs{NAS}-Daten (nur Punkte)
	&
	\acs{NAS} Datensatz nur Punkte, wird als \acs{XML}-Datei ausgegeben. \\
	\hline
	\acs{DXF}-Daten (Stadtgrundkarte kommunal)
	&
	Die Stadtgrundkarte (kommunal) als \acs{DXF}-Datei.  \\
	\hline
	GEOTIFF-Daten (Stadtgrundkarte kommunal)
	&
	Die Stadtgrundkarte (kommunal) als GEOTIFF-Datei. \\
	\hline	
\end{longtable} 

\section{Beleuchtung typischer Workflows der externen Benutzer}
\begin{itemize}
	\item Um einen noch besseren Einblick in die Arbeit zu erhalten werden typische Arbeitsschritte der externen Benutzer aufgezeigt
	\item typische Workflows aufzeichnen...
	\item Aufzeigen dass Arbeitsaufwand vorhanden ist und Rationalisierung tatsächlich nicht sofort durchgeführt werden kann
\end{itemize}

\subsection{Teilungsvermessung}
Ein solcher Workflow ist die Teilungsvermessung, die durch einen \ac{ÖbVI} durchgeführt wird. Durch die Teilungsvermessung wird ein neues Grundstück aus einem abgetrennten Teil eines bestehenden Grundstücks erstellt. Die folgende Beschreibung wurde nach \autocite{klein-vermessung} und \autocite{jungemann-alltag} erstellt.

Die Teilungsvermessung kann aus Sicht des \ac{ÖbVI} in drei Teile zerlegt werden:
\begin{itemize}
	\item Messungsvorbereitung
	\item Örtliche Vermessungsarbeiten
	\item Auswertung
\end{itemize}

Auf jeden dieser drei Schritte wird im folgenden eingegangen.

\subsubsection{Messungsvorbereitung}

Nachdem der \ac{ÖbVI} die Auftragserteilung zur einer Teilungsvermessung durch den Eigentümer, Erwerber oder Bauträger erhalten hat, bezieht er die erforderlichen Katasterunterlagen vom Katasteramt und die Teilungsgenehmigung von der für das Baurecht zuständigen Baugenehmigungsbehörde.

Die zur Vermessung neu zu erstellenden Punkte müssen beim Katasteramt reserviert werden, so dass diese später eine eindeutige Kennung besitzen. Dieser Vorgang wird Punktnummernreservierung genannt. Weiterhin müssen die folgenden Katasterunterlagen bezogen werden:
\begin{itemize}
	\item Eigentümerangaben
	\item Flurkarte
	\item NAS-Bestandsdaten
	\item Fortführungsrisse
	\item AP-Karten
	\item AP-Übersichten
	\item Punktnummernübersichten
\end{itemize}
Sobald sämtliche Unterlagen vorhanden sind kann die eigentliche Vorbereitung der Teilungsvermessung beginnen. So müssen die zuvor erhaltenen NAS"=Bestandsdaten in ein CAD-System sowie in ein Berechnungsprogramm importiert werden.
\todo{folgendes Fachchinesisch verstehen}
\begin{itemize}
	\item Einlesen in Tachymeter und GPS
	\item Übernahme CAD- und Berechnungsauftrag auf Außendienstrechner
	\item CODE-Liste für die automationsgerechte Erfassung der Vermessungspunkte im Außendienst
\end{itemize}

%\begin{itemize}
%\item nach der Auftragserteilung
%\item erforderlichen Katasterunterlagen und Teilungsgenehmigung beim Kat.Amt und für das Baurecht zuständigen Baugenehmigungsbehörde
%\item Produkte sind: ...
%\item alles zusammen, vorbereitung
%\item Importieren der NAS Bestandsdaten CAD und Berechnungsprogramm
%\item Ausgabe von Koordinatendateien für den Außendienst
%\begin{itemize}
%\item Einlesen in Tachymeter und GPS
%\item Übernahme CAD- und Berechnungsauftrag auf Außendienstrechner
%\item CODE-Liste für die automationsgerechte Erfassung der Vermessungspunkte im Außendienst
%\end{itemize}
%\item dann erfolgt örtliche Vermessung
%\end{itemize}

\subsubsection{Örtliche Vermessungsarbeiten}

Die erste Aufgabe an Ort und Stelle ist das Untersuchen der alten Grenzen und das Aufsuchen der bereits vorhandenen Abmarkungen (Grenzsteine etc.).
Danach kann der Lageanschluss über \acs{SAPOS} durchgeführt werden. Beim Lageanschluss handelt es sich um einen "`Anschluss einer Vermessung an das amtliche Bezugskoordinatensystem,[ dieser] wird durch differenzielle Beobachtung von Navigationssatelliten realisiert"' \autocite{bier-lage}.
Diese Beobachtung erfolgt über \acs{SAPOS}, dieses steht für \acl{SAPOS} und ermöglicht eine deutschlandweite Positionsbestimmung mittels Satelliten \autocite[vgl.][2]{sapos-prospekt}. \todo{AP-Karten und AP-Übersichten werden normalerweise nur für Kontrollpunkte benötigt, ggfls. für die örtliche Grenzuntersuchung.}
Danach kann der Aufmaß der neuen Vermessungspunkte \todo{mit Codierung} erfolgen und 
die neue Grenze kann, wie gewünscht, durch neue Abmarkungen gekennzeichnet werden um den neuen Grenzverlauf festzulegen. Zu Schluss werden die Rohmessdaten zur weiteren Verarbeitung ausgegeben.

%\begin{itemize}
%\item untersuchen der alten Grenzen
%\item Lageanschluss über SAPOS (Normalfall) (AP-Karten und AP-Übersichten werden normalerweise nur für Kontrollpunkte benötigt, ggfls. für die örtliche Grenzuntersuchung.)
%\item aufsuchen der Abmarkungen
%\item neue Abmarkungen für neuen Grenzverlauf setzen
%\item Aufmaß der Vermessungspunkte mit Codierung
%\item Ausgabe der Rohmessdaten zur Weiterverarbeitung
%\end{itemize}

\subsubsection{Auswertung}

Nach der örtlichen Arbeit folgt die Auswertung der ermittelten Rohmessdaten. Hierzu gehört es die Protokolle z.B. die Risse anzufertigen und die Berechnungen z.B. der neuen Flächen durchzuführen. 

Der erste Schritt hierfür ist das Einlesen der Rohmessdaten. \todo{wohin?} Danach wird die Dokumentation gem. Anlage 6 erstellt. \todo{Welche Anlage?} Nach diesen zwei Schritten kann der Fortführungsriss und die Skizze zur Grenzniederschrift automatisch erzeugt wird. \todo{Magie?} Danach folgen Berechnungen:
\begin{itemize}
	\item Hierarchische Berechnung der Polaraufnahme
	\item Flächenhafte Ausgleichung
	\item VP Liste
\end{itemize}

%\begin{itemize}
%\item Im Büro werden nun die Protokolle (Risse) und Berechnungen (neue Flächen) angefertigt
%\item Einlesen der Rohmessdaten
%\item Erstellung der Dokumentation gem. Anlage 6
%\item Automatisierte Erzeugung von Fortführungsriss und Skizze zur Grenzniederschrift
%\item Berechnung
%\begin{itemize}
%\item Hierarchische Berechnung der Polaraufnahme
%\item Flächenhafte Ausgleichung
%\item VP Liste
%\end{itemize}
%\item Mindestumfang der Erhebungsdaten für neue Punkte in der NAS-ERH Stufe 1
%\item Unterschiedliche ALKIS-Datenbestände bei den Katasterämtern in NRW am Beispiel der Attributart abmarkung\_Marke zur Objektart AX\_Grenzpunkt
%\item Die Messung wird nun dem Katasteramt eingereicht und in die öffentlichen Karten und Verzeichnisse übernommen.
%\end{itemize}

\section{Aktuelle Unterstützung der Workflows durch das System}
\begin{itemize}
	\item fachlicher IST-Zustand
	\item Der Arbeitsschritt xy und andere bereits durch das System unterstützt in dem der ext. Benutzer die Produkte aus seinem Büro heraus beziehen kann.
	\item Nach einem bestimmten Abrechnungsturnus erhält er die manuell erstellte Rechnung.
	\item Möglich da: Protokollierung der bezogenen Produkte funktioniert bereits
	\item Da sich die benötigten Änderungen vor allem auf das Erstellen der Rechnungen bezieht ändert sich für den ext. Benutzer nicht so viel.
\end{itemize}









