\chapter*{Abkürzungen}
\addcontentsline{toc}{chapter}{Abkürzungen}
\begin{acronym}
	\acrodefplural{AP}[AP]{Aufnahmepunkte}
	\acro{AP}{Aufnahmepunkt}
	\acrodefplural{NivP}[NivP]{Nivellement Punkte}
	\acro{NivP}{Nivellement Punkt}
	\acro{TP}{Trigonometrische Punkte}
	\acro{DGK}{Digitale Grundkarte}
	\acro{PNÜ}{Punktnummerierungs\-übersicht}
	\acro{NAS}{Normbasierte Austauschschnittstelle}
	\acro{XML}{Extensible Markup Language}
	\acro{DXF}{Drawing Interchange Format}
	\acrodefplural{ÖbVI}[ÖbVIs]{Öffentlich bestellte Vermessungsingenieure}
	\acro{ÖbVI}{Öffentlich bestellter Vermessungsingenieur}
	\acro{WuNDa}{Wuppertaler Navigations- und Datenmanagementsystem}
	\acro{ALKIS}{Amtliche Liegenschaftskatasterinformationssystem}
	\acro{ALB}{Au\-to\-ma\-ti\-sie\-rte Liegenschaftsbuch}
	\acro{ALK}{Au\-to\-ma\-ti\-sie\-rte Liegenschaftskarte}
	\acro{AdV}{Arbeitsgemeinschaft der Vermessungsverwaltungen der Länder der Bundesrepublik Deutschland}
	\acro{SAPOS}{Satellitenpositionierungsdienst der deutschen Landesvermessung}
	\acro{GML}{Geography Markup Language}
	\acro{WFS}{Web Feature Service}
	\acro{CAD}{computer-aided design}
	\acro{NRW}{Nordrhein-Westfalen}
\end{acronym}
