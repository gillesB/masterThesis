\chapter{Realisierung} \label{ch:realisierung}

In diesem Kapitel wird die Realisierung der oben beschriebenen Erweiterungen erläutert.
Im ersten Teil wird der aktuelle technische Stand erläutert, anschließend wird die Realisierung der Erweiterungen beschrieben. 

\section{Aktueller technischer Stand}

In diesem Abschnitt soll der aktuelle technische Stand aufgezeigt werden, dies dient einerseits der Verständnis bezüglich des zweiten Abschnitts dieses Kapitels.
Andererseits soll somit verdeutlicht werden von welcher Basis ausgegangen wurde und welche Funktionalitäten implementiert werden mussten.

Die Informationen zu den folgenden Abschnitten wurden aus \autocite{cismet-cids} und \autocite{cismet-readme} bezogen.

\subsection{Cids-Umgebung}
In der bisherigen Arbeit wurde \ac{WuNDa} mehr oder weniger als einheitliches System betrachtet, aus technischer Sicht trifft dies jedoch nicht zu. Tatsächlich ist \ac{WuNDa} ein System, das sich aus mehreren Bestandteilen zusammensetzt und innerhalb der Cids-Umgebung realisiert wurde und diese auch erweitert. In dieser Umgebung gibt es drei Hauptbestandteile den Navigator, der Kernel und verschiedene Entwicklungswerkzeuge.
Eine Übersicht der Cids-Umgebung kann der Abbildung \vref{fig:cids-komponenten} entnommen werden. Diese Hauptbestandteile sind in weitere Unterkomponenten eingeteilt. Die in der Abbildung \ref{fig:cids-komponenten} nicht ausgegrauten Komponenten sind für diese Arbeit relevant und werden im Folgenden detaillierter erläutert.

\begin{figure}[htb]
	\centering
	\incgraph{width=\textwidth}{./img/cids_components_modified.png}
	\caption{Cids Komponenten}
	\label{fig:cids-komponenten}
\end{figure}

Zunächst wird der Hauptzweck und der Aufbau des Cids-Umfelds kurz beschrieben. So ist dieses Entwicklungsumfeld darauf ausgelegt Informationssysteme, insbesondere \acp{GIS}, zu erstellen.
Dabei stehen verschiedene, häufig benötigte, Funktionalitäten bereits zur Verfügung. Dazu gehören eine Benutzerverwaltung, die Verwaltung der Berechtigungen die den Benutzern Zugriff auf Informationen erlauben bzw. verbieten, sowie das Anzeigen und Bearbeiten von Informationen. Weiterhin wird eine interaktive zweidimensionale Ansicht zu geographischen Daten, die sogenannte Cismap, bereit gestellt.

Durch das Cids-Umfeld kann ein verteiltes System erstellt werden, dessen Architektur auf dem Client-Server-Prinzip basiert, wobei es aus mehreren Clients, Server, sowie Datenbanken bestehen kann.
Diese Bestandteile können in verschiedenen Ausprägungen nebeneinander existieren.
In Wuppertal steht so z.B. neben dem \ac{WuNDa} auch die BELIS-Fachanwendung zur Verfügung. BELIS dient zur Verwaltung des städtischen Beleuchtungskataster.
Um die Logik der Systeme zu trennen und um eine Skalierbarkeit zu gewährleisten, besitzt jedes System einen eigenen Domain Server und eine eigene Integration Base.
Bei einer solchen Integration Base handelt es sich um eine Datenbank, in der neben den üblichen Daten auch Meta-Daten gespeichert werden.
Durch diese Unterteilung der Daten können in der Integration Base beliebige Objekte abgebildet werden, weiterhin können dadurch auch Beziehungen zwischen sogenannten Meta-Klassen und Objekten abgebildet werden.
Eine Integration Base erlaubt es zum Beispiel die Benutzer und Benutzergruppen eines Systems sowie deren Zugriffsrechte auf Objekte abzubilden.
\todo{ref auf abschnitt mit ABF. bildliche darstellung}
Bei einem Domain Server handelt sich um eine Schnittstelle zur Integration Base und hat die Aufgabe aus deren Daten konkrete Klasse und Objekte zu erstellen. Weiterhin ist der Domain Server auch für die Erstellung des Katalogs zuständig.
Eine Beschreibung des Katalogs kann im \autoref{subsubsec:katalog} gefunden werden.
In der Cids-Umgebung gibt es weiterhin die sogenannten Broker. Diese dienen den Clients als Proxy zu den einzelnen Domain Servern. Sie verstecken somit die Komplexität des verteilten Systems, indem sie die Anfragen der Clients an die entsprechenden Domain Server weiterleiten.
Darüber hinaus gibt es noch den Registry Server. Dieser gewährleistet das Zusammenspiel der einzelnen Server, indem er z.B. die Namensauflösung für die restlichen Server bereitstellt.

\subsection{Navigator}
Beim Navigator handelt es sich um eine erweiterbare graphische Benutzeroberfläche für \acp{GIS}. So ist der Client von \ac{WuNDa} eine Erweiterung des Navigators. Die für diese Arbeit wichtigsten Unterkomponenten sind der Katalog, die Suche sowie der Renderer und der Editor.

\subsubsection{Katalog} \label{subsubsec:katalog}
Beim Katalog handelt es sich um die Hauptnavigationsmöglichkeit des Navigators.
Dieser Baum wird in der Integration Base mittels Meta-Daten beschrieben und wird bei Bedarf erstellt und im Navigator angezeigt.
Dieses Erstellen erfolgt dynamisch und ermöglicht es sämtliche Unterknoten, bis hin zur Objektebene, aus der Datenbank zu beziehen und zu erzeugen.
Um den Aufbau des Katalogs mittels eines Beispiels zu verdeutlichen, kann der Katalog eines \ac{ÖbVI} in \ac{WuNDa} der \myfig{fig:katlog-oebvi} entnommen werden.
Bei dem Knoten \ac{ALWIS} handelt es sich um einen sogenannten dynamischen Knoten.
Ein solcher Knoten besitzt eine SQL-Querie, mit der seine Unterknoten ermittelt und erstellt werden können.
Diese Unterknoten können ebenfalls solche Querie besitzen, so dass diese wiederum eigene Unterknoten mit eigenen Queries besitzen.
Auf diese Weise kann der Baum dynamisch gefüllt werden, wobei seine Blätter, in der Regel, konkrete Objekte darstellen.
Die Vorteile dieser Herangehensweise sind die Aktualität des Katalogs, da dieser zu jedem Zeitpunkt neu erstellt werden kann, sowie das Anzeigen sämtlicher relevanten Kategorien und Objekte.

\begin{figure}[htb]
	\centering
	\incgraph{width=0.8\textwidth}{./rimg/gui-alwis-katalog.png}
	\caption{WuNDa-Katlog eines ÖbVIs}
	\label{fig:katlog-oebvi}
\end{figure}

\subsubsection{Suche}
Bei der Suche handelt es sich um eine Schnittstelle zum Kernel, die das Suchen von Objekten ermöglicht.
Diese Suche kann an verschiedene Anwendungsfälle angepasst werden.
Eine Suche kann aus der Eingabe von Suchkriterien in einer Maske bestehen, woraufhin nach passenden Objekten gesucht wird.
Eine andere Suche kann die Karte (cismap) Suchkriterium miteinbeziehen. Ein Beispiel einer solchen Suche wäre das Festlegen eines Bereiches in der Karte, woraufhin nach sämtlichen Objekte, einer bestimmten Klasse, gesucht wird, die sich in diesem Bereich befinden.

\subsubsection{Renderer und Editor}
Die Renderer und Editoren wurden bereits in \autoref{sec:gewuenschte-erweiterungen} beschrieben und werden deshalb hier nicht weiter erläutert.

\subsection{ABF}
Der \ac{ABF} ist ein Werkzeug, mit dem die Meta-Daten der Integration Bases visualisiert und bearbeitet werden können.
Mit ihm können Klassen (vergleiche \autoref{sec:gewuenschte-erweiterungen}) und ihre Attribute, sowie Benutzer, Benutzergruppen und deren Rechte angelegt und verwaltet werden.
Weiterhin kann der Katalog hier erweitert und bearbeitet werden.

\subsubsection{Rechte}
In diesem Unterabschnitt wird kurz auf das Rechtesystem der Cids-Umgebung eingegangen. In dieser Arbeit werden zwei Arten von Rechte benötigt. Die Lese- und Schreibrechte auf Klassen, Attribute und Knoten des Katalogs sowie die Aktionsrechte.

\paragraph{Rechte auf Objekte}
Für jede Klasse, Attribut oder Knoten lassen sich drei Standardverhalten festlegen, wobei immer ein Verhalten benutzt werden muss.
Damit nicht für jedes Objekt ein Verhalten explicit gesetzt werden muss, wurde die Fallback Strategie eingeführt.
Diese Strategie ermöglicht es z.B. ein Verhalten als Standard-Server-Verhalten festzulegen, woraufhin jede Klasse ohne explizit gesetztes Verhalten auf dieses zurückgreift.
Die verfügbaren Standardverhalten sind:
\begin{description}
\item[Standard] Besitzt ein Objekt dieses Verhalten so bedeutet dies, dass jeder Benutzer lesen und niemand schreiben darf.
\item[Wiki] Bei diesem Standardverhalten darf jeder lesen und schreiben.
\item[Secure] Bei diesem Standardverhalten darf niemand lesen und schreiben
\end{description}
Für jedes Objekt können diese Rechte für verschiedene Benutzergruppen erweitert oder eingeschränkt werden.
Für eine Klasse mit dem Verhalten "`Secure"' kann z.B. festgelegt werden, dass die Administratoren Lese- und Schreibrechte besitzen.
Im Gegensatz hierzu kann einer Benutzergruppe für eine Klasse mit dem Verhalten "`Standard"', die Leserechte entzogen werden. \autocite[vgl.][]{cismet-workshop}

\paragraph{Aktionsrecht}
In manchen Fällen sind die Rechte auf Objekte zu grobkörnig und es müssen feinere Einschränkungen definiert werden können.
In diesen Fällen können die Aktionsrechte, auch Action Tags genannt, benutzt werden.
Ein Aktionsrecht besitzt jeweils einen eindeutigen Namen sowie eine Liste mit Besitzern.
In dieser Liste können sich ganze Benutzergruppen oder einzelne Benutzer aus einer Benutzergruppe befinden.

Ein Aktionsrecht kann benutzt werden um, im laufenden Programm, zu überprüfen, ob der aktuell angemeldete Benutzer das Recht hat eine bestimmte Aktion durchzuführen.
Für jede solche Aktion muss es demnach jeweils ein solches konfiguriertes Aktionsrecht geben.
Weiterhin hat der Benutzer das Recht die Aktion durchzuführen, falls er ein Besitzer des dazugehörigen Aktionsrechtes ist.
Zu solchen Aktionen gehört zum Beispiel die Anzeige bestimmter Suchen oder das Aktivieren/Deaktivieren von Benutzungselementen in einem Renderer.

\subsection{JPresso}
JPresso ist ein ETL (Extraction, Transaction \& Loading)-Werkzeug das eine graphische Benutzeroberfläche besitzt und es ermöglicht verschiedene Datenbestände in eine Datenbank zu importieren.
Bei diesem Importieren können die Zieltabellen und Spalten für die Datenquellen frei gewählt werden und es können beliebige Datenmanipulationen vorgenommen werden.
Weiterhin bietet JPresso eine Normalisierungsfunktion an.
Durch diese Funktion wird die Verbindung zwischen verschiedenen, bereits bestehenden Tabellen oder Datenquellen aufgedeckt und sie werden automatisch mittels ihrer entsprechenden ID referenziert. \autocite[vgl.][]{cismet-jpresso}

%\begin{itemize}
%	\item Vor der Realisierung wurde der technische IST-Zustand ermittelt.
%	\item Dabei wurde festgestellt, dass folgendes bereits vorhanden ist:
%	\begin{itemize}
%		\item Cids- System (Renderer/Editoren, DB-Modell, Rechte...)
%		\item GUI: Buchungen erstellen
%		\item ...
%	\end{itemize}
%	\item Auf dieser Basis wurde aufgebaut um die Realisierung durchzuführen.
%\end{itemize}


\section{Technische Realisierung}
Dieser Abschnitt beschreibt die Realisierung der in \autoref{ch:spezi} beschriebenen gewünschten Erweiterungen in \ac{WuNDa}.

\subsection{Datenbankmodell}
Um die in \autoref{ch:spezi} beschriebenen Anforderungen zu realisieren, muss in einem ersten Schritt das Datenbankmodell der verwendeten Datenbank angepasst werden.
Dies wird zum einen benötigt da die, durch die neuen Erweiterungen, anfallenden Daten abgespeichert werden müssen. Zum anderen müssen aus den Daten, die im Client benötigten, CidsBeans erstellt werden können. Dies ist notwendig, da die verschiedenen Erweiterung quasi alle auf den Funktionalitäten der CidsBeans aufbauen.

Durch die objektrelationale Abbildung die zwischen den Tabellen der Datenbank und den CidsBeans passiert wird im Folgenden von Klassen geredet.
Dabei sind auch die Tabelle der Datenbank und ihr Aufbau gemeint, da deren Struktur nicht oder nur geringfügig von der Struktur der erstellten Klassen abweicht.

\subsubsection{Ausgangslage}
In einem ersten Schritt wurden die bereits vorhandenen, benötigten Klassen betrachtet. Hierbei wurde festgestellt, dass einzig die Klasse der Buchungen, die \inline{BILLING}-Klasse vorhanden ist. Diese Klasse ist in der \myfig{fig:db-billing} abgebildet.
\todo{beschreiben}
\begin{figure}[htb]
	\centering
	\incgraph{scale=0.8}{./rimg/db_model_billing.png}
	\caption{Ausgangslage des Datenbankmodells}
	\label{fig:db-billing}
\end{figure}

\subsubsection{Weitere Schritte}
Wie bereits in \autoref{ch:spezi} beschrieben, waren sämtliche Anforderungen nicht von Anfang an bekannt, sondern wurden schrittweise erweitert. Dieser Prozess ist am deutlichsten am Datenbankmodell zu betrachten, welches mit den Anforderungen gewachsen ist.
Da die einzelnen Schritte für das Verständnis der Arbeit nicht relevant sind, werden diese nicht im Detail erläutert.
Um das Anwachsen des Datenbankmodells trotzdem zu verdeutlichen, wird eine der ersten Versionen kurz beschrieben, um anschließend die schlussendliche Version abzubilden und zu kommentieren.

Für den ersten Schritt wurden die anfänglich existierenden Anforderungen analysiert und die erste Version des Datenbankmodells wurde erstellt. 
Die \myfig{fig:db-two} zeigt einen recht frühen Stand des Datenbankmodells.
Dennoch sind die meisten Klassen die in den Spezifikationen aufgeführt werden bereits vorhanden und die Verbindungen zwischen den einzelnen Klassen bestehen ebenfalls bereits. 

\begin{sidewaysfigure}
	\centering
	\incgraph{width=\textwidth}{./rimg/db_model_two.png}
	\caption{Zweiter Stand des Datenbankmodells}
	\label{fig:db-two}
\end{sidewaysfigure}

\subsubsection{Letzter Stand}
Der letzte Stand des Datenbankmodell kann in \myfig{fig:db-final} betrachtet werden.
\todo{Schönere DB-Modell beschreibung}
Im Vergleich zu der vorherigen Version haben sämtliche Klassen das Präfix "`BILLING\_"' erhalten.
Die Buchungen-Klasse hat die Felder zum Speichern der letzten Änderung erhalten.
Weiterhin wurde ihr Feld "`gebuhr"' in "`netto\_summe"' umbenannt und sie hat die Felder "`brutto\_summe"' und "`mwst\_satz"' erhalten. Außerdem wurde das Feld "`abgerechnet"' hinzugefügt.
Die Kunden-Login-Klasse hat ein eigens Kontaktfeld erhalten.
Die Kunden-Klasse hat ein Feld "`name\_intern"' erhalten.
In sämtlichen Klassen wurden die Bezeichnungsfelder in "`name"' umbenannt. Somit erfüllen sie eine Namenskonvention und der Wert dieser Felder wird automatisch in Java bei der toString()-Methode ausgegeben.
Einige Felder wurden umbenannt, so dass auf den ersten Blick erkennbar ist, dass es sich um besondere Felder handelt. Dies ist der Fall bei Array-Felder oder 1:n Felder. Siehe z.B. "`produkte\_arr"' und "`benutzer\_n"' der Kundenklasse.
Weiterhin wurde die Kundengruppen-Klasse eingeführt, dies zeigt z.B. dass diese erst recht spät bei dem Ermitteln der Anforderungen aufgetreten ist. 
\begin{sidewaysfigure}
	\centering
	\incgraph{width=\textwidth}{./rimg/db_model_final.png}
	\caption{Letzter Stand des Datenbankmodells}
	\label{fig:db-final}
\end{sidewaysfigure}

\subsection{Import von neuen Daten}
Nachdem Erstellen des Datenbankmodell, musste die Datenbank entsprechend erweitert werden.
Eine gute Möglichkeit dies im Cids-Umfeld zu realisieren bietet der bereits erwähnte \ac{ABF}. Mit diesem Werkzeug wurden die benötigten Klassen angelegt und die entsprechenden Tabellen konnten in die Datenbank eingespielt werden.

Für die weitere Entwicklung der Funktionalitäten wurden Anfangsdaten benötigt, die vom R102 in Form von Excel-Dateien geliefert wurden.
Diese Datei enthielt die Informationen zu tatsächlichen Kunden und Kunden-Logins, wobei z.B. deren Abrechnungsturnus und Produkte angegeben war.
Diese Datei wurde in mehrere CSV-Dateien aufgeteilt, die dann in eine neue Datenbank eingespielt wurden.
Mit Hilfe von JPresso konnten die Daten aus dieser Datenbank in die eigentliche Datenbank importiert werden.
Weiterhin konnten mit Hilfe dieses Werkzeuges die Buchungen aus der "`BILLING"'-Tabelle in die neue "'BILLING\_BILLING"'-Tabelle überführt werden.
Bei diesem Vorgang wurde vermehrt auf die Normalisierungsfunktion von JPresso zurückgegriffen um die Fremdschlüssel zwischen den neuen Tabellen zu realisieren.

\subsection{Anpassung des Downloadprotokoll-Dialoges}
Durch die Änderung der Struktur der Buchungen-Klasse musste auch die Klasse des Downloadprotokoll"=Dialoges \inline{BillingPopup} angepasst werden.
Dabei sollte an der eigentlichen Funktionalität von \inline{BillingPopup}, die bereits im \autoref{subsec:beschaffung} beschrieben wurde, nichts geändert werden.
Lediglich die Graphische Oberfläche wurde dahingehend verändert, dass die Netto und Brutto-Summen der anfallenden Gebühren sowie der anfallende MwSt-Satz angezeigt werden
Ein Screenshot des überarbeiteten Downloadprotokoll-Dialoges kann in \myfig{fig:alkis-protocol-new} gefunden werden.

\begin{figure}[htb]
	\centering
	\incgraph{width=0.9\textwidth}{./rimg/alkis-download-protocol_neu.png}
	\caption{Downloadprotokoll-Dialog mit MwSt.-Satz}
	\label{fig:alkis-protocol-new}
\end{figure}

\subsection{Zeitliche Filter und Verwendungszweck-Filter}
Beim Betrachten der benötigten Filter in der Spezifikation fällt auf, dass ähnliche Filtermöglichkeiten in den verschiedenen Renderer und der Suche benötigt werden. Aus diesem Grund wurde der zeitliche Filter und Verwendungszweck-Filter als eigene \inline{JPanel} angelegt, so dass diese mehrfach verwendet werden können.

\subsubsection{Zeitlicher Filter}
Das eigenständige \inline{JPanel} \inline{TimeFilterPanel} gibt dem Benutzer die Möglichkeit nichts, einen Tag oder einen Zeitspanne auszuwählen.
Dabei werden ihm Schnellwahlmöglichkeiten für den aktuellen Tag, einen Monat oder ein Quartal angeboten.
Weiterhin kann er mittels der Angabe eines von-Datums und eines bis-Datums einen Zeitspanne festlegen.
Bei Bedarf kann die aktuelle Einstellung des Panels abgefragt werden, wobei der Rückgabewert zwei \inline{Date}-Variablen sind.
Sind beide Variablen \inline{null} so wurde keine Zeit ausgewählt.
Ist nur das von-Datum belegt so wurde ein einzelner Tag ausgewählt.
Sind beide Variablen belegt so wurde eine Zeitspanne festgelegt.

\begin{figure}[htb]
	\centering
	\incgraph{scale=0.8}{./rimg/gui-jxdatepicker.png}
	\caption{Datumsauswahl mit einer Schnellauswahl für das Jahr}
	\label{fig:jxdatepicker}
\end{figure}

Während dem Testen der Renderer kam die zusätzliche Anforderung auf, dass die Datumsauswahl jeweils eine Schnellauswahl für das Jahr benötigt.
Das gewünschte Result kann in der \myfig{fig:jxdatepicker} betrachtet werden.
Da die verwendete Komponente \inline{JXDatePicker} dies offiziell nicht anbietet musste ein Workaround gefunden werden, welches unter \autocite{so-jxdatepicker} gefunden wurde.
Um diesen Workaround umzusetzen musste ein sogenanntes \inline{TakeoffHook} implementiert werden. 
Dies ist eine Klasse die mittels des Lookup-Mechanismus beim Starten von \ac{WuNDa} gefunden und ausgeführt wird.
Somit wird beim Starten des Programmes das \inline{TakeoffHook} \inline{JXDatePickerHeaderTakeoff} ausgeführt, das folgende zwei Zeilen beinhaltet:
\begin{lstlisting}
UIManager.put(CalendarHeaderHandler.uiControllerID, "org.jdesktop.swingx.plaf.basic.SpinningCalendarHeaderHandler");
UIManager.put(SpinningCalendarHeaderHandler.ARROWS_SURROUND_MONTH, Boolean.TRUE);
\end{lstlisting}
In der ersten Zeile wird angegeben dass für die \inline{JXDatePicker} eine Kopfzeile verwendet wird, die eine Schnellauswahl des Jahres zulässt.
In der zweiten Zeile wird die Anordnung des rechten Pfeils geändert, dieser würde sich ansonsten ganz rechts außen befinden.
Damit die Kopfzeile einer \inline{JXDatePicker}-Instanz \inline{datePicker} entsprechend geändert wird, muss die folgende Zeile ausgeführt werden:
\begin{lstlisting}
datePicker.getMonthView().setZoomable(true)
\end{lstlisting}
In genau dieser Zeile liegt der Nachteil dieses Workaround, da das Zoomable-Attribut von \inline{JX\-MonthView}, eine Anzeige von Monaten, über die ein Datum gewählt werden kann, noch nicht richtig implementiert ist und keine Auswirkungen haben dürfte \autocite[vgl.][]{swingx-jxdatepicker}.
Dennoch wird hierdurch die gewünschte Kopfzeile angezeigt.

\subsubsection{Verwendungszweck-Filter}
Der Verwendungszweck-Filter ist ebenfalls ein eigenständiges \inline{JPanel} und wurde in der Klasse \inline{VerwendungszweckPanel} implementiert.
In diesem Panel werden die möglichen Verwendungszwecke untereinander angezeigt, wobei jedes eine Checkbox besitzt, mit der es an oder abgewählt werden kann. Weiterhin ist es nicht möglich sämtliche Verwendungszwecke abzuwählen.
Die Besonderheit dieses Panel besteht darin, dass die Verwendungszwecke nicht hardkodiert sind, sondern aus der \inline{billing.json} Datei geladen werden.
Diese JSON-Datei beinhaltet sämtliche Informationen die zu den Buchungen benötigt werden. So enthält sie neben den Verwendungszwecken auch die Gebühren und MwSt.-Sätze zu den einzelnen Produkten. 

\subsection{Suche nach Buchungen}
Beim Analyseren der Anforderungen ist ebenfalls aufgefallen, dass sowohl die Kunden- und Kundengruppenrenderer als auch der Suchdialog zu den Buchungen eine Möglichkeit benötigen um in der Datenbank nach Buchungen mit bestimmten Kriterien zu suchen.
Aus diesem Grund wurde auf die Such-Funktionalität der Cids-Umgebung zurückgegriffen und es wurde eine sogenannte ServerSearch implementiert.
Um diesen Mechanismus umzusetzen wurde die Klasse \inline{Cids\-Billing\-Search\-Statement} erstellt.
Diese erbt von der, bereits bestehenden, Klasse \inline{Abstract\-Cids\-Server\-Search} und besitzt somit die Grundfunktionalität zu einer solchen ServerSearch.
Um diese ServerSearch in den Renderern oder dem Suchdialog auszuführen müssen lediglich die gewünschten Kriterien einem Objekt der Klasse \inline{Cids\-Billing\-Search\-Statement} übergeben werden und die eigentliche Suche muss gestartet werden.
Daraufhin können die gefundenen Buchungen dort weiterverarbeitet und angezeigt werden.

Die Kriterien die \inline{Cids\-Billing\-Search\-Statement} entgegen nehmen kann, sind die Strings bezüglich zu der Geschäftsbuchnummer, der Projektbezeichnung und dem Kundenname.
Dabei wird jeweils nach Substrings gesucht und es können die beiden Wildcards '\_' und '\%' für ein einzelnes Zeichen bzw. für eine beliebige Anzahl von Zeichen benutzt werden.
Weiterhin kann ein Kunden-Login angegeben werden.
Ist dies der Fall dann wird nur nach solche Buchungen gesucht, die mit jenem Kunden-Login angelegt wurden.
Ein anderes Kriterium ist der Zeitbereich in denen die Buchungen erstellt werden mussten, sowie der Kostentyp, sprich ob die einzelne Buchung kostenpflichtig oder kostenfrei ist.
Weiterhin kann angegeben werden, ob stornierte oder abgerechnete Buchungen gefunden werden sollen.
Außerdem kann eine Liste mit Kunden und einem, optionalen, Abrechnungsturnus übergeben werden.
In dem Fall werden nur Buchungen von jenen Kunden mit eben diesem  Abrechnungsturnus gefunden.
Das einzige Pflichtkriterium ist eine Information bezüglich den Kunden. So muss ein Kundenname oder eine Liste mit Kunden angegeben werden.

\subsection{Kunden}
\subsubsection{Kundenrenderer}
Die Funktionalität des Kundenrenderers wurde bereits größtenteils in dem \autoref{subsubsec:kundenrenderer} beschrieben.
Der dort nicht beschriebene Knopf "`Ergebnisse anzeigen"' wurde hinzugefügt, damit die Suche bei Bedarf ausgeführt werden kann.
Dieser wurde während der Entwicklung hinzugefügt, da sich die erneute Suche bei jeder Änderung der Filter als unpraktikabel herausstellte.
Dies lag an der Dauer der Suche, die für eine solche Funktionalität zu ausgedehnt war, auch wenn diese, in den meisten Fällen, nur ein paar Sekunden beträgt.
Bei Betätigen des Knopfes wird ein Objekt der Klasse \inline{Cids\-Billing\-Search\-Statement} mit den verschiedenen Kriterien aus den Filtern gefüllt und die Suche wird ausgeführt.
Die Ergebnisse dieser Suche werden dann in der Tabelle angezeigt und eine entsprechende Zusammenfassung der Suchergebnisse wird über der Tabelle ausgegeben.
Weiterhin können die beiden Berichte Rechnungsanlage und Buchungsbeleg erstellt werden, wobei die im \autoref{subsec:berichte} aufgeführten Bedingungen gelten.
So dürfen externe Benutzer die Rechnungsanlage nicht erstellen und es kann ausgewählt werden, ob kostenfreie Buchungen in den Berichten angezeigt werden sollen oder nicht.
Ein Screenshot des Kundenrenderers kann in \myfig{fig:kundenrenderer} eingesehen werden.

\begin{figure}[htbp]
	\centering
	\incgraph{width=0.9\textwidth}{./rimg/gui-kunderenderer-real.png}
	\caption{Kundenrenderer}
	\label{fig:kundenrenderer}
\end{figure}
\subsubsection{Kundeneditor}
Der Kundeneditor, der in \myfig{fig:kundeneditor} dargestellt ist, ermöglicht es einen neuen Kunden anzulegen oder die Daten eines bereits bestehenden Kunden zu bearbeiten.
Bei den meisten Felder handelt es sich um gewöhnliche Texteingaben oder Auswahlmöglichkeiten.
Allerdings konnte bei den Felder für die Produkte, Kunden-Logins und Kundengruppen auf die Drag-and-Drop Funktionalität der Cids-Umgebung zurückgegriffen werden.
Somit können Objekte dieser Klassen aus dem Katalog in den Editor gezogen werden und die Beziehung zwischen den jeweiligen Objekten wird automatisch erstellt. 
\begin{figure}[htbp]
	\centering
	\incgraph{width=0.9\textwidth}{./rimg/gui-kundeneditor-real.png}
	\caption{Kundeneditor}
	\label{fig:kundeneditor}
\end{figure}

\subsection{Anlegen der Berechtigungen}


\begin{itemize}
	\item Realisiert wurde
	\begin{itemize}
		\item in der DB
		\item in der GUI
		\item ...
	\end{itemize}
\end{itemize}


\section{Vergleich Realisierung und Spezifikation}
\begin{itemize}
	\item Durch diese Realisierung ist es möglich das zu tun was in den Anforderungen beschrieben wurde
	\item D.h. durch die Rationalisierung kann durchgeführt werden
\end{itemize}