\chapter{Realisierung} \label{ch:realisierung}

In diesem Kapitel wird die Realisierung der oben beschriebenen Erweiterungen erläutert.
Im ersten Teil wird der aktuelle technische Stand erläutert, anschließend wird die Realisierung der Erweiterungen beschrieben. 

\section{Aktueller technischer Stand}

In diesem Abschnitt soll der aktuelle technische Stand aufgezeigt werden, dies dient einerseits der Verständnis bezüglich des zweiten Abschnitts dieses Kapitels.
Andererseits soll somit verdeutlicht werden von welcher Basis ausgegangen wurde und welche Funktionalitäten implementiert werden mussten. \todo{Quellen}

\subsection{Cids-Umgebung}
\ac{WuNDa} ist eine Erweiterung des Cids-Navigators. Bei dem Navigator handelt es sich um eine graphische Oberfläche eines Informationssystem und ist ein Hauptbestandteil der Cids-Umgebung. Die weiteren Hauptbestandteile dieser Umgebung sind  
der Kernel und verschiedene Entwicklungswerkzeuge. Eine Übersicht der Cids-Umgebung kann der Abbildung \vref{fig:cids-komponenten} entnommen werden. Diese Hauptbestandteile sind in weitere Unterkomponenten eingeteilt. Die in der Abbildung \ref{fig:cids-komponenten} nicht ausgegrauten Komponenten sind für diese Arbeit relevant und werden im Folgenden detaillierter erläutert.

\begin{figure}[htb]
	\centering
	\incgraph{width=\textwidth}{./img/cids_components_modified.png}
	\caption{Cids Komponenten}
	\label{fig:cids-komponenten}
\end{figure}

Zunächst wird der Hauptzweck und der Aufbau des Cids-Umfelds kurz beschrieben. So ist dieses Entwicklungsumfeld darauf ausgelegt Informationssysteme, insbesondere \acp{GIS}, zu erstellen.
Dabei stehen verschiedene, häufig benötigte, Funktionalitäten bereits zur Verfügung. Dazu gehören eine Benutzerverwaltung, die Verwaltung der Berechtigungen die den Benutzern Zugriff auf Informationen erlauben bzw. verbieten, sowie das Anzeigen und Bearbeiten von Informationen. Weiterhin wird eine interaktive zweidimensionale Ansicht zu geographischen Daten, die sogenannte Cismap, bereit gestellt.
Diese ist für den Implementierungsteil dieser Arbeit jedoch nicht relevant.

Technisch gesehen ist das Cids-Umfeld ein verteiltes System das aus mehreren Clients, Server, sowie Datenbanken bestehen kann.
Diese Bestandteile können in verschiedenen Ausprägungen nebeneinander existieren.
In Wuppertal steht so z.B. neben dem \ac{WuNDa}-Navigator auch die BELIS-Fachanwendung als Client zur Verfügung. BELIS dient zur Verwaltung des städtischen Beleuchtungskataster.
Um die Logik dieser beiden Anwendungen zu trennen und um eine Skalierbarkeit zu gewährleisten, besitzen beide einen eigenen Domain Server und ihre eigene Integration Base.
Bei einer solchen Integration Base handelt es sich um eine Datenbank mit zuzüglichen Meta-Daten.
Somit kann eine Integration Base beliebige Objekte speichern sowie die Beziehung zwischen diesen Objekten und den Meta-Daten abbilden.
Eine Integration Base erlaubt es zum Beispiel die Benutzer und Benutzergruppen eines Systems sowie deren Zugriffsrechte auf Objekte abzubilden.
\todo{ref auf abschnitt mit ABF. bildliche darstellung}
Bei einem Domain Server handelt sich um eine Schnittstelle zur Integration Base und hat die Aufgabe aus deren Daten konkrete Klasse und Objekte zu erstellen. Weiterhin ist der Domain Server auch für die Erstellung des Katalogs zuständig. \todo{ref setzen}
In der Cids-Umgebung gibt es weiterhin die sogenannten Broker. Diese dienen den Clients als Proxy zu den einzelnen Domain Servern. Sie verstecken somit die Komplexität des verteilten Systems, indem sie die Anfragen der Clients an die entsprechenden Domain Server weiterleiten.
Darüber hinaus gibt es noch den Registry Server. Dieser gewährleistet das Zusammenspiel der einzelnen Server, indem er z.B. die Namensauflösung für die restlichen Server bereitstellt.

\subsection{Navigator}
Beim Navigator handelt es sich um eine erweiterbare graphische Benutzeroberfläche für \acp{GIS}. Die für diese Arbeit wichtigsten Unterkomponenten sind der Katalog, die Suche sowie der Renderer und der Editor.

\subsubsection{Katalog}
Beim Katalog handelt es sich um die Hauptnavigationsmöglichkeit des Navigators.
Dieser Baum wird in der Integration Base mittels Meta-Daten beschrieben und wird bei Bedarf erstellt und im Navigator angezeigt.
Dieses Erstellen erfolgt dynamisch und ermöglicht es, dass im Katalog sämtliche Unterknoten aus der Datenbank, bis hin zur Objektebene, bezogen werden können.
Um sein Aufbau zu verdeutlichen wird in \myfig{fig:katlog-oebvi} der Katlog eines \ac{ÖbVI} in \ac{WuNDa} gezeigt.
Hierbei handelt es sich bei dem Knoten \ac{ALWIS} um einen dynamischen Knoten.
Dies bedeutet, dass der Knoten eine SQL-Querie besitzt, mit der seine Unterknoten ermittelt und erstellt werden können.
Die Unterknoten können ebenfalls solche SQL-Queries besitzen, so dass mehrere Ebenen im Katalog dynamisch erstellt werden können.
Die unterste Ebene stellt dabei, in der Regel, die konkreten Objekte dar.

\begin{figure}[htb]
	\centering
	\incgraph{width=0.8\textwidth}{./rimg/gui-alwis-katalog.png}
	\caption{WuNDa-Katlog eines ÖbVIs}
	\label{fig:katlog-oebvi}
\end{figure}

\subsubsection{Suche}
Bei der Suche handelt es sich um eine Schnittstelle zum Kernel, die das Suchen von Objekten ermöglicht.
Diese Suche kann an verschiedene Anwendungsfälle angepasst werden.
Der Standardfall wäre die Eingabe von Suchkriterien in eine Maske, woraufhin nach passenden Objekten gesucht wird.
Ein speziellerer Fall wäre das Einbeziehen der Karte als Suchkriterium. Ein Beispiel hiervon wäre das Festlegen eines Bereiches in der Karte, woraufhin nach sämtlichen Objekte, einer bestimmten Klasse, gesucht wird, die sich in diesem Bereich befinden.

\subsubsection{Renderer und Editor}
Die Renderer und Editoren wurden bereits in \autoref{sec:gewuenschte-erweiterungen} beschrieben und werden deshalb hier nicht weiter erläutert.

\subsection{ABF}
Der \ac{ABF} ist ein Werkzeug, mit dem die Meta-Daten der Integration Bases visualisiert und bearbeitet werden können.
Mit ihm können Klassen (vergleiche \autoref{sec:gewuenschte-erweiterungen}) und ihre Attribute, sowie Benutzer, Benutzergruppen und deren Rechte angelegt und verwaltet werden.
Weiterhin kann der Katalog hier erweitert und bearbeitet werden.

\subsubsection{Rechte}
In diesem Unterabschnitt wird kurz auf das Rechtesystem der Cids-Umgebung eingegangen. In dieser Arbeit werden zwei Arten von Rechte benötigt. Die Lese- und Schreibrechte auf Klassen, Attributen und Knoten des Katalogs sowie die Aktionsrechte.

\paragraph{Rechte auf Objekte}
Für jede Klasse, Attribut oder Knoten lassen sich drei Standardverhalten festlegen, wobei immer ein Verhalten benutzt werden muss. Die verfügbaren Standardverhalten sind:
\begin{description}
\item[Standard] Besitzt ein Objekt dieses Verhalten so bedeutet dies, dass jeder Benutzer lesen und niemand schreiben darf.
\item[Wiki] Bei diesem Standardverhalten darf jeder lesen und schreiben.
\item[Secure] Bei diesem Standardverhalten darf niemand lesen und schreiben
\end{description}
Für jedes Objekt können diese Rechte erweitert oder eingeschränkt werden. Für eine Klasse mit dem Verhalten Secure kann z.B. festgelegt werden dass die Administratoren Lese- und Schreibrechte besitzen.
Im Gegensatz hierzu kann einer Benutzergruppe für eine Klasse mit dem Verhalten Standard, die Leserechte entzogen werden. \todo{Quelle: Einführungspresi}

\paragraph{Aktionsrecht}
In manchen Fällen sind die Rechte auf Objekte zu grobkörnig und es müssen feinere Einschränkungen definiert werden können.
In diesen Fällen können die Aktionsrechte, auch Action Tags genannt, benutzt werden.
Ein Aktionsrecht besitzt jeweils einen eindeutigen Namen sowie eine Liste mit Besitzern.
In dieser Liste können sich ganze Benutzergruppen oder einzelne Benutzer aus einer Benutzergruppe befinden.

Ehe im laufenden Programm eine sicherheitsrelevante Aktion durchgeführt wird, kann abgefragt werden, ob der aktuell angemeldete Benutzer ein Besitzer eines bestimmten Aktionsrechts ist.
Ist dies der Fall, so kann die sicherheitsrelevante Aktion ausgeführt werden, andernfalls nicht.
Zu solchen Aktionen gehört zum Beispiel die Anzeige bestimmter Suchen oder das Aktivieren/Deaktivieren von Benutzungselementen in einem Renderer.


%\begin{itemize}
%	\item Vor der Realisierung wurde der technische IST-Zustand ermittelt.
%	\item Dabei wurde festgestellt, dass folgendes bereits vorhanden ist:
%	\begin{itemize}
%		\item Cids- System (Renderer/Editoren, DB-Modell, Rechte...)
%		\item GUI: Buchungen erstellen
%		\item ...
%	\end{itemize}
%	\item Auf dieser Basis wurde aufgebaut um die Realisierung durchzuführen.
%\end{itemize}


\section{Technische Realisierung}
\begin{itemize}
	\item Realisiert wurde
	\begin{itemize}
		\item in der DB
		\item in der GUI
		\item ...
	\end{itemize}
\end{itemize}


\section{Vergleich Realisierung und Spezifikation}
\begin{itemize}
	\item Durch diese Realisierung ist es möglich das zu tun was in den Anforderungen beschreiben wurde
	\item D.h. durch die Rationalisierung kann durchgeführt werden
\end{itemize}