\chapter{Spezifikation}
In diesem Kapitel werden die gewünschten Erweiterungen von \ac{WuNDa} spezifiziert.
Diese Erweiterungen werden der Übersicht halber zuerst aufgelistet und anschließend detaillierter beschrieben.
Im Folgenden werden die bis jetzt genannten Download-Protokolle auch Buchungen genannt. \todo{Immer einheitlicher Name verwenden? Bzw. begründung warum mal das eine bzw. das andere verwendet wird.}

Die Informationen zu diesem Kapitel sind aus einer Zusammenarbeit zwischen dem Ressort 102 und Cismet entstanden. \todo{Evt. beschreiben wie dies geschah.}
\section{Gewünschte Erweiterungen}
Dieser Abschnitt zeigt eine Übersicht über die gewünschten Erweiterungen. Dabei werden zunächst die wichtigsten, neu benötigten Objektarten aufgelistet, die in \ac{WuNDa} noch nicht  vorhanden sind. Danach erfolgt eine Auflistung der eigentlichen Erweiterungen.

Zu den neu benötigten Objektarten gehören
\begin{itemize}
\item die Kunden,
\item die Kunden-Logins
\item und die Kundengruppen.
\end{itemize}
Diese drei Objektarten sind am wichtigsten für das automatische Erstellen der Abrechnungen und werden im nächsten Abschnitt stärker beleuchtet.
Die restlichen neuen Objektarten werden in \todo{ref} aufgeführt und beschrieben. 

Damit im nächsten Abschnitt die Zusammenhänge der Erweiterungen aufgeführt werden können, ohne dass sämtliche Erweiterungen bereits erläutert wurden, werden diese hier aufgelistet.
Die Liste ist in Kategorien eingeteilt, wobei eine Kategorie die "`Beschaffenheit"' der Erweiterung darstellt. So finden sich unter den Kategorien die neuen Objektarten Kunde und Kundengruppe, aber auch die bereits bestehende Objektart Buchung. 
\todo{Erweitern, oder wegmachen da die Informationen dem TOC zu entnehmen sind.}
\begin{itemize}
	\item Allgemeines
	\begin{itemize}
		\item Hinzufügen des Mehrwertsteuer-Satzes bei Gebühren
	\end{itemize}
	\item Kunde
	\begin{itemize}
		\item Kundenrenderer und Editor
		\item Erneutes Herunterladen von Produkten
		\item Ansicht der eigenen Buchungen durch externe Benutzer
	\end{itemize}
	\item Kundengruppen
	\begin{itemize}
		\item Kundengruppenrenderer und Editor
		\item Erstellen von E-Mails an Kundengruppen
	\end{itemize}
	\item Buchungen
	\begin{itemize}
		\item Buchungrenderer und Editor
		\item Stornieren und Ändern einzelner Buchungen
		\item Suchdialog zu den Buchungen
	\end{itemize}
	\item Berichte
	\begin{itemize}
		\item Buchungsbeleg
		\item Rechnungsanlage
		\item Geschäftsstatistik
	\end{itemize}	 
	\item Renderer und Editoren für sonstige neue Objektarten
\end{itemize}

\section{Spezifizieren der Erweiterungen}
In diesem Abschnitt werden die oben genannten Erweiterungen spezifiziert. Dabei werden die Erweiterungen in der Regel zuerst erläutert und danach mittels möglichst wenigen und einfachen Sätzen spezifiziert. \todo{Nur verschieden Beamte des Amtes haben berechtigungen aus falls dies anders beschrieben wird.}

\subsection{Hinzufügen des Mehrwertsteuer-Satzes bei Gebühren}
Bei der bisherigen Implementierung wurde bei den Buchungen jeweils nur der Brutto-Preis abgespeichert. Dies ist nach der Umstellung nicht mehr ausreichend, da die Abrechnung mittels \ac{WuNDa} erstellt werden soll. Für diese Erstellung ist jedoch der Netto-Preis und der Mehrwertsteuer-Satz eines Produktes von Nöten.

Demnach soll für eine Buchung der Netto-Preis, der Mehrwertsteuer-Satz sowie der Brutto-Preis gespeichert werden. Diese neuen Daten sollen ebenfalls im Downloadprotokoll-Dialog angezeigt werden. \todo{erweitern}
\subsection{Kunde}
Die neue Objektart Kunde soll eingeführt werden und repräsentiert einen Kunden des Ressort 102.
Ein Kunde soll folgende Informationen enthalten: \todo{Was bedeuten die einzelnen Felder? Evt. descpition benutzen}
\begin{itemize}
\item Name
\item Vermessungsstellennummer
\item Vertragskennzeichen
\item Vertragsende
\item Abrechnungsturnus
\item Direktkontakt
\item Weiterverkaufsvertragskennzeichen
\item Weiterverkaufsvertragsende
\item Branche
\item Produkte
\end{itemize}
Eine andere neue Objektart soll das Kunden-Login sein, diese repräsentieren die externen Benutzer.
Dies bedeutet dass ein Kunden-Login einem \ac{WuNDa}-Benutzer entspricht.
Weiterhin soll ein Kunden-Login genau einem Kunden zugeordnet sein und beinhaltet folgende Informationen: \todo{Benutzer in WUnda}
\begin{itemize}
\item Kunde
\item Name
\item Kontakt
\end{itemize}
\subsubsection{Kundenrenderer}
Der Kundenrenderer soll Buchungen eines Kunden darstellen.
Dabei sollen verschiedene Filtermöglichkeiten angegeben werden können, so dass nur Buchungen angezeigt werden, die die Kriterien erfüllen.
Die Buchungen sollen in einer Tabelle angezeigt werden, weiterhin soll ein kurzer Satz die angezeigten Buchungen zusammenfassen.
Für die angezeigten Buchungen soll es möglich sein den Bericht Rechnungsanlage und Buchungsbeleg zu erstellen.

Die einzelnen Filtermöglichkeiten und der Aufbau der Tabelle kann dem Mock-up des Kundenrenderers \vref{fig:mockup-kunderenderer} entnommen werden.

\begin{figure}[htb]
	\centering
	\incgraph{width=\textwidth}{./rimg/gui-kunderenderer.png}
	\caption{Mockup des Kundenrendereres}
	\label{fig:mockup-kunderenderer}
\end{figure}

\subsubsection{Ansicht der eigenen Buchungen durch externe Benutzer}
Die Funktionalität des Kundenrenderers soll dem Kunden selbst nicht vorenthalten werden.
Aus diesem Grund soll es für die externen Benutzer eines Kunden möglich sein auf den eigenen Kundenrenderer zuzugreifen.
Hierbei muss sichergestellt werden, dass die externen Benutzer nur den Kundenrenderer des eigenen Kunden öffnen können und somit keine Einsicht in die Buchungen der anderen Kunden erhalten. 
Weiterhin dürfen sie den Bericht Rechnungsanlage nicht erstellen können, da es sich bei diesem um einen Teil der Abrechnung handelt.

Demnach sollen externe Benutzer nur den eigenen Kundenrenderer öffnen können und es soll ihnen nicht möglich sein den Bericht Rechnungsanlage zu erstellen.

\subsubsection{Erneutes Herunterladen von Produkten}
Die Produkte, die über den ALKIS-Druckdialog erzeugt werden, werden von einem externen Server bezogen.
Durch dessen längere Initialisierungsphase passiert es häufiger, dass Produkte fehlerhaft heruntergeladen werden.
Da der Download dieser Produkte trotzdem protokolliert wird, wie im Abschnitt \vref{subsec:beschaffung-download} bereits erwähnt wurde, muss der externe Benutzer diese Buchung stornieren lassen und den Download erneut durchführen.
Zwar gehört eine vereinfachte Stornierung auch zu den gewünschten Erweiterungen, trotzdem soll es für den Benutzer möglich sein ein Download nochmals anzufordern, ohne dass er eine Buchung stornieren lassen muss und dass auch keine neue Buchung erstellt werden muss.
Aus diesem Grund soll es für den Benutzer möglich sein, Produkte die bereits heruntergeladen wurden, erneut herunterzuladen. Dies soll allerdings nur für Produkte möglich sein, die am gleichen Tag bezogen wurden. 

Dieses erneute Herunterladen soll wie folgt umgesetzt werden. Bei einer Buchung, die am gleichen Tag erfolgte, soll in der Tabelle im Kundenrenderer das Datumsfeld eine zusätzliche Funktion als Auslöser des  erneuten Herunterladen des Produkts erhalten.
Das Datumsfeld soll in dem Fall blau und unterstrichen sein, wie dies der \myfig{fig:mockup-kunderenderer} zu entnehmen ist.
Bei einem Klick soll das bereits protokollierte Produkt erneut heruntergeladen werden, ohne dass eine neue Buchung erstellt wird.

\subsubsection{Kundeneditor}
Mit Hilfe des Kundeneditors soll es möglich sein einen neuen Kunden anzulegen, bzw. Informationen zu einem bereits existierenden Kunden zu ändern.

\subsection{Kundengruppen}
Bei den Kundengruppe handelt es sich um Mengen von Kunden und werden benötigt um Kunden mit ähnlichen Eigenschaften zu gruppieren.
Diese Kundengruppen besitzen kein Wissen über ihren Inhalt, sondern müssen manuell erstellt und verwaltet werden.

Ein Beispiel wäre eine Gruppe, die sämtliche Kunden zusammenfasst, die den Abrechnungsturnus "`Quartalsweise"' besitzen. Wird ein neuer Kunde erstellt, mit eben dem Abrechnungsturnus, so muss dieser manuell der Kundengruppe hinzugefügt werden.
Diese Kundegruppe hätte den Nutzen, dass beim Erstellen der Abrechnung für ein Quartal die benötigten Kunden nicht zusammengesucht werden müssen, sondern bereits in dieser Gruppe vorhanden sind.

Kurzum eine Kundengruppe soll eine Menge von Kunden sein, die manuell erstellt und verwaltet werden muss. Weiterhin soll eine Kundengruppe folgende Informationen enthalten:
\begin{itemize}
\item Name
\item Beschreibung
\item Liste der Kunden
\end{itemize}
 
\subsubsection{Kundengruppenrenderer}
Der Kundengruppenrenderer hat einen ähnlichen Aufbau und Funktionsweise wie der Kundenrenderer.
Demnach können wieder verschiedene Filter angegeben werden \todo{Welche Filter?} und die Berichte Rechnungsanlage, Buchungsbeleg und Geschäftsstatistik können erstellt werden.

Da im Vergleich zu dem Kundenrenderer nicht ein einzelner Kunde dargestellt wird, sondern mehrere, werden hier nicht sämtliche Buchungen angezeigt. Deshalb werden die Buchungen für die einzelnen Kunden zusammengefasst und somit in aggregierter Form in der Tabelle dargestellt.

Eine frühes Mock-up des Kundengruppenrenderer kann in \myfig{fig:mockup-kundengruppenrenderer} betrachtet werden.
\begin{figure}[htb]
	\centering
	\incgraph{width=\textwidth}{./rimg/gui-kundengruppenrenderer.png}
	\caption{Mockup des Kundengruppenrenderer}
	\label{fig:mockup-kundengruppenrenderer}
\end{figure}

\subsubsection{Kundengruppeneditor}
Mittels des Kundengruppeneditor sollen neue Kundengruppen angelegt werden können und bereits bestehende Kundengruppen sollen bearbeitet werden können. Zu diesen Bearbeitungen gehört z.B. das Hinzufügen bzw. Entfernen eines Kunden aus der Kundengruppe.

\subsubsection{Erstellen von E-Mails an Kundengruppen}
In verschiedenen Fällen müssen Benachrichtigung-E-Mails an alle oder verschiedene Kunden versendet werden.
Dies muss u.a. bei geplanten Systemausfällen, die durch Wartungsarbeiten verursacht werden, beim Einführen von neuen Funktionen oder sonstigen neuen Funktionalitäten wie z.B. neue Kartenthemen erfolgen.
Im Augenblick besitzt jeder Beamte, der solche E-Mails versenden soll, einen eigenen Verteiler, den er manuell aktuell halten muss.
Nach dem Hinzufügen eines neuen Kunden oder dem Ändern einer E-Mail-Adresse kann es somit zu Unvollständigkeiten in den einzelnen Verteiler kommen, da diese noch nicht aktualisiert wurden.
Mittels den Kundengruppen können diese, auf die Beamten verteilten, Verteilter durch eine zentrale Lösung ersetzt werden, die für jeden Beamten den gleichen Stand besitzt.

Der gewünschte Lösungsansatz ist die Möglichkeit für eine Kundengruppe auszuwählen ob eine E-Mail an die jeweiligen Direktkontakte oder an jeden Kunden-Login der Kunden innerhalb der Gruppe erstellt werden soll.
Nach dieser Auswahl soll eine E-Mail Vorlage gewählt werden können. Diese Vorlage soll die To-Adressaten, den Betreff und den Text der Vorlage enthalten.
Nach der Auswahl der Vorlage soll ein E-Mail-Erstellungsfenster des Standard-E-Mail-Programms des Benutzers geöffnet werden.
In diesem Erstellungsfenster sollen die Daten der Vorlage eingesetzt werden. Weiterhin sollen die ausgewählten E-Mail-Adressen (sprich die Direktkontakte bzw. die Adressen aller Kunden-Logins) als BCC-Adressen hinzugefügt werden.

Daraufhin kann der Benutzer die E-Mail wie gewohnt bearbeiten und z.B. den bereits eingesetzten Text beliebig verändern und schlussendlich die E-Mail an sämtliche eingetragenen Kontakte versenden.

%\begin{itemize}
%	\item E-Mail an Kundengruppe
%	\item immer aktueller Verteiler
%	\item sonst pflegen der einzelnen E-Mail Verteiler. Änderung während Urlaub, Schon ist wer vergessen
%	\item benötigt bei Wartungsarbeiten, neue Funktionen,\\ oder sonstiges Neues (Kartenthemen)
%\end{itemize}

\subsection{Buchungen}
Bei den Buchungen handelt es sich um die in \todo{ref} bereits erwähnten Download-Protokolle. Im Augenblick können die Buchungen nicht in \ac{WuNDa} dargestellt oder bearbeitet werden.
\todo{Änderungen nur über den Suchdialog zu öffnen. Sämtliche Werte von Buchungen}


\subsubsection{Suchdialog zu den Buchungen}
Der Suchdialog soll es ermöglichen Buchungen über verschiedene Suchkriterien zu finden und anschließend im Buchungsrenderer oder -editor zu öffnen.
Der Suchdialog ist die einzige Möglichkeit diesen Renderer und Editor zu öffnen. Die Suchkriterien, die angegeben werden können, können dem Mock-up \myfig{fig:mockup-buchungssuchdialog} entnommen werden. In diesem Mock-up fehlen allerdings die Angabe des Abrechnungsturnus und der Markierung von stornierten und abrechneten Buchungen, diese sollen im fertigen Dialog ebenfalls enthalten sein.  
\begin{figure}[htbp]
	\centering
	\incgraph{width=0.95\textwidth}{./rimg/gui-buchung_suchdialog.png}
	\caption{Mockup des Buchungssuchdialog}
	\label{fig:mockup-buchungssuchdialog}
\end{figure}

\subsubsection{Buchungseditor und Buchungsrenderer}
Beim Buchungseditor und dem Buchungsrenderer soll es sich um die gleiche Ansicht handeln, wobei der Renderer keine Benutzereingabe entgegen nimmt.
Der Buchungseditor soll es nicht erlauben neue Buchungen zu erstellen, da diese nur über den Downloadprotokoll-Dialog nach dem Beziehen eines Produktes angelegt werden.
Weiterhin soll es im Buchungseditor nur möglich sein die Geschäftsbuchnummer und die Projektbezeichnung zu ändern, siehe unten.
Außerdem soll eine Buchung storniert werden können, siehe ebenfalls unten.

\subsubsection{Ändern einzelner Buchungen} \label{subsubsec:aendern_buchung}
Wie bereits erwähnt dürfen nicht sämtliche Werte einer Buchung im Buchungseditor verändert werden sondern es sollen nur die Werte Geschäftsbuchnummer und Projektbezeichnung verändert werden können.
Diese beiden Felder werden händisch eingegeben und deshalb sind Fehleingaben vorstellbar.

Weiterhin soll die jeweils letzte Änderung protokolliert werden. In dieser Protokollierung  sollen folgende Werte erfasst werden:
\begin{itemize}
\item Veränderter Wert (Geschäftsbuchnummer oder Projektbezeichnung)
\item alter Wert
\item neuer Wert
\item Benutzer
\item Zeitstempel
\end{itemize}

\subsubsection{Stornieren einzelner Buchungen} \label{subsubsec:storno_buchung}
Wie oben bereits erwähnt soll es über den Buchungseditor möglich sein Buchungen zu stornieren.

Bei diesem Vorgang soll ein Beamte ein Storno-Grund auswählen können und die Buchung stornieren. Daraufhin soll die betroffene Buchung als storniert markiert werden und das Datum und der Benutzer, der die Stornierung vorgenommen hat, sollen abgespeichert werden.

\subsection{Berichte}
Wie bereits erwähnt sollen im Kundenrenderer und im Kundengruppenrenderer verschiedene Berichte bezüglich den dargestellten Kunden und Buchungen erstellt werden. In diesem Abschnitt werden diese drei Berichte beleuchtet. 
\subsubsection{Buchungsbeleg}
Dieser Bericht soll die im Renderer angezeigten Buchungen tabellarisch darstellen.
Dabei ist der Bericht in einen Kopfbereich und eine Tabelle mit den Buchungen eingeteilt.

Im Kopfbereich des Berichts sollen sich folgende Basisinformationen befinden:
\begin{itemize}
  \item Kundename 
  \item Vertragsnummer
  \item Abrechnungszeitraum
\end{itemize}  
Die Tabelle soll folgende Felder der Buchungen enthalten: \todo{MwSt?}
\begin{itemize}
 \item Gebühr
 \item Geschäftsbuchnummer 
 \item Projektbezeichnung
 \item Zeit
 \item Verwendungszweck
 \item Produktbezeichnung
\end{itemize}

In den Renderern soll es möglich sein auszuwählen, ob im Bericht kostenlose Downloads angezeigt werden sollen oder nicht.
Falls im Renderer keine Buchungen ausgewählt sind, so soll der Bericht nicht erstellt werden.

\subsubsection{Rechnungsanlage}
Die Rechnungsanlage ist eine Erweiterung des Buchungsbeleg.
Diese besitzt zwischen dem Kopfbereich und der Tabelle ein zusätzliches Statistikfeld.
Weiterhin ist die Rechnungsanlage ein Teil der Abrechnung, so dass die angezeigten Buchungen, beim Erstellen des Berichts, als abgerechnet markiert werden. Dies ist auch der Grund weshalb der Bericht im Kundenrenderer nicht von den externen Benutzern erstellt werden darf.
Eine Vorlage eines solchen Berichts kann der Abbildung \vref{fig:vorlage_rechnungsanlage} entnommen werden.

In den Renderern soll es möglich sein auszuwählen, ob im Bericht kostenlose Downloads angezeigt werden sollen oder nicht. Weiterhin soll festgelegt werden können, ob die Buchungen tatsächlich als abgerechnet markiert werden sollen, oder nicht.

\subsubsection{Geschäftsstatistik}

\subsection{sonstige neu benötigte Objektarten}



