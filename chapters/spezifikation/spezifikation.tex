\chapter{Spezifikation}
In diesem Kapitel werden die gewünschten Erweiterungen von \ac{WuNDa} spezifiziert. Im ersten Teil des Kapitels werden diese der Übersicht halber aufgelistet und im zweiten Teil werden die einzelnen Erweiterungen detaillierter beschrieben.

Die Informationen zu diesem Kapitel sind aus einer Zusammenarbeit zwischen dem Ressort 102 und Cismet entstanden. \todo{Evt. beschreiben wie dies geschah.}
\section{Gewünschte Erweiterungen}
Die gewünschten Erweiterungen sind:
\begin{itemize}
	\item Allgemeines
	\begin{itemize}
    \item Hinzufügen des Mehrwertsteuer-Satzes bei Gebühren
    \end{itemize}
    \item Kunde
    \begin{itemize}
    \item Kunderenderer und Editor
    \item Erneutes Herunterladen von Produkten
    \item Ansicht der eigenen Buchungen durch externe Benutzer
    \end{itemize}
	\item Kundengruppen
	\begin{itemize}
    \item Kundengruppenrenderer und Editor
    \item Erstellen von E-Mails an Kundengruppen
    \end{itemize}
    \item Buchungen
	\begin{itemize}
    \item Buchungrenderer und Editor
    \item Suchdialog zu den Buchungen
    \end{itemize}    
	\item Renderer und Editoren für sonstige neue Objekte
\end{itemize}




\section{Spezifizieren der Erweiterungen}
Einzeln bedeutet jede Erweiterung
\subsection{Auflisten der Buchungen der Benutzer}
\subsection{Kundengruppen}
\subsection{Erstellen von E-Mails an Kundengruppen}
\begin{itemize}
	\item E-Mail an Kundengruppe
	\item immer aktueller Verteiler
	\item sonst pflegen der einzelnen E-Mail Verteiler. Änderung während Urlaub, Schon ist wer vergessen
	\item benötigt bei Wartungsarbeiten, neue Funktionen,\\ oder sonstiges Neues (Kartenthemen)
\end{itemize}
\subsection{...}

