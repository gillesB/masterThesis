\chapter{Spezifikation} \label{ch:spezi}
Dieses Kapitel spezifiziert die gewünschten Erweiterungen von \ac{WuNDa}, welche zuerst aufgelistet und danach detaillierter beschrieben werden.

Die Informationen zu diesem Kapitel sind aus einer Zusammenarbeit zwischen dem Ressort 102 und Cismet entstanden.
Dabei entstanden die Anforderungen nicht auf Anhieb, sondern es wurde von den Grundideen ausgegangen, so dass ein erstes Datenmodell und erste Mock-Ups entstanden.
Basierend auf diesen Dokumenten wurden die Ideen, in mehreren Schritten, verfestigt und die konkreten Anforderungen an \ac{WuNDa} entstanden.
Dieses Kapitel fasst diese Anforderungen in ihrer schlussendlichen Form zusammen. 

\section{Gewünschte Erweiterungen} \label{sec:gewuenschte-erweiterungen}
Dieser Abschnitt soll als Übersicht der gewünschten Erweiterungen dienen.
Dabei werden zunächst einige Begriffe kurz erklärt, die wichtigsten Klassen erwähnt und anschließend werden die gewünschten Erweiterungen aufgelistet.

Bei der Spezifikation werden möglichst nur allgemein bekannte technische Begriffe verwendet, die wenigen spezifischeren Ausnahmen werden hier erklärt.
\begin{description}
\item[Klasse] Mit Klasse ist eine Cids-Klasse gemeint. Objekte von diesen Klassen werden in einer Datenbank gespeichert.
Aus diesen Datenbankeinträgen lassen sich Java-Objekte erstellen, die auch CidsBean genannt werden.
Diese Objekte können in einem Java-Programm benutzt und verändert werden.
Anschließend können sie wieder in die Datenbank zurückgeführt werden.
\item[Renderer] Eine solche Klasse kann einen Renderer besitzen. Bei einem Renderer handelt es sich um eine Beschreibungsseite in einem Java-Programm, die Informationen bezüglich eines Objektes dieser Klasse anzeigt.
\item[Editor] Eine Cids-Klasse kann außerdem einen Editor besitzen. Ein Editor ist von dem Anzeigemechanismus ähnlich dem Renderer, im Gegensatz hierzu können die Informationen zu einem Objekt jedoch verändert und in der Datenbank gespeichert werden. Mit einem Editor ist es in der Regel auch möglich neue Objekte einer Cids-Klasse anzulegen. 
\end{description}

Um die Spezifikation umzusetzen muss mit solchen Cids-Klassen gearbeitet werden, dabei sind die drei neu eingeführten Klassen Kunden, Kunden-Logins und Kundenguppe sowie die bereits bestehende Klasse Buchungen am relevantesten für das automatische Erstellen der Abrechnungen und werden im \autoref{sec:spezifikation} stärker beleuchtet. 
Die restlichen neuen Klassen werden gesondert im \autoref{subsec:neue_klassen} aufgeführt und beschrieben.

%Damit im nächsten Abschnitt die Zusammenhänge der Erweiterungen aufgeführt werden können, ohne dass sämtliche Erweiterungen bereits erläutert wurden, werden diese hier aufgelistet.
Die Liste ist in Kategorien eingeteilt, wobei eine Kategorie die logische Zuordnung der Erweiterung darstellt.
So befinden sich unter den Kategorien die neuen Klassen Kunde und Kundengruppe, aber auch die bereits bestehende Klasse Buchung. 
\begin{itemize}
	\item Allgemeines
	\begin{itemize}
		\item Berechtigungen
		\item Hinzufügen des Mehrwertsteuer-Satzes bei Gebühren
	\end{itemize}
	\item Kunde
	\begin{itemize}
		\item Kundenrenderer und Editor
		\item Erneutes Herunterladen von Produkten
		\item Ansicht der eigenen Buchungen durch externe Benutzer
	\end{itemize}
	\item Kundengruppen
	\begin{itemize}
		\item Kundengruppenrenderer und Editor
		\item Erstellen von E-Mails an Kundengruppen
	\end{itemize}
	\item Buchungen
	\begin{itemize}
		\item Buchungrenderer und Editor
		\item Stornieren und Ändern einzelner Buchungen
		\item Suchdialog zu den Buchungen
	\end{itemize}
	\item Berichte
	\begin{itemize}
		\item Buchungsbeleg
		\item Rechnungsanlage
		\item Geschäftsstatistik
	\end{itemize}	 
	\item Renderer und Editoren für sonstige neue Klassen
\end{itemize}

\section{Spezifikation der Erweiterungen} \label{sec:spezifikation}
In diesem Abschnitt werden die oben genannten Erweiterungen spezifiziert. Dabei werden die Erweiterungen in der Regel zuerst erläutert und danach mittels möglichst wenigen und einfachen Sätzen spezifiziert.

\subsection{Berechtigungen} \label{subsec:berechtigungen}
Durch die gewünschten Erweiterungen können vertrauliche Informationen betrachtet und verändert werden.
Aus diesem Grund dürfen nur einige ausgewählte Benutzer Zugriff auf diese Erweiterungen erhalten.
Bei den Benutzern handelt es sich um die \ac{WuNDa}"=Administratoren und einige Beamten, die für die Kundenverwaltung zuständig sind.
Die einzige Ausnahme hiervon bildet die Ansicht der Kunden auf ihren eigenen Kundenrenderer, mehr Informationen hierzu befinden sich im Unterabschnitt \autoref{subsubsec:ansicht_externe_benutzer}. 

Demnach sollen nur die \ac{WuNDa}"=Administratoren und einige Beamten der Kundenverwaltung Zugriff auf die gewünschte Erweiterungen erhalten. Wie bereits erwähnt, soll es hiervon nur eine einzige Ausnahme geben.

\subsection{Hinzufügen des Mehrwertsteuer-Satzes bei Gebühren}
Bei der bisherigen Implementierung wurde bei den Buchungen jeweils nur der Brutto-Preis abgespeichert. Dies ist nach der Umstellung nicht mehr ausreichend, da die Abrechnung mittels \ac{WuNDa} erstellt werden soll. Für diese Erstellung ist der Netto-Preis und der Mehrwertsteuer-Satz eines Produktes von Nöten. Weiterhin soll aus Gründen der Nachvollziehbarkeit ebenfalls der Brutto-Preis mit gespeichert werden.

Demnach soll für eine Buchung der Netto-Preis, der Mehrwertsteuer-Satz sowie der Brutto-Preis gespeichert werden. Diese neuen Daten sollen ebenfalls im Downloadprotokoll-Dialog angezeigt werden.
\subsection{Kunde}
Im aktuellen Stand werden die Kunden des Ressorts 102 nicht in \ac{WuNDa} verwaltet, jedoch wird dies durch die gewünschten Erweiterungen zwingend notwendig. Daher wird eine Klasse  benötigt die es ermöglicht Kunden zu verwalten und darzustellen. 

Für die Klasse Kunde sollen folgende Felder bereitstehen:
\begin{description}
\item[Name] Der Name des Kunden, dies kann z.\,B. eine Organisationsbezeichnung sein.
\item[Vermessungsstellennummer] Dieses Feld enthält die Zulassungsnummer einer Vermessungsstelle. Eine Vermessungsstelle ist eine Behörde oder Person, die berechtigt ist Liegenschaftsvermessungen durchzuführen und Abmarkungen vorzunehmen. Zu den Vermessungsstellen zählen z.\,B. das Katasteramt oder ÖbVI \autocite[vgl.][]{recklinghausen-vermessungsstelle}.
\item[Vertragskennzeichen] Das Kennzeichen des Vertrages der dem Kunden die Benutzung von \ac{WuNDa} erlaubt.
\item[Vertragsende] Das Auslaufdatum des Vertrages.
\item[Abrechnungsturnus] Der zeitliche Abstand wann ein Kunde seine Abrechnung erhält. (z.\,B.. quartalsweise, jährlich)
\item[Direktkontakt] Eine E-Mail-Adresse mit der, der Kunde sofort erreicht werden kann. Dies kann zum Beispiel die Adresse des Geschäftsleiters sein.
\item[Weiterverkaufsvertragskennzeichen] Das Kennzeichen des Vertrages der dem Kunden das Weiterverkaufen von bezogenen Produkten erlaubt.
\item[Weiterverkaufsvertragsende] Das Auslaufdatum des Weiterverkaufsvertrags.
\item[Branche] Die Branche in der, der Kunde tätig ist. Ein Kunde kann in genau einer Branche tätig sein. Hierbei handelt es sich um eine neue Klasse.
\item[Produkte] Die Softwareprodukte die ein Kunde benutzt. Ein Kunde kann mehrere Produkte besitzen. Hierbei handelt es sich ebenfalls um eine neue Klasse.
\end{description}

Neben den Kunden sollen die externen Benutzer in \ac{WuNDa} abgebildet werden, hierfür wird die Klasse Kunden-Login eingeführt.
Somit repräsentiert ein Kunden-Login einen \ac{WuNDa}-Benutzer.
Weiterhin soll ein Kunden-Login genau einem Kunden zugeordnet sein. 

Zusammenfassend benötigt ein Kunden-Login folgende Felder:
\begin{description}
\item[Name] Der Name des Kunden-Logins. Dies ist ebenfalls der Name mit dem sich der externe Benutzer in \ac{WuNDa} anmeldet.
\item[Kontakt] Die E-Mail-Adresse des externen Benutzers.
\item[Kunde] Die Zuordnung des Kunden-Logins zu genau einem Kunden.
\end{description}

Nach der Beschreibung der Klassen, folgen nun die eigentlichen Erweiterungen, die den Kunden betreffen.

\subsubsection{Kundenrenderer} \label{subsubsec:kundenrenderer}
Der Kundenrenderer soll Buchungen eines Kunden tabellarisch darstellen.
Dabei sollen verschiedene Filtermöglichkeiten angegeben werden können, so dass nur Buchungen angezeigt werden, die die Kriterien erfüllen.
Die Filterergebnisse sollen in einem kurzen Satz zusammengefasst werden.
Für die angezeigten Buchungen soll es möglich sein den Bericht \enquote{Rechnungsanlage} und \enquote{Buchungsbeleg} zu erstellen.

Die Buchungen sollen zeitlich gefiltert werden können.
Dabei soll es eine Schnellauswahl für die Buchungen des aktuellen Tages, eines beliebigen Monats und eines beliebigen Quartals geben.
Weiterhin soll ein Datumsbereich als Filter benutzt werden können.
Die Buchungen sollen auch nach Geschäftsbuchnummer, Projekt und nach dem Kunden-Login, der die Buchung angelegt hat, gefiltert werden können.
Weitere Filter sind der Kostentyp, sprich ob eine Buchung kostenfrei oder kostenpflichtig ist, sowie der Verwendungszweck der Buchung.
Zusätzlich besteht das Kriterium, dass sämtliche Filter eine neutrale Einstellung haben müssen, bei der sie keine filternde Funktionalität übernehmen.

In der Tabelle müssen die Geschäftsbuchnummer, die Projektbezeichnung, der Verwendungszweck, die Art des bezogenen Produktes, die Netto-Gebühr, der MwSt.-Satz, das Erstellungsdatum sowie der Kunden-Login einer Buchung angezeigt werden.

Die einzelnen Filtermöglichkeiten und der Aufbau der Tabelle kann dem Mock-up des Kundenrenderers \vref{fig:mockup-kunderenderer} entnommen werden.

\begin{figure}[htb]
	\centering
	\incgraph{width=\textwidth}{./rimg/gui-kunderenderer.png}
	\caption{Mock-up des Kundenrendereres}
	\label{fig:mockup-kunderenderer}
\end{figure}

\subsubsection{Ansicht der eigenen Buchungen durch externe Benutzer} \label{subsubsec:ansicht_externe_benutzer}
Die Funktionalität des Kundenrenderers soll dem Kunden selbst nicht vorenthalten werden.
Aus diesem Grund soll es für die externen Benutzer eines Kunden möglich sein auf den eigenen Kundenrenderer zuzugreifen.
Hierbei muss sichergestellt werden, dass die externen Benutzer nur den Kundenrenderer des eigenen Kunden öffnen können und somit keine Einsicht in die Buchungen der anderen Kunden erhalten. 
Zudem dürfen sie den Bericht \enquote{Rechnungsanlage} nicht erstellen können, da es sich bei diesem um einen Teil der Abrechnung handelt.

Demnach sollen externe Benutzer nur den eigenen Kundenrenderer öffnen können und es soll ihnen nicht möglich sein den Bericht \enquote{Rechnungsanlage} zu erstellen.

\subsubsection{Erneutes Herunterladen von Produkten}
Die Produkte, die über den ALKIS-Druckdialog erzeugt werden, werden von einem externen Server bezogen.
Durch dessen längere Initialisierungsphase passiert es häufiger, dass Produkte fehlerhaft heruntergeladen werden.
Da der Download dieser Produkte trotzdem protokolliert wird, wie im Abschnitt \vref{subsec:beschaffung-download} bereits erwähnt wurde, muss der externe Benutzer diese Buchung stornieren lassen und den Download erneut durchführen.
Zwar gehört eine vereinfachte Stornierung auch zu den gewünschten Erweiterungen, trotzdem soll es für den Benutzer möglich sein ein Download nochmals anzufordern, ohne dass er eine Buchung stornieren lassen muss.
Weiterhin soll durch diesen Vorgang keine neue Buchung entstehen.
Aus diesem Grund soll es für den Benutzer möglich sein, Produkte die bereits heruntergeladen wurden, erneut herunterzuladen. Einschränkend soll dies nur für Produkte möglich sein, die am selben Tag und über den ALKIS"=Druckdialog bezogen wurden. 

Dieses erneute Herunterladen soll wie folgt umgesetzt werden. Bei einer Buchung, die am selben Tag erfolgte, soll in der Tabelle im Kundenrenderer das Datumsfeld eine zusätzliche Funktion als Auslöser zum erneuten Herunterladen des Produkts erhalten.
Wenn das Datumsfeld ebenfalls als Auslöser dient, dann soll es blau unterstrichen sein, wie dies der \myfig{fig:mockup-kunderenderer} zu entnehmen ist.
Bei einem Klick auf ein solches Datumsfeld soll das bereits protokollierte Produkt erneut heruntergeladen werden, ohne dass eine neue Buchung erstellt wird.

\subsubsection{Kundeneditor}
Mit Hilfe des Kundeneditors soll es möglich sein einen neuen Kunden anzulegen, bzw. Informationen zu einem bereits existierenden Kunden zu ändern.

\subsection{Kundengruppen}
Kundengruppen werden benötigt um Kunden mit ähnlichen Eigenschaften zu gruppieren.
Die Kundengruppen besitzen kein Wissen über ihren Inhalt, sondern müssen manuell erstellt und verwaltet werden.

Die Funktionsweise einer Kundengruppe soll an dem Beispiel Gruppe "`Quartalsweise"' erläutert werden.
Diese Gruppe enthält sämtliche Kunden die den Abrechnungsturnus "`Quartal"' besitzen. Wird ein neuer Kunde erstellt, mit eben dem Abrechnungsturnus, so muss dieser manuell der Kundengruppe hinzugefügt werden.
Diese Kundegruppe hat den Nutzen, dass beim Erstellen der Abrechnung für ein Quartal die benötigten Kunden nicht zusammengesucht werden müssen, sondern bereits in dieser Gruppe gesammelt sind.

Kurzum eine Kundengruppe soll eine Gruppe von Kunden sein, die manuell erstellt und verwaltet werden muss. Eine Kundengruppe soll folgende Informationen enthalten:
\begin{itemize}
\item Name
\item Beschreibung
\item Liste der Kunden
\end{itemize}
 
\subsubsection{Kundengruppenrenderer}
Der Kundengruppenrenderer soll die Buchungen eines jeden Kundens einer Kundengruppe aggregieren und tabellarisch darstellen.
Seine Funktionsweise ist ähnlich dem des Kundenrenderers und besitzt deshalb auch einen ähnlichen Aufbau.
Demnach können wieder verschiedene Filtermöglichkeiten angegeben werden und die Berichte \enquote{Rechnungsanlage} und \enquote{Buchungsbeleg} können erstellt werden. Zusätzlich steht hier auch der Bericht \enquote{Geschäftsstatistik} zur Auswahl.

Da im Vergleich zu dem Kundenrenderer nicht ein einzelner Kunde dargestellt wird, sondern mehrere, werden hier nicht sämtliche Buchungen angezeigt. Deshalb werden die Buchungen für die einzelnen Kunden zusammengefasst und somit in aggregierter Form in der Tabelle dargestellt.
Die in der Tabelle angezeigten Informationen sind die Kundennamen, die aggregierten Brutto-Preise und die Anzahl der kostenpflichtigen Downloads. Weiterhin soll jeder Eintrag in der Tabelle ein Auswahlfeld besitzen, mit dem der Benutzer auswählen kann, ob die Buchungen eines Kunden beim Erstellen eines Berichtes benutzt werden sollen, oder nicht.

Die Buchungen können genau wie im Kundenrenderer zeitlich und nach Verwendungszweck gefiltert werden. Weiterhin kann nach dem Abrechnungsturnus der Kunden gefiltert werden und kostenfreie Buchungen können ausgeblendet werden.

Eine Mock-up des Kundengruppenrenderer kann in \myfig{fig:mockup-kundengruppenrenderer} betrachtet werden.
\begin{figure}[htb]
	\centering
	\incgraph{width=\textwidth}{./rimg/gui-kundengruppenrenderer.png}
	\caption{Mock-up des Kundengruppenrenderer}
	\label{fig:mockup-kundengruppenrenderer}
\end{figure}

\subsubsection{Kundengruppeneditor}
Mittels des Kundengruppeneditor sollen neue Kundengruppen angelegt werden können und bereits bestehende Kundengruppen sollen bearbeitet werden können. Zu diesen Bearbeitungen gehört z.\,B.das Hinzufügen bzw. Entfernen eines Kunden aus der Kundengruppe.

\subsubsection{Erstellen von E-Mails an Kundengruppen}
In verschiedenen Fällen müssen Benachrichtigung-E-Mails an alle oder verschiedene Kunden versendet werden.
Dies muss u.\,a. bei geplanten Systemausfällen, die durch Wartungsarbeiten verursacht werden, beim Einführen von neuen Funktionen oder sonstigen neuen Funktionalitäten wie z.\,B. neue Kartenthemen erfolgen.
Im aktuellen Stand besitzt jeder Beamte, der solche E-Mails versenden soll, einen eigenen Verteiler, den er manuell aktuell halten muss.
Nach dem Hinzufügen/Entfernen eines Kunden oder dem Ändern einer E-Mail-Adresse kann es somit zu Unvollständigkeiten in den einzelnen Verteiler kommen, da diese noch nicht aktualisiert wurden.
Mittels der Kundengruppen können diese, von den Beamten einzeln verwalteten, Verteiler durch eine zentrale Lösung ersetzt werden, die für jeden Beamten den gleichen Stand besitzt.

Der gewünschte Lösungsansatz ist die Möglichkeit für eine Kundengruppe auszuwählen, ob eine E-Mail an die jeweiligen Direktkontakte oder an jeden Kunden-Login der Kunden innerhalb der Gruppe erstellt werden soll.
Nach dieser Auswahl soll eine E-Mail Vorlage gewählt werden können, welche den Betreff und den Text der E-Mail enthält.
Weiterhin soll sie zusätzlich die To-Adressaten enthalten, dies sind in der Regel Adressen von Mitarbeiter aus dem Katasteramt.
Diese Mitarbeiter sollen die später versendete E-Mail als Kontrolle erhalten. 
Nach der Auswahl der Vorlage soll ein E-Mail-Erstellungsfenster des Standard-E-Mail-Programms des Benutzers geöffnet werden.
In diesem Erstellungsfenster sollen die Daten der Vorlage eingesetzt werden. Weiterhin sollen die ausgewählten E-Mail-Adressen (sprich die Direktkontakte bzw. die Adressen aller Kunden-Logins) als BCC-Adressen hinzugefügt werden.

Daraufhin kann der Benutzer die E-Mail wie gewohnt bearbeiten und z.\,B. den bereits eingesetzten Text beliebig verändern und schlussendlich die E-Mail an sämtlich eingetragene Kontakte versenden.

%\begin{itemize}
%	\item E-Mail an Kundengruppe
%	\item immer aktueller Verteiler
%	\item sonst pflegen der einzelnen E-Mail Verteiler. Änderung während Urlaub, Schon ist wer vergessen
%	\item benötigt bei Wartungsarbeiten, neue Funktionen,\\ oder sonstiges Neues (Kartenthemen)
%\end{itemize}

\subsection{Buchungen} \label{subsec:buchungen}
Bei den Buchungen handelt es sich um die bereits mehrmals erwähnten Download"=Protokolle. Im aktuellen Stand können die Buchungen nicht in \ac{WuNDa} dargestellt oder bearbeitet werden.

\subsubsection{Suchdialog zu den Buchungen} 
Der Suchdialog soll es ermöglichen Buchungen über verschiedene Suchkriterien zu finden und anschließend im Buchungsrenderer oder -editor zu öffnen.
Diesen Renderer und Editor soll nur über den Umweg des Suchdialogs geöffnet werden können. 
Die Suchkriterien, die angegeben werden können, sind: Kundenname, Abrechnungsturnus des Kunden, Geschäftsbuchnummer, Projekt der Buchung sowie Kunden-Login, mit dem die Buchung angelegt wurde.
Weiterhin soll, es wie im Kundenrender, möglich sein zeitliche Angaben, den Verwendungszweck und den Kostentyp anzugeben. Zusätzlich kann angegeben werden ob stornierte oder abgerechnete Buchungen gefunden werden sollen.

Der prinzipielle Aufbau des Suchdialogs kann dem Mock-up in \autoref{fig:mockup-buchungssuchdialog} entnommen werden.
\begin{figure}[htbp]
	\centering
	\incgraph{width=0.95\textwidth}{./rimg/gui-buchung_suchdialog.png}
	\caption{Mockup des Buchungssuchdialog}
	\label{fig:mockup-buchungssuchdialog}
\end{figure}

\subsubsection{Buchungseditor und Buchungsrenderer}
Beim Buchungseditor und dem Buchungsrenderer soll es sich um die gleiche Ansicht handeln, wobei der Renderer keine Benutzereingabe entgegen nimmt.
Der Buchungseditor soll es nicht erlauben neue Buchungen zu erstellen, da diese nur über den Downloadprotokoll-Dialog nach dem Beziehen eines Produktes angelegt werden sollen.
Es soll im Buchungseditor nur möglich sein die Geschäftsbuchnummer und die Projektbezeichnung zu ändern.
Außerdem soll eine Buchung storniert werden können, weitere Informationen hierzu befinden sich in den zwei folgenden Abschnitten.

\subsubsection{Änderung von einzelnen Buchungen} \label{subsubsec:aendern_buchung}
Im Buchungseditor dürfen, wie bereits erwähnt, nicht alle Werte einer Buchung verändert werden, sonders nur die Werte \enquote{Geschäftsbuchnummer} und \enquote{Projektbezeichnung}. Da die beiden letztgenannten Werte manuell vom externen Benutzer eingegeben werden, kann es zu Fehleingaben kommen, die die Beamten des Katasteramts verbessern können.

Für jede Buchung wird die jeweils letzte Änderung protokolliert und somit wird auf eine Historie von den Änderungen bewusst verzichtet. Wird eine Buchung also zwei oder mehrmals verändert, wird das Protokoll der vorherigen Änderung überschrieben.
In dieser Protokollierung sollen folgende Werte erfasst werden:
\begin{itemize}
\item veränderter Wert (Geschäftsbuchnummer oder Projektbezeichnung)
\item alter Wert
\item neuer Wert
\item Benutzer der die Änderung vornimmt
\item Zeitstempel
\end{itemize}
Diese Felder werden der Klasse Buchung hinzugefügt, so dass jede Buchung sein eigenes Änderungsprotokoll mit speichert.

\subsubsection{Stornierung von einzelnen Buchungen} \label{subsubsec:storno_buchung}
Über den Buchungseditor, wie vorher bereits erwähnt, soll es möglich sein Buchungen zu stornieren, bei diesem Vorgang soll jedoch ein Grund ausgewählt werden, weshalb es zu der Stornierung kommt. Um diese Auswahl darzustellen wird die Klasse Storno-Grund eingeführt, diese ist in Kapitel \ref{subsec:neue_klassen} beschrieben.

Nachdem ein Beamte ein Storno-Grund ausgewählt hat und somit den Vorgang der Stornierung angestoßen hat, soll die Buchung als storniert markiert werden und mit dem Datum der Stornierung, sowie dem Storno-Grund und dem Benutzer abgespeichert werden.

\subsection{Berichte} \label{subsec:berichte}
Wie bereits erwähnt sollen im Kundenrenderer und im Kundengruppenrenderer verschiedene Berichte bezüglich den dargestellten Kunden und Buchungen erstellt werden. In diesem Abschnitt werden diese drei Berichte beleuchtet.

\subsubsection{Buchungsbeleg}
Der \enquote{Buchungsbeleg} soll die im Renderer angezeigten Buchungen tabellarisch darstellen.
Dabei ist der Bericht in einen Kopfbereich und eine Tabelle mit den Buchungen eingeteilt.

Im Kopfbereich des Berichts sollen sich folgende Basisinformationen befinden:
\begin{itemize}
  \item Kundename 
  \item Vertragsnummer
  \item Abrechnungszeitraum
\end{itemize}  
Die Tabelle soll folgende Felder der Buchungen enthalten:
\begin{itemize}
 \item Gebühr
 \item Geschäftsbuchnummer 
 \item Projektbezeichnung
 \item Zeit
 \item Verwendungszweck
 \item Produktbezeichnung
 \item MwSt.-Satz
\end{itemize}

In den Renderern soll es möglich sein auszuwählen, ob im Bericht kostenlose Downloads angezeigt werden sollen oder nicht.
Falls im Renderer keine Buchungen ausgewählt sind, so soll der Bericht nicht erstellt werden.

\subsubsection{Rechnungsanlage}
Die \enquote{Rechnungsanlage} ist eine Erweiterung des \enquote{Buchungsbelegs}, welcher zwischen dem Kopfbereich und der Tabelle ein zusätzliches Statistikfeld einnimmt.
Zudem ist die \enquote{Rechnungsanlage} ein Teil der Abrechnung, sodass die angezeigten Buchungen beim Erstellen der Berichte als abgerechnet markiert werden können.
Was auch der Grund ist, weshalb der Bericht im Kundenrenderer nicht von den externen Benutzern erstellt werden darf.
Eine Vorlage eines solchen Berichts kann der Abbildung \vref{fig:vorlage_rechnungsanlage} entnommen werden.

In den Renderern soll es möglich sein auszuwählen, ob im Bericht kostenlose Downloads angezeigt werden und ob die Buchungen tatsächlich als abgerechnet markiert werden.

\subsubsection{Geschäftsstatistik}
Die \enquote{Geschäftsstatistik} ist ein Bericht der Statistiken der Downloads von den verschiedenen Kunden.
Durch die jeweilige Darstellung mehrerer Kunden, kann der Bericht nur im Kundengruppenrenderer erstellt werden.
Es gibt verschiedene Informationen die in der \enquote{Geschäftsstatistik} angezeigt werden: die Aktivität der Benutzer, die Kostenarten der Produkte, die Verwendungszwecke, die Angaben für den Jahresbericht und die Übersicht der bezogenen Produkte.
Der Bericht wurde allerdings nicht vollständig vom R102 spezifiziert und kann somit nicht komplett umgesetzt werden.
Aus diesem Grund soll der Bericht lediglich in den Masken vorgesehen werden und ein Beispielbericht soll generiert werden können. Dieser entspricht allerdings nicht den tatsächlichen Geschäftsstatistiken.
Eine etwaige Vorlage des Berichts kann den Abbildungen \vrefrange{fig:vorlage_geschaeftsstatistik_1}{fig:vorlage_geschaeftsstatistik_2} entnommen werden.
 
\subsection{sonstige neu benötigte Klassen} \label{subsec:neue_klassen}
Der Hauptgrund, warum diese, im Folgenden aufgelisteten, Klassen eingeführt werden, ist die Unterstützung des Benutzers bei der Eingabe von den jeweiligen Werten.

Dem Benutzer wird eine vorgefertigte Auswahl an Elementen angeboten, somit muss er die entsprechenden Felder nicht händisch in der GUI eintragen.
Dies beschleunigt einerseits das Eintragen und verhindert zudem Fehleingaben. 

Zum Einführen der Klasse Kunde und zum Erweitern der Buchungen werden folgende zusätzlichen Klassen benötigt:
\begin{description}
\item[Abrechnungsturnus] Der Abrechnungsturnus gibt an wie häufig ein Kunde eine Abrechnung erhält. Sein einziges Feld ist die Bezeichnung des Abrechnungsturnus (z.\,B. Jahr, Quartal)
\item[Client] Das einzige Feld eines Clients ist seine Bezeichnung.
\item[Produkt] Ein Produkt repräsentiert die verschiedenen Softwareprodukte, die ein Kunde benutzen kann. Seine Felder sind eine Bezeichung des Produktes, eine Beschreibung sowie der konkrete Client, mit dem sich der Benutzer anmelden kann.
\item[Stornogrund] Ein Stornogrund ist der Grund, warum eine Buchung storniert wurde. Sein einziges Feld ist die Bezeichnung des Grundes.
\end{description}

Bei diesen Klassen sollen die sogenannten Standardrenderer und -editoren verwendet werden. Diese werden für jede Klasse automatisch erstellt und müssen nicht eigens implementiert werden.
Mit dem Standardeditor ist es möglich neue Objekte einer Klasse anzulegen und sämtliche Felder der Klasse zu verändern.
Die Anforderungen genügen allerdings für die hier aufgelisteten Klassen.

