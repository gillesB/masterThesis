\chapter{Spezifikation} \label{ch:spezi}
In diesem Kapitel werden die gewünschten Erweiterungen von \ac{WuNDa} spezifiziert.
Diese Erweiterungen werden der Übersicht halber zuerst aufgelistet und danach detaillierter beschrieben.

Die Informationen zu diesem Kapitel sind aus einer Zusammenarbeit zwischen dem Ressort 102 und Cismet entstanden.
Dabei entstanden die Anforderungen nicht auf einen Schlag, sondern es wurde von den Grundideen ausgegangen, so dass ein erstes Datenmodell und erste Mock-Ups entstanden. Basierend auf diesen Dokumenten wurden die Ideen verfestigt und die konkreten Anforderungen an \ac{WuNDa} entstanden.
Dieses Kapitel fasst diese Anforderungen in ihrer schlussendlichen Form zusammen. 
 \todo{Evt. beschreiben wie dies geschah; rund ist das noch nicht}

\section{Gewünschte Erweiterungen} \label{sec:gewuenschte-erweiterungen}
Dieser Abschnitt soll als Übersicht der gewünschten Erweiterungen dienen.
Dabei werden zunächst einige Begriffe kurz erklärt, die wichtigsten Klassen erwähnt und anschließend werden die gewünschten Erweiterungen aufgelistet.

Bei der Spezifikation werden möglichst nur allgemein bekannte technische Begriffe verwendet, die wenigen spezifischeren Ausnahmen werden hier erklärt. \todo{ref zum technischen Umfeld?}
\begin{description}
\item[Klasse] Mit Klasse ist eine Cids-Klasse gemeint. Objekte von diesen Klassen werden in einer Datenbank gespeichert. Aus diesen Datenbankeinträgen lassen sich Java-Objekte erstellen, die in einem Java-Programm benutzt und verändert werden können und anschließend wieder in der Datenbank gespeichert werden können.
\item[Renderer] Eine solche Klasse kann einen Renderer besitzen. Bei einem Renderer handelt es sich um eine Beschreibungsseite in einem Java-Programm, die Informationen bezüglich eines Objektes dieser Klasse anzeigt.
\item[Editor] Eine Cids-Klasse kann außerdem einen Editor besitzen. Ein Editor ist von dem Anzeigemechanismus ähnlich dem Renderer, im Gegensatz hierzu können die Informationen zu einem Objekt jedoch verändert und in der Datenbank gespeichert werden. Mit einem Editor ist es in der Regel auch möglich neue Objekte einer Cids-Klasse anzulegen. 
\end{description}

Um die Spezifikation umzusetzen muss mit solchen Cids-Klassen gearbeitet werden, dabei sind die drei neu eingeführten Klassen Kunden, Kunden-Logins und Kundenguppe sowie die bereits bestehende Klasse Buchungen am relevantesten für das automatische Erstellen der Abrechnungen und werden im \autoref{sec:spezifikation} stärker beleuchtet. 
Die restlichen neuen Klassen werden gesondert im \autoref{subsec:neue_klassen} aufgeführt und beschrieben.

Damit im nächsten Abschnitt die Zusammenhänge der Erweiterungen aufgeführt werden können, ohne dass sämtliche Erweiterungen bereits erläutert wurden, werden diese hier aufgelistet.
Die Liste ist in Kategorien eingeteilt, wobei eine Kategorie die logische Zuordnung der Erweiterung darstellt. So finden sich unter den Kategorien die neuen Klassen Kunde und Kundengruppe, aber auch die bereits bestehende Klasse Buchung. 
\begin{itemize}
	\item Allgemeines
	\begin{itemize}
		\item Berechtigungen
		\item Hinzufügen des Mehrwertsteuer-Satzes bei Gebühren
	\end{itemize}
	\item Kunde
	\begin{itemize}
		\item Kundenrenderer und Editor
		\item Erneutes Herunterladen von Produkten
		\item Ansicht der eigenen Buchungen durch externe Benutzer
	\end{itemize}
	\item Kundengruppen
	\begin{itemize}
		\item Kundengruppenrenderer und Editor
		\item Erstellen von E-Mails an Kundengruppen
	\end{itemize}
	\item Buchungen
	\begin{itemize}
		\item Buchungrenderer und Editor
		\item Stornieren und Ändern einzelner Buchungen
		\item Suchdialog zu den Buchungen
	\end{itemize}
	\item Berichte
	\begin{itemize}
		\item Buchungsbeleg
		\item Rechnungsanlage
		\item Geschäftsstatistik
	\end{itemize}	 
	\item Renderer und Editoren für sonstige neue Klassen
\end{itemize}

\section{Spezifikation der Erweiterungen} \label{sec:spezifikation}
In diesem Abschnitt werden die oben genannten Erweiterungen spezifiziert. Dabei werden die Erweiterungen in der Regel zuerst erläutert und danach mittels möglichst wenigen und einfachen Sätzen spezifiziert.

\subsection{Berechtigungen}
Durch die Erweiterungen können vertrauliche Informationen betrachtet und verändert werden.
Aus diesem Grund dürfen nur einige ausgewählte Benutzer Zugriff auf diese Erweiterungen erhalten.
Bei den Benutzern handelt es sich um die \ac{WuNDa}"=Administratoren und einigen Beamten die für die Kundenverwaltung zuständig sind.
Die einzige Ausnahme hiervon bildet die Ansicht der Kunden auf ihren eigenen Kundenrenderer.
Siehe hierzu auch den Unterabschnitt "`\nameref{subsubsec:ansicht_externe_benutzer}"' im \autoref{subsubsec:ansicht_externe_benutzer}. 

Demnach sollen nur die \ac{WuNDa}"=Administratoren und einige Beamten der Kundenverwaltung Zugriff auf die gewünschte Erweiterungen erhalten. Wie bereits erwähnt, soll es hiervon nur eine einzige Ausnahme geben.

\subsection{Hinzufügen des Mehrwertsteuer-Satzes bei Gebühren}
Bei der bisherigen Implementierung wurde bei den Buchungen jeweils nur der Brutto-Preis abgespeichert. Dies ist nach der Umstellung nicht mehr ausreichend, da die Abrechnung mittels \ac{WuNDa} erstellt werden soll. Für diese Erstellung ist der Netto-Preis und der Mehrwertsteuer-Satz eines Produktes von Nöten. Weiterhin soll aus Gründen der Nachvollziehbarkeit ebenfalls der Brutto-Preis mit gespeichert werden.

Demnach soll für eine Buchung der Netto-Preis, der Mehrwertsteuer-Satz sowie der Brutto-Preis gespeichert werden. Diese neuen Daten sollen ebenfalls im Downloadprotokoll-Dialog angezeigt werden.
\subsection{Kunde}
Im aktuellen Stand werden die Kunden des Ressort 102 nicht in \ac{WuNDa} verwaltet, durch die gewünschten Erweiterungen wird dies jedoch zwingend notwendig. Daher wird eine Klasse  benötigt die es ermöglicht Kunden zu verwalten und darzustellen. 

Für die Klasse Kunde sollen folgende Felder bereitstehen:
\begin{description}
\item[Name] Der Name des Kunden, dies kann z.B. eine Organisationsbezeichnung sein.
\item[Vermessungsstellennummer] Dieses Feld enthält die Zulassungsnummer einer Vermessungsstelle. Eine Vermessungsstellen ist eine Behörde oder Person, die berechtigt ist Liegenschaftsvermessungen durchzuführen und Abmarkungen vorzunehmen. Zu den Vermessungsstellen zählen z.B. das Katasteramt oder ÖbVI \autocite[vgl.][]{recklinghausen-vermessungsstelle}.
\item[Vertragskennzeichen] Das Kennzeichen des Vertrages der dem Kunde die Benutzung von \ac{WuNDa} erlaubt.
\item[Vertragsende] Das Auslaufdatum dieses Vertrages.
\item[Abrechnungsturnus] Der zeitliche Abstand wann ein Kunde seine Abrechnung erhält. (z.B. quartalsweise, jährlich)
\item[Direktkontakt] Eine E-Mail-Adresse mit der, der Kunde sofort erreicht werden kann. Dies kann zum Beispiel die Adresse des Geschäftsleiters sein.
\item[Weiterverkaufsvertragskennzeichen] Das Kennzeichen des Vertrages der dem Kunden das Weiterverkaufen von bezogenen Produkten erlaubt.
\item[Weiterverkaufsvertragsende] Das Auslaufdatum des Weiterverkaufsvertrags.
\item[Branche] Die Branche in der der Kunde tätig ist. Ein Kunde kann in genau einer Branche tätig sein. Hierbei handelt es sich um eine neue Klasse.
\item[Produkte] Die Softwareprodukte die ein Kunde benutzt. Ein Kunde kann mehrere Produkte besitzen. Hierbei handelt es sich ebenfalls um eine neue Klasse.
\end{description}

Neben den Kunden sollen die externen Benutzer in \ac{WuNDa} abgebildet werden, hierfür wird die Klasse Kunden-Login eingeführt.
Somit repräsentiert ein Kunden-Login einen \ac{WuNDa}-Benutzer.
Weiterhin soll ein Kunden-Login genau einem Kunden zugeordnet sein. 

Zusammenfassend benötigt ein Kunden-Login folgende Felder:
\begin{description}
\item[Name] Der Name des Kunden-Logins. Dies ist ebenfalls der Name mit dem sich der externe Benutzer in \ac{WuNDa} anmeldet.
\item[Kontakt] Die E-Mail-Adresse des externen Benutzers.
\item[Kunde] Die Zuordnung des Kunden-Logins zu genau einem Kunden.
\end{description}
\subsubsection{Kundenrenderer}
Der Kundenrenderer soll Buchungen eines Kunden tabellarisch darstellen.
Dabei sollen verschiedene Filtermöglichkeiten angegeben werden können, so dass nur Buchungen angezeigt werden, die die Kriterien erfüllen.
Die Filterergebnisse sollen in einem kurzen Satz zusammengefasst werden.
Für die angezeigten Buchungen soll es möglich sein den Bericht Rechnungsanlage und Buchungsbeleg zu erstellen.

Die Buchungen sollen zeitlich gefiltert werden können.
Dabei soll es eine Schnellauswahl für die Buchungen des aktuellen Tages, eines beliebigen Monats und eines beliebigen Quartals geben.
Weiterhin soll ein Datumsbereich als Filter benutzt werden können.
Die Buchungen sollen auch nach ihrer Geschäftsbuchnummer, ihrem Projekt und dem Kunden-Login, der die Buchung angelegt hat, gefiltert werden können.
Weitere Filter sind der Kostentyp, sprich ob eine Buchung kostenfrei oder kostenpflichtig ist, sowie der Verwendungszweck der Buchung. Zusätzliches besteht das Kriterium dass sämtliche Filter eine neutrale Einstellung haben müssen, bei der sie keine filternde Funktionalität übernehmen.

In der Tabelle müssen die Geschäftsbuchnummer, die Projektbezeichnung, der Verwendungszweck, das Art des bezogenen Produktes, die Netto-Gebühr, der MwSt.-Satz, das Erstellungsdatum sowie der Kunden-Login der Buchung angezeigt werden.

Die einzelnen Filtermöglichkeiten und der Aufbau der Tabelle kann dem Mock-up des Kundenrenderers \vref{fig:mockup-kunderenderer} entnommen werden.

\begin{figure}[htb]
	\centering
	\incgraph{width=\textwidth}{./rimg/gui-kunderenderer.png}
	\caption{Mockup des Kundenrendereres}
	\label{fig:mockup-kunderenderer}
\end{figure}

\subsubsection{Ansicht der eigenen Buchungen durch externe Benutzer} \label{subsubsec:ansicht_externe_benutzer}
Die Funktionalität des Kundenrenderers soll dem Kunden selbst nicht vorenthalten werden.
Aus diesem Grund soll es für die externen Benutzer eines Kunden möglich sein auf den eigenen Kundenrenderer zuzugreifen.
Hierbei muss sichergestellt werden, dass die externen Benutzer nur den Kundenrenderer des eigenen Kunden öffnen können und somit keine Einsicht in die Buchungen der anderen Kunden erhalten. 
Weiterhin dürfen sie den Bericht Rechnungsanlage nicht erstellen können, da es sich bei diesem um einen Teil der Abrechnung handelt.

Demnach sollen externe Benutzer nur den eigenen Kundenrenderer öffnen können und es soll ihnen nicht möglich sein den Bericht Rechnungsanlage zu erstellen.

\subsubsection{Erneutes Herunterladen von Produkten}
Die Produkte, die über den ALKIS-Druckdialog erzeugt werden, werden von einem externen Server bezogen.
Durch dessen längere Initialisierungsphase passiert es häufiger, dass Produkte fehlerhaft heruntergeladen werden.
Da der Download dieser Produkte trotzdem protokolliert wird, wie im Abschnitt \vref{subsec:beschaffung-download} bereits erwähnt wurde, muss der externe Benutzer diese Buchung stornieren lassen und den Download erneut durchführen.
Zwar gehört eine vereinfachte Stornierung auch zu den gewünschten Erweiterungen, trotzdem soll es für den Benutzer möglich sein ein Download nochmals anzufordern, ohne dass er eine Buchung stornieren lassen muss.
Weiterhin soll durch diesen Vorgang keine neue Buchung entstehen.
Aus diesem Grund soll es für den Benutzer möglich sein, Produkte die bereits heruntergeladen wurden, erneut herunterzuladen. Einschränkend soll dies nur für Produkte möglich sein, die am gleichen Tag und über den ALKIS"=Druckdialog bezogen wurden. 

Dieses erneute Herunterladen soll wie folgt umgesetzt werden. Bei einer Buchung, die am gleichen Tag erfolgte, soll in der Tabelle im Kundenrenderer das Datumsfeld eine zusätzliche Funktion als Auslöser zum erneuten Herunterladen des Produkts erhalten.
Wenn das Datumsfeld ebenfalls als Auslöser dient, dann soll es blau unterstrichen sein, wie dies der \myfig{fig:mockup-kunderenderer} zu entnehmen ist.
Bei einem Klick auf ein solches Datumsfeld soll das bereits protokollierte Produkt erneut heruntergeladen werden, ohne dass eine neue Buchung erstellt wird.

\subsubsection{Kundeneditor}
Mit Hilfe des Kundeneditors soll es möglich sein einen neuen Kunden anzulegen, bzw. Informationen zu einem bereits existierenden Kunden zu ändern.

\subsection{Kundengruppen}
Bei den Kundengruppen handelt es sich um Mengen von Kunden. Diese Klasse wird benötigt um Kunden mit ähnlichen Eigenschaften zu gruppieren.
Die Kundengruppen besitzen kein Wissen über ihren Inhalt, sondern müssen manuell erstellt und verwaltet werden.

Die Funktionsweise einer Kundengruppe soll an dem Beispiel Gruppe "`Quartalsweise"' erläutert werden.
Diese Gruppe enthält sämtliche Kunden die den Abrechnungsturnus "`Quartal"' besitzen. Wird ein neuer Kunde erstellt, mit eben dem Abrechnungsturnus, so muss dieser manuell der Kundengruppe hinzugefügt werden.
Diese Kundegruppe hat den Nutzen, dass beim Erstellen der Abrechnung für ein Quartal die benötigten Kunden nicht zusammengesucht werden müssen, sondern bereits in dieser Gruppe gesammelt sind.

Kurzum eine Kundengruppe soll eine Menge von Kunden sein, die manuell erstellt und verwaltet werden muss. Eine Kundengruppe soll folgende Informationen enthalten:
\begin{itemize}
\item Name
\item Beschreibung
\item Liste der Kunden
\end{itemize}
 
\subsubsection{Kundengruppenrenderer}
Der Kundengruppenrenderer soll die Buchungen eines jeden Kundens einer Kundengruppe aggregieren und tabellarisch darstellen.
Seine Funktionsweise ist ähnlich dem des Kundenrenderers und besitzt deshalb auch einen ähnlichen Aufbau.
Demnach können wieder verschiedene Filtermöglichkeiten angegeben werden und die Berichte Rechnungsanlage, Buchungsbeleg können erstellt werden. Zusätzlich steht hier auch der Bericht Geschäftsstatistik zur Auswahl.

Da im Vergleich zu dem Kundenrenderer nicht ein einzelner Kunde dargestellt wird, sondern mehrere, werden hier nicht sämtliche Buchungen angezeigt. Deshalb werden die Buchungen für die einzelnen Kunden zusammengefasst und somit in aggregierter Form in der Tabelle dargestellt.
Die in der Tabelle angezeigten Informationen sind der Kundenname, der aggregierte Brutto-Preis und die Anzahl der kostenpflichtigen Downloads. Weiterhin soll jeder Eintrag in der Tabelle ein Auswahlfeld besitzen. Mit diesem Feld kann der Benutzer auswählen, ob die Buchungen eines Kunden beim Erstellen eines Berichtes benutzt werden sollen, oder nicht.

Die Buchungen können genau wie im Kundenrenderer zeitlich und nach Verwendungszweck gefiltert werden. Weiterhin kann nach dem Abrechnungsturnus des Kunden gefiltert werden und kostenfreie Buchungen können ausgeblendet werden.

Eine frühes Mock-up des Kundengruppenrenderer kann in \myfig{fig:mockup-kundengruppenrenderer} betrachtet werden.
\begin{figure}[htb]
	\centering
	\incgraph{width=\textwidth}{./rimg/gui-kundengruppenrenderer.png}
	\caption{Mockup des Kundengruppenrenderer}
	\label{fig:mockup-kundengruppenrenderer}
\end{figure}

\subsubsection{Kundengruppeneditor}
Mittels des Kundengruppeneditor sollen neue Kundengruppen angelegt werden können und bereits bestehende Kundengruppen sollen bearbeitet werden können. Zu diesen Bearbeitungen gehört z.B. das Hinzufügen bzw. Entfernen eines Kunden aus der Kundengruppe.

\subsubsection{Erstellen von E-Mails an Kundengruppen}
In verschiedenen Fällen müssen Benachrichtigung-E-Mails an alle oder verschiedene Kunden versendet werden.
Dies muss u.a. bei geplanten Systemausfällen, die durch Wartungsarbeiten verursacht werden, beim Einführen von neuen Funktionen oder sonstigen neuen Funktionalitäten wie z.B. neue Kartenthemen erfolgen.
Im aktuellen Stand besitzt jeder Beamte, der solche E-Mails versenden soll, einen eigenen Verteiler, den er manuell aktuell halten muss.
Nach dem Hinzufügen/Entfernen eines Kunden oder dem Ändern einer E-Mail-Adresse kann es somit zu Unvollständigkeiten in den einzelnen Verteiler kommen, da diese noch nicht aktualisiert wurden.
Mittels der Kundengruppen können diese, von den Beamten einzeln verwalteten, Verteilter durch eine zentrale Lösung ersetzt werden, die für jeden Beamten den gleichen Stand besitzt.

Der gewünschte Lösungsansatz ist die Möglichkeit für eine Kundengruppe auszuwählen ob eine E-Mail an die jeweiligen Direktkontakte oder an jeden Kunden-Login der Kunden innerhalb der Gruppe erstellt werden soll.
Nach dieser Auswahl soll eine E-Mail Vorlage gewählt werden können. Diese Vorlage soll den Betreff und den Text der E-Mail enthalten.
Weiterhin soll sie zusätzlich die To-Adressaten enthalten, dies sind in der Regel Adressen von Mitarbeiter aus dem Katasteramt.
Diese Mitarbeiter sollen die später versendete E-Mail als Kontrolle erhalten. 
Nach der Auswahl der Vorlage soll ein E-Mail-Erstellungsfenster des Standard-E-Mail-Programms des Benutzers geöffnet werden.
In diesem Erstellungsfenster sollen die Daten der Vorlage eingesetzt werden. Weiterhin sollen die ausgewählten E-Mail-Adressen (sprich die Direktkontakte bzw. die Adressen aller Kunden-Logins) als BCC-Adressen hinzugefügt werden.

Daraufhin kann der Benutzer die E-Mail wie gewohnt bearbeiten und z.B. den bereits eingesetzten Text beliebig verändern und schlussendlich die E-Mail an sämtlich eingetragene Kontakte versenden.

%\begin{itemize}
%	\item E-Mail an Kundengruppe
%	\item immer aktueller Verteiler
%	\item sonst pflegen der einzelnen E-Mail Verteiler. Änderung während Urlaub, Schon ist wer vergessen
%	\item benötigt bei Wartungsarbeiten, neue Funktionen,\\ oder sonstiges Neues (Kartenthemen)
%\end{itemize}

\subsection{Buchungen}
Bei den Buchungen handelt es sich um die bereits mehrmals erwähnten Download"=Protokolle. Im aktuellen Stand können die Buchungen nicht in \ac{WuNDa} dargestellt oder bearbeitet werden.

\subsubsection{Suchdialog zu den Buchungen}
Dieser Suchdialog soll es ermöglichen Buchungen über verschiedene Suchkriterien zu finden und anschließend im Buchungsrenderer oder -editor zu öffnen.
Diesen Renderer und Editor soll nur über den Umweg über den Suchdialog geöffnet werden können. 
Die Suchkriterien, die angegeben werden können, sind der Kundenname, der Abrechnungsturnus des Kunden, die Geschäftsbuchnummer und das Projekt der Buchung sowie das Kunden-Login, mit dem die Buchung angelegt wurde.
Weiterhin soll, es wie im Kundenrender, möglich sein zeitliche Angaben, den Verwendungszweck und den Kostentyp anzugeben. Zusätzlich kann angegeben werden ob stornierte oder abgerechnete Buchungen gefunden werden sollen.

Der prinzipielle Aufbau des Suchdialogs kann dem Mock-up \ref{fig:mockup-buchungssuchdialog} entnommen werden.
\begin{figure}[htbp]
	\centering
	\incgraph{width=0.95\textwidth}{./rimg/gui-buchung_suchdialog.png}
	\caption{Mockup des Buchungssuchdialog}
	\label{fig:mockup-buchungssuchdialog}
\end{figure}

\subsubsection{Buchungseditor und Buchungsrenderer}
Beim Buchungseditor und dem Buchungsrenderer soll es sich um die gleiche Ansicht handeln, wobei der Renderer keine Benutzereingabe entgegen nimmt.
Der Buchungseditor soll es nicht erlauben neue Buchungen zu erstellen, da diese nur über den Downloadprotokoll-Dialog nach dem Beziehen eines Produktes angelegt werden.
Weiterhin soll es im Buchungseditor nur möglich sein die Geschäftsbuchnummer und die Projektbezeichnung zu ändern.
Außerdem soll eine Buchung storniert werden können.
Siehe hierzu auch die zwei folgenden Abschnitte.

\subsubsection{Änderung von einzelnen Buchungen} \label{subsubsec:aendern_buchung}
Wie bereits erwähnt dürfen nicht sämtliche Werte einer Buchung im Buchungseditor verändert werden sondern es sollen nur die Werte Geschäftsbuchnummer und Projektbezeichnung verändert werden können.
Diese beiden Felder werden vom externen Benutzer händisch eingegeben und deshalb sind Fehleingaben vorstellbar, die von Beamten des Katasteramtes verbessert werden müssen.

Weiterhin soll die jeweils letzte Änderung protokolliert werden. Auf eine Historie mit den gemachten Änderungen wird bewusst verzichtet, erfolgen also zwei Änderungen so wird das Protokoll der erste Änderung überschrieben. In dieser Protokollierung sollen folgende Werte erfasst werden:
\begin{itemize}
\item Veränderter Wert (Geschäftsbuchnummer oder Projektbezeichnung)
\item alter Wert
\item neuer Wert
\item Benutzer der die Änderung vornimmt
\item Zeitstempel
\end{itemize}
Diese Felder werden der Klasse Buchung hinzugefügt, so dass jede Buchung sein eigenes Änderungsprotokoll mit speichert.

\subsubsection{Stornierung von einzelnen Buchungen} \label{subsubsec:storno_buchung}
Wie oben bereits erwähnt soll es über den Buchungseditor möglich sein Buchungen zu stornieren. Beim Stornieren einer Buchung soll ein Grund ausgewählt werden können. Dieser gibt an, warum eine Buchung storniert werden musste. Um diese Auswahl darzustellen wird die Klasse Storno-Grund eingeführt, diese ist in \ref{subsec:neue_klassen} beschrieben.

Bei dem Vorgang der Stornierung soll ein Beamte ein Storno-Grund auswählen können und die Stornierung selbst anstoßen. Daraufhin soll die betroffene Buchung als storniert markiert werden und das aktuelle Datum, der Benutzer, der die Stornierung vornimmt, sowie der Storno-Grund sollen mit der Buchung abgespeichert werden.

\subsection{Berichte}
Wie bereits erwähnt sollen im Kundenrenderer und im Kundengruppenrenderer verschiedene Berichte bezüglich den dargestellten Kunden und Buchungen erstellt werden. In diesem Abschnitt werden diese drei Berichte beleuchtet.  \todo{Müssen noch besser spezifiziert werden.}

\subsubsection{Buchungsbeleg}
Dieser Bericht soll die im Renderer angezeigten Buchungen tabellarisch darstellen.
Dabei ist der Bericht in einen Kopfbereich und eine Tabelle mit den Buchungen eingeteilt.

Im Kopfbereich des Berichts sollen sich folgende Basisinformationen befinden:
\begin{itemize}
  \item Kundename 
  \item Vertragsnummer
  \item Abrechnungszeitraum
\end{itemize}  
Die Tabelle soll folgende Felder der Buchungen enthalten: \todo{MwSt?}
\begin{itemize}
 \item Gebühr
 \item Geschäftsbuchnummer 
 \item Projektbezeichnung
 \item Zeit
 \item Verwendungszweck
 \item Produktbezeichnung
\end{itemize}

In den Renderern soll es möglich sein auszuwählen, ob im Bericht kostenlose Downloads angezeigt werden sollen oder nicht.
Falls im Renderer keine Buchungen ausgewählt sind, so soll der Bericht nicht erstellt werden.

\subsubsection{Rechnungsanlage}
Die Rechnungsanlage ist eine Erweiterung des Buchungsbeleg.
Diese besitzt zwischen dem Kopfbereich und der Tabelle ein zusätzliches Statistikfeld.
Weiterhin ist die Rechnungsanlage ein Teil der Abrechnung, so dass die angezeigten Buchungen, beim Erstellen des Berichts, als abgerechnet markiert werden. Dies ist auch der Grund weshalb der Bericht im Kundenrenderer nicht von den externen Benutzern erstellt werden darf.
Eine Vorlage eines solchen Berichts kann der Abbildung \vref{fig:vorlage_rechnungsanlage} entnommen werden.

In den Renderern soll es möglich sein auszuwählen, ob im Bericht kostenlose Downloads angezeigt werden sollen oder nicht. Weiterhin soll festgelegt werden können, ob die Buchungen tatsächlich als abgerechnet markiert werden sollen, oder nicht.

\subsubsection{Geschäftsstatistik}
Die Geschäftsstatistik ist ein Bericht der Statistiken von den Downloads von verschiedenen Kunden darstellt.
Durch die jeweilige Darstellung mehrerer Kunden, kann der Bericht nur im Kundengruppenrenderer erstellt werden.
Die angezeigten Statistiken stellen Informationen zu der Aktivität der Benutzer, den Kostenarten der Produkte, den Verwendungszwecken, den Angaben für den Jahresbericht, sowie eine Übersicht über die bezogenen Produkte dar.
Eine Vorlage des Berichts kann den Abbildungen \vrefrange{fig:vorlage_geschaeftsstatistik_1}{fig:vorlage_geschaeftsstatistik_2} entnommen werden.
 
\subsection{sonstige neu benötigte Klassen} \label{subsec:neue_klassen}
Der Hauptgrund warum diese, hier aufgeführten, Klassen eingeführt werden ist, die Unterstützung des Benutzers bei der Eingabe von den jeweiligen Werten.
Mit diesen Klassen kann dem Benutzer eine vorgefertigte Auswahl an Elementen angeboten werden.
Somit müssen die entsprechenden Felder nicht händisch in der GUI eintragen werden.
Diese Auswahl beschleunigt einerseits das Eintragen von Werten und verhindert andererseits Fehleingaben. 

Zum Einführen der Klasse Kunde und zum Erweitern der Buchungen werden folgende zusätzlichen Klassen benötigt:
\begin{description}
\item[Abrechnungsturnus] Der Abrechnungsturnus gibt an wie häufig ein Kunde eine Abrechnung erhält. Sein einziges Feld ist die Bezeichnung des Abrechnungsturnus (z.B. Jahr, Quartal)
\item[Produkt] Ein Produkt repräsentiert die verschiedenen Softwareprodukte, die ein Kunde benutzen kann. Seine Felder sind eine Bezeichung des Produktes, eine Beschreibung sowie der konkrete Client, mit dem sich der Benutzer anmelden kann. Beim Client handelt es sich ebenfalls um eine Klasse.
\begin{description}
\item[Client] Ein Client kann von verschiedenen Produkten benutzt werden. Sein einziges Feld ist seine Bezeichnung.
\end{description}
\item[Stornogrund] Ein Stornogrund ist der Grund, warum eine Buchung storniert wurde. Sein einziges Feld ist die Bezeichnung des Grundes.
\end{description}

Bei diesen Klassen sollen die sogenannten Standardrenderer und -editoren verwendet werden. Diese werden für jede Klasse automatisch erstellt und müssen nicht eigens implementiert werden.
Mit dem Standardeditor ist es möglich neue Objekte einer Klasse anzulegen und sämtliche Felder der Klasse verändern.
Diese Anforderungen genügen allerdings für die hier aufgelisteten Klassen.

