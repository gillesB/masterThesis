%% change \numberline so that it will print "Appendix A"
\part*{Appendix}
\appendix
\chapter{Datenbankmodell}
\begin{sidewaysfigure}
	\centering
	\incgraph{width=\textwidth}{./rimg/db_model_two.png}
	\caption{Früher Stand des Datenbankmodells}
	\label{fig:db-two}
\end{sidewaysfigure}
\begin{sidewaysfigure}
	\centering
	\incgraph{width=\textwidth}{./rimg/db_model_three.png}
	\caption{Fortgeschrittener Stand des Datenbankmodells}
	\label{fig:db-three}
\end{sidewaysfigure}

\chapter{Beispielprodukte}
\begin{figure}[htbp]
	\centering
	\incgraph{width=\textwidth}{./rimg/beispiel-flurkarte.pdf}
	\caption{Flurkarte}
	\label{fig:flurkarte}
\end{figure}
\begin{figure}[htbp]
	\centering
	\incgraph{width=\textwidth}{./rimg/prod_flurstuecks_und_eigentumsnachweis_nrw.png}
	\caption{Flurstücks- und Eigentumsnachweis (NRW)}
	\label{fig:eigentumsnachweis_nrw}
\end{figure}
\begin{figure}[htbp]
	\centering
	\incgraph{width=\textwidth}{./rimg/prod_flurstuecks_und_eigentumsnachweis_kom.png}
	\caption{Flurstücks- und Eigentumsnachweis (kommunal)}
	\label{fig:eigentumsnachweis_kom}
\end{figure}
\begin{figure}[htbp]
	\centering
	\incgraph{width=\textwidth}{./rimg/prod_bestandsnachweis_nrw.png}
	\caption{Bestandsnachweis (NRW)}
	\label{fig:bestandsnachweis_nrw}
\end{figure}
\begin{figure}[htbp]
	\centering
	\incgraph{width=\textwidth}{./rimg/beispiel-stadtgrundkarte.pdf}
	\caption{Stadtgrundkarte - mit kommunalen Ergänzungen}
	\label{fig:stadtgrundkarte}
\end{figure}
\begin{figure}[htbp]
	\centering
	\incgraph{angle=90, width=\textwidth}{./rimg/beispiel_punktnachweis.pdf}
	\caption{Punktnachweis}
	\label{fig:punktnachweis}
\end{figure}
\begin{figure}[htbp]
	\centering
	\incgraph{width=\textwidth}{./rimg/beispiel-digitale-grundkarte.pdf}
	\caption{Digitale Grundkarte}
	\label{fig:digitale_grundkarte}
\end{figure}
\begin{figure}[htbp]
	\centering
	\incgraph{width=\textwidth}{./rimg/beispiel-orthofoto-kataster.pdf}
	\caption{Orthofoto mit Katasterdarstellung}
	\label{fig:orthofoto-katasterdarstellung}
\end{figure}
\begin{figure}[htbp]
	\centering
	\incgraph{width=\textwidth}{./rimg/beispiel-nivp-uebersicht.pdf}
	\caption{NivP-Übersicht}
	\label{fig:nivp-uebersicht}
\end{figure}
\begin{figure}[htbp]
	\centering
	\incgraph{width=\textwidth}{./rimg/beispiel-ap-uebersicht.pdf}
	\caption{AP-Übersicht}
	\label{fig:ap-uebersicht}
\end{figure}
\begin{figure}[htbp]
	\centering
	\incgraph{width=\textwidth}{./rimg/beispiel-punktnumerierungsuebersicht.pdf}
	\caption{Punktnumerierungsübersicht}
	\label{fig:punktnumerierungsuebersicht}
\end{figure}
\begin{figure}[htbp]
	\centering
	\incgraph{width=\textwidth}{./rimg/beispiel-ap-karte.jpg}
	\caption{AP-Karte}
	\label{fig:ap-karte}
\end{figure}
\begin{figure}[htbp]
	\centering
	\incgraph{width=\textwidth}{./rimg/beispiel-nivp-beschreibung.jpg}
	\caption{NivP-Beschreibung}
	\label{fig:nivp-beschreibung}
\end{figure}
\begin{figure}[htbp]
	\centering
	\incgraph{width=\textwidth}{./rimg/beispiel-fortfuehrungsriss.jpg}
	\caption{Fortführungsriss}
	\label{fig:fortfuehrungsriss}
\end{figure}
\begin{figure}[htbp]
	\centering
	\incgraph{width=\textwidth}{./rimg/bericht_vorlage_rechnungsanlage.png}
	\caption{Vorlage der Rechnungsanlage}
	\label{fig:vorlage_rechnungsanlage}
\end{figure}
\begin{figure}[htbp]
	\centering
	\incgraph{page=1, width=\textwidth}{./rimg/vorlage-statistik.pdf}
	\caption{Vorlage der Geschäftsstatistik (Seite 1)}
	\label{fig:vorlage_geschaeftsstatistik_1}
\end{figure}
\begin{figure}[htbp]
	\centering
	\incgraph{page=2, width=\textwidth}{./rimg/vorlage-statistik.pdf}
	\caption{Vorlage der Geschäftsstatistik (Seite 2)}
	\label{fig:vorlage_geschaeftsstatistik_2}
\end{figure}
\begin{figure}[htbp]
	\centering
	\incgraph{width=\textwidth}{./rimg/beispiel_karte_rathaus.pdf}
	\caption{Von der Cismap generiertes Foto des Rathauses}
	\label{fig:cismap-rathaus}
\end{figure}



