\chapter{Workflow von Projektentwickler} \label{ch:aussicht}

In diesem Kapitel wird der Workflow von Projektentwicklern untersucht um Funktionalitäten zu finden, die \ac{WuNDa} noch nicht anbietet, jedoch deren Arbeitsalltag vereinfacht.
Das Ziel hierbei ist es \ac{WuNDa} für diese attraktiver zu machen, da es sich bei den Projektentwickler um eine Kundengruppe handelt, die das Katasteramt als externe Benutzer gewinnen will. 

Um dies zu verdeutlichen wird zuerst auf die Motivation des Katasteramtes eingegangen, danach wird die Kundengruppe der Projektentwickler näher erläutert.
Daraufhin werden mögliche neue Funktionalitäten in \ac{WuNDa} kurz beschrieben und anschließend wird eine Auswahl von diesen detaillierter beleuchtet.

Die Informationen zu diesem Kapitel stammen aus einem gemeinsamen Treffen zwischen dem Katasteramt, dem Geschäftsleiter des Unternehmens Colemus \autocite{colemus}, der als Projektentwickler tätig ist, sowie einem Architekten und einem \ac{ÖbVI}, welche häufig mit Colemus zusammenarbeiten. Der Autor dieser Arbeit war bei diesem Treffen ebenfalls anwesend.

\section{Motivation des Katasteramtes.}

Durch die bereits beschriebene Rationalisierung innerhalb des Katasteramtes, wird es nötig, dass mehr Kunden des Katasteramtes ihre Informationen über \ac{WuNDa} beziehen und somit die Beamten des Katasteramtes entlasten.
Aus diesem Grund soll die Gesamtanzahl der externen Benutzer erhöht werden.
Dabei hat sich das Katasteramt das Ziel gesetzt für diese Benutzer, trotz des Personalabbaus, die Nutzbarkeit zu erhöhen, wobei der Umsatz, der durch die eingenommenen Gebühren, gemacht wird in etwa gleich bleiben soll.
Im Augenblick benutzen nur die Polizei und die \acp{ÖbVI} \ac{WuNDa} als externe Benutzer.
Dieses hat bei diesen Berufsgruppen bereits eine breite Akzeptanz gefunden, so dass hier nur wenige weitere externe Benutzer erschlossen werden können.
Somit muss das Katasteramt Anstrengungen unternehmen um \ac{WuNDa} interessant für andere Berufsgruppen zu machen.
Dabei werden Gespräche mit Repräsentanten dieser Gruppen geführt um diesen den aktuellen Stand von \ac{WuNDa} vorzuführen und um eventuelle neue Funktionalitäten ausfindig zu machen.
Falls die Funktionalitäten realisierbar sind, so sollen diese auch umgesetzt werden, so dass schlussendlich die Attraktivität von \ac{WuNDa} erhöht wird und mehr Zuspruch bei den einzelnen Kunden findet.

Das Katasteramt hat einige Kriterien für die möglichen Funktionalitäten festgelegt. 
So soll insgesamt der Zugriff auf die benötigten Informationen, für die Kunden, schneller und direkter werden.
Weiterhin soll möglichst wenig zusätzliche Arbeit für das Katasteramt anfallen.
So sollen z.\,B. keine neuen Datenbestände erstellt werden, die zusätzlich gepflegt werden müssen.
Dies bedeutet im Umkehrschluss, dass lediglich Datenbestände benutzt werden können die das Katasteramt aus anderen, z.\,B. rechtlichen, Gründen pflegen muss.
Weiterhin muss bei den Funktionalitäten darauf geachtet werden dass keine rechtliche Gegebenheiten verletzt werden, so z.\,B. das Datenschutzrecht.

Zu den angesprochenen Berufsgruppen gehören die Projektentwickler, welche in dem nächsten Abschnitt beleuchtet wird.


%\begin{itemize}
%	\item Oben wurde beschreiben warum die ent. Benutzer Daten anfordern müssen, was der Grunde der gewünschten Änderungen war.
%	\item Nun wird ein Schritt weiter gedacht
%	\item Quelle: Gespräch mit Projektentwickler
%\end{itemize}

\section{Projektentwickler}

Unter Projektentwickler versteht man eine Berufsgruppe, die größere Immobilienprojekte konzipieren und erstellen.
Dabei organisieren sie die einzelne Schritte des Projektes, von der Suche nach geeigneten Grundstücken, die Finanzierung und Planung des Projektes bis zum Fertigstellen der schlüsselfertigen Immobilie, sowie deren Vermarktung.
Die meisten Teilaufgaben werden an externe Dienstleister weitergeleitet, wie etwa Architekten, Vermessungsingenieuren oder Makler.

Besonders interessant für das Katasteramt sind die kleinen bis mittleren Projektentwickler, da diese standortgebunden sind und somit lediglich Projekte in Wuppertal entwickeln können.
Zu dieser Zielgruppe gehört z.\,B. das Unternehmen Colemus, das in einem durchschnittlichen Projekt circa zehn Wohnhäuser baut.
Diese Gebundenheit der Projektentwickler ist einerseits auf die Beschaffung des Kredites zurückzuführen, da lediglich die lokalen Sparkassen solche Projekte finanzieren.
Ein weiterer Grund ist die Kenntnis der Verwaltungswegen an einem Standort, die durchlaufen werden müssen, um ein solches Projekt durchzuführen.
Diese können von Region zu Region sehr unterschiedlich sein.
Ein dritter Grund sind die Beziehungen zu den Partnern, so arbeitet das Unternehmen Colemus bevorzugt immer mit den gleichen Dienstleistern zusammen.

Die Projektentwickler sind durch ihre verschiedene Dienstleister mit denen sie zusammenarbeiten eine interessante Kundengruppe für das Katasteramt.
Durch sie steigt der Bekanntheitsgrad von \ac{WuNDa} bei ihren Dienstleister die dadurch ebenfalls als externe Benutzer gewonnen werden können.
Dies zeigte sich bereits im Gespräch, da hier solche Dienstleister ebenfalls am Tisch saßen und von \ac{WuNDa} angesprochen wurden, obwohl sie nicht die direkte Zielgruppe waren.

%Weiterhin sind die Projektentwickler interessant für die Stadt Wuppertal an sich.
%So werden diese von einer guten Unterstützung einer Stadt angezogen, da diese somit mehr Projekte in kürzerer Zeit durchführen können, wovon die Stadt schlussendlich profitiert da in sie investiert wird und der Immobilienmarkt angetrieben wird.
%Zu dieser Unterstützung gehören möglichst einfache und schnelle Verwaltungswege sowie eine gute Zusammenarbeit zwischen den verschiedenen Ämtern.
%Kurzum das schnelle Beziehen von Informationen über \ac{WuNDa} gehört auch zu dieser Unterstützung, die das Katasteramt anbieten und unter Umständen noch ausbauen kann.

Eine Aufgabe bei der Entwicklung von Projekten besteht in der Überzeugung von Entscheidern.
Dies ist u.\,a. der Fall, wenn das Projekt bei Kreditinstituten und verschiedenen Ämtern vorgestellt wird und mit Dokumenten hinterlegt werden muss um die benötigten Kredite und Genehmigungen zu erhalten.
Dabei müssen diese Dokumente eine möglichst professionelle Form und Aussehen besitzen, so dass die Projekte einen seriösen und durchdachten Eindruck hinterlassen.
Aus diesem Grund befindet sich in den Featurewünschen der eine oder andere Wunsch um das Erstellen von solchen Dokumenten direkt oder indirekt zu vereinfachen.
Mit einer indirekten Unterstützung ist z.\,B. das Anzeigen von zusätzlichen Kartenlayer gemeint.
Diese Layer können mit Hilfe von der bereits existierenden Export-Funktion der Cismap  einfach in Dokumente eingebunden werden, da diese Funktion es ermöglicht einen Ausschnitt der Karte als Bild- oder PDF-Datei abzuspeichern. 

Aus diesem Grund wird, falls die Projektentwickler als externe Benutzer und ihre Dienstleister gewonnen werden können, in Zukunft \ac{WuNDa} dahingehend entwickelt, so dass es attraktiver für diese Kundengruppen wird.
Deshalb wurde in dem gemeinsamen Gespräch auch auf die fehlenden Funktionalitäten eingegangen, die in den nächsten zwei Abschnitten behandelt werden.

\section{Aufgekommene Featurewünsche}
In diesem Abschnitt werden die Features kurz beschrieben, die während dem Gespräch geäußert wurden und die in \ac{WuNDa} noch nicht realisiert wurden.

\subsubsection{Plotrahmen anpassen}
Der Benutzer kann zu jedem Zeitpunkt einen Ausschnitt der Cismap, sprich der Karte in \ac{WuNDa}, in eine PDF-Datei speichern.
Eine solche PDF-Datei kann in Abbildung \vref{fig:cismap-rathaus} betrachtet werden.
Ein jeder solcher Abdruck der Cismap besitzt einen sogenannten Plotrahmen. In diesem Rahmen befinden sich einige Informationen zu dem abgebildeten Bild.
Im Augenblick ist es möglich in \ac{WuNDa} einen Text anzugeben, der dann in dem mittleren Feld des Rahmens abgebildet wird.
Eine neue Funktionalität wäre es den Rahmen individueller gestalten zu können, so dass externe Benutzer z.\,B. ihr Logo in dem Rahmen abbilden können.
Dabei soll das Logo des Katasteramtes erhalten bleiben, da es die Herkunft des Bildes angibt und somit die Richtigkeit des Bildes gewährleistet. 

\subsubsection{Zugriff auf Daten der WSW}
Die \ac{WSW} bietet Interessenten eine kostenlose Auskunft bezüglich den Daten, die das \ac{WSW} verwaltet.
Hierzu gehören u.\,a. Trassen, wie etwa Strom- und Gasleitungen, Kanaldaten aber auch Buslinien.
Für die Projektentwickler und seine Dienstleister wäre es nützlich diese Daten auch in \ac{WuNDa} importieren zu können, um sie z.\,B. in der Cismap anzeigen zu lassen.
Aus technischer Sicht wäre dies machbar, allerdings scheitert eine Umsetzung  im Augenblick an den Verhandlungen zwischen dem Katasteramt und dem \ac{WSW}.
Aus diesem Grund stehen die Daten nicht zur Verfügung und können folglich nicht in \ac{WuNDa} eingebunden werden.
\subsubsection{Zugriff auf Daten des Grundbuches}
Die Projektentwickler und ihre Dienstleister müssen regelmäßig Dokumente vom Grundbuchamt beziehen.
Theoretisch könnte dies in \ac{WuNDa} ähnlich, wie das Beziehen von Produkten, umgesetzt werden.

Ein solcher einfacher Zugriff auf die Daten des Grundbuchamtes scheitert allerdings an dem starken Datenschutz, dem das Grundbuch und insbesondere sein Online-Zugriff unterliegt.
So dürfen, per Gesetz, z.\,B. nur Gerichte, Notare und \acp{ÖbVI} uneingeschränkt auf die Internet-Grundbucheinsicht zugreifen.
Andere Nutzungsbrechtigte, wie etwa Banken und Rechtsanwälte, können die Internet-Grundbucheinsicht lediglich eingeschränkt nutzen.
Die bedeutet dass sie vor jeder Einsicht ein berechtigtes Interesse angeben müssen \autocite[vgl.][]{justiz-grundbucheinsicht}.

 
\subsubsection{Abfragen eines berechtigten Interesse}
Das Katasteramt besitzt ebenfalls Daten, die erst ausgegeben werden dürfen, nachdem ein berechtigtes Interesse angegeben wurde, dies ist insbesondere der Fall bei Baulasten.
Diese Funktionalität würde dem Katasteramt und den externen Kunden Zeitaufwand sparen und wird im \autoref{subsec:berechtigtes-interesse} näher behandelt.

\subsubsection{Erschließungsbeiträge}
Die Erschließungsbeiträge werden für die Herstellung von Straßen, Wege, Plätzen, etc. erhoben und müssen von den Eigentümern oder Erbbauberechtigten der anliegenden Grundstücken bezahlt werden.
Weiterhin werden die Erschließungsbeiträge von dem Ressort Straßen und Verkehr beantragt \autocite[vgl.][]{wupp-erschliessungsbeitrag}.

In \ac{WuNDa} werden die Erschließungsbeiträge nicht abgebildet, diese Funktionalität wäre jedoch nützlich für die Projektentwickler und ihre Dienstleister.
Zusätzlich wurde die Funktionalität, in einem anderen Gespräch, von der Kundengruppe der Gutachter ebenfalls angefragt.
Somit handelt es sich hierbei um eine Funktionalität deren Umsetzung näher betrachtet werden sollte, insbesondere da die Erschließungsbeiträge von einem Ressort der Stadtverwaltung Wuppertal verwaltet wird.

\subsubsection{Abfrage der Dachhöhen}
In \ac{WuNDa} ist es möglich die Dachhöhen abzufragen, diese Funktionalität wurde allerdings noch nicht für die externen Benutzer freigegeben.
Sie wäre jedoch nützlich für die Projektentwickler und ihre Dienstleister um ein neues Gebäude ins bestehende Stadtbild einzufügen.
Mehr Informationen hierzu befinden sich im \autoref{subsec:dachhoehen}.

\subsubsection{3D-Ansicht}
Im Gespräch kam auf, dass eine 3D-Ansicht der Stadt Wuppertal nützlich, aber nicht unbedingt notwendig wäre.
Diese Ansicht könnte z.\,B. dafür verwendet werden um sich ein Bild der Lage zu verschaffen, oder um in Dokumenten darzustellen, wie sich die Gebäude eines Projektes, in die aktuelle Umgebung einfügen.

Diese Funktionalität wird allerdings nicht in näher Zukunft realisiert werden, da sie sehr kostspielig ist.
Zum einen ist hier die Technik in \ac{WuNDa} noch nicht ausgreift, insbesondere fehlt aber ein vollständiges 3D-Modell der Stadt Wuppertal, welches zuerst angefertigt werden müsste.
Weiterhin gibt es für das Katasteramt selbst keinen konkreten Anwendungsfall, welcher eine solche Investition rechtfertigen würde.  

\subsubsection{Verwaltung von eigenen Stammdaten}
Im Augenblick haben sämtliche Daten, die von \ac{WuNDa} benutzt werden, einen direkten Bezug zu dem R102.
Durch das Aufkommen der externen Benutzer sind Anwendungsfälle denkbar, bei denen diese eigene Daten in \ac{WuNDa} anzeigen und verwalten wollen.
Ein einfaches Beispiel hierzu wäre das Hinzufügen und Verwalten von Bookmarks für häufig benutzte Objekte oder angezeigte Stellen in der Karte.
Andere Anwendungsfälle mit persönlicheren Daten sind ebenfalls denkbar, hier wurde im Gespräch bereits deutlich dass solche Daten im Besitz der externen Benutzer bleiben muss und nicht, wie vorgeschlagen, von dem R102 verwaltet werden kann.
Welche Funktionalitäten hier genau gewünscht sind, muss in weiteren Gesprächen mit den externen Benutzern ermittelt werden, da hierzu noch keine näheren Angaben gegeben werden konnten.
\subsubsection{Freigeben der Layer für Problemimmobilien und Schrottimmobilien}
Das R102 besitzt einige Kartenlayer die für die Öffentlichkeit nicht freigegeben sind.
Im Gespräch wurde hier das Beispiel der Problem- und Schrottimmobilien genannt.
Als Schrottimmobilien werden Immobilien bezeichnet, die in einem minderwertigen oder mangelhaften Zustand deutlich über ihren Wert verkauft wurden.
Hierunter fallen häufig Gebäude, die nicht vollständig oder nur mangelhaft saniert wurden und dann an Privatkunden verkauft wurden \autocite[vgl.][]{iva-schrottimmo}.
Unter Problemimmobilien werden auf Grund von Mängel nur schwer verkäufliche Immobilien verstanden \autocite[vgl.][]{immo-problemimmo}.

Das Wissen um solche Immobilien ist nützlich für Projektentwickler, da diese sich häufig auf Grundstücken befinden, die für ein neues Projekt interessant sind.
So kann z.\,B. das Grundstück einer Problemimmobilie für einen Neubau geeignet sein.
Andererseits ist eine Gegend mit zu vielen solcher Immobilien eher zu meiden.

Die entsprechenden Layer können jedoch nicht ohne weiteres freigegeben werden, da auch hier  der Datenschutz beachtet werden muss.
Nichtsdestoweniger besitzt die Stadt Wuppertal selbst solche Immobilien, so dass der Datenschutz in diesem Fall keine Rolle spielt und die Layer in einem ersten Schritt eingeschränkt zur Verfügung gestellt werden könnten.
  
\subsubsection{Anzeigen von zukünftige Vorhaben}
Informationen zu zukünftigen Vorhaben auf Grundstücken, können das Investitionspotential einer Gegend erhöhen.
So wurde z.\,B. im vorherigen Abschnitt erwähnt, dass zu viele Schrottimmobilien in einer Gegend einen Projektentwickler davon abschrecken können in diese Gegend zu investieren.
Dem könnte entgegengewirkt werden in dem die Projektentwickler Informationen dazu hätten, welche Vorhaben bereits auf diesen Grundstücken in Planung sind, wie etwa eine Komplettsanierung von solchen Immobilien oder das Erstellen von Neubauten auf diesen Grundstücken.
Weitere Informationen zu anderen zukünftigen Vorhaben, wären aber auch hilfreich, wie etwa welche Grundstücke in Bauland umgewandelt werden oder entwickelt werden. 
Dadurch hätten die Projektentwickler einen Einblick in die nahe Zukunft und die Entwicklung einer Gegend und könnten somit ein Projekt dort vorbereiten, obwohl diese Gegend zu diesem Zeitpunkt noch nicht interessant ist.
Kurzum das Investitionspotential einer Gegend wäre lediglich durch Informationen gestiegen, die zwar bereits vorhanden, jedoch in solcher Form noch nicht veröffentlicht oder nur schwer zugänglich sind.

Diese Funktionalität wäre aus technischer Sicht umsetzbar, da lediglich zusätzliche Informationen zu einem Grundstück angezeigt werden müssten.
Dies könnte über neue Layer oder neue Objekte in \ac{WuNDa} realisiert werden.
Allerdings gibt es diese Art der Informationen im Augenblick nicht und müssten in einem ersten Schritt zusammengetragen werden und in Zukunft auf einem aktuellen Stand gehalten werden.
Aus diesem Grund müsste deshalb zuerst analysiert werden, welche Daten relevant sind, in welchen Ämtern diese erfasst werden und es müssten Gespräche mit diesen Ämtern für dieses Kooperationsprojekt angestoßen werden.
 
\subsubsection{Luftlinienentfernung von einem Grundstück zu interessanten Orten}
Für die Projektentwickler und die Vertreiber der Immobilien spielt die Qualität der Lage eine wichtige Rolle, da über diese den Endpreis mitbestimmt wird und sich somit auch in deren Rendite widerspiegelt.
Eine Möglichkeit die Qualität der Lage zu bestimmen, ist es die Luftlinienentfernung zu verschiedenen interessanten Orten für die späteren Bewohner zu berechnen.
Die Art dieser Orte unterscheidet sich je nach Zielgruppe, typischerweise handelt es sich um Orte wie Schulen, Supermärkte, Ärzte, Freizeitbeschäftigungen, Bushaltestellen etc.

Über die Cismap kann in \ac{WuNDa} die Entfernung zwischen zwei Punkten berechnet werden.
Trotzdem kann diese Funktionalität nicht sofort umgesetzt werden, da Informationen bezüglich diesen interessanten Orten fehlen.
Weiterhin müssten auch die verschiedenen Zielgruppen bestimmt werden sowie ihr Interessegrad an den verschiedenen Orten.
So sind z.\,B. junge Eltern interessierter an Schulen (Vor-, Grund- und Sekundarschulen) als Senioren.
Mit diesen Informationen könnte dann bestimmt werden, für welche Zielgruppe ein bestimmtes Grundstück am interessantesten ist und die Planung könnte darauf angepasst werden.
Ein weiterer Anwendungszweck, wäre das Anzeigen dieser Entfernung in einem Bericht, so dass ein Vertreiber einer Immobilie diesen für Präsentationszwecke benutzen kann.

\subsubsection{Anzeigen des Baujahres}
Ein weiterer Kartenlayer der im Gespräch aufkam war das Anzeigen des Baujahres in Zehnjahres-Gruppen.
Dadurch ist sofort ersichtlich wie eine Gegend entstanden ist und wann die letzten Investitionen in ihr getätigt wurden.


\section{Beleuchtung einiger Features}

In diesem Abschnitt werden einige gewünschte Features fachlich näher beleuchtet und ihre potentielle Umsetzung in \ac{WuNDa} wird beschrieben.

\subsection{Abfragen eines berechtigten Interesses}
\label{subsec:berechtigtes-interesse}
Das Katasteramt besitzt Informationen deren Einsicht erst nach einer Hinterlegung eines berechtigten Interesses herausgegeben werden dürfen.
Zu diesen Informationen gehören die Baulasten.

Bei Baulasten handelt es sich um Eintragungen in das Baulastenverzeichnis, die getätigt werden müssen, wenn ein Bauvorhaben nicht baurechtskonform auf einem Baugrundstück durchgeführt werden kann und andere Grundstücke mit herangezogen werden müssen, so dass das Bauvorhaben genehmigt werden kann.
Baulasten werden in unterschiedlichen Fälle benötigt, so z.\,B. für die \enquote{Abstandflächenregelung} oder bei der baurechtliche \enquote{Vereinigung}.
Eine Baulast ist eine freiwillige Erklärung eines Grundstückseigentümers, die ihn danach zu einem bestimmten Tun, Dulden oder Unterlassen bezüglich des Grundstückes verpflichtet.
Dadurch kann eine solche Baulast mit einer Wertminderung des Grundstückes einhergehen, da z.\,B. seine Bebaubarkeit eingeschränkt wird.
In anderen Fällen kann der Wert des Grundstückes jedoch auch steigen.
Die Baulasten für ein Grundstück bleiben auch nach einem Eigentümerwechsel bestehen und können nur dann gelöscht werden falls zwischenzeitlich Ereignisse eingetreten sind, wodurch das Bauvorhaben auch ohne Baulast baurechtskonform ist \autocite[vgl.][]{herne-baulasten}.

Dadurch dass Baulasten bei einem Eigentümerwechsel nicht verfallen, sind sie für den Projektentwickler bereits bei der Grundstücksauswahl für ein neues Projekt von Bedeutung, da sich erst durch sie herausstellt, ob ein Grundstück für ein Projekt eignet oder nicht. 
Demnach ist eine Einsicht, ob und mit welchen Baulasten ein Grundstück belastet ist, notwendig, weiterhin müssen die Dokumente dieser Baulast in einer späteren Phase bezogen werden.

Um Informationen bezüglich der Baulasten zu erhalten, muss in einem ersten Schritt beim  Wuppertaler Baulastenverzeichnis ein berechtigtes Interesse an einem Grundstück hinterlegt werden.
Ein solches Interesse haben u.\,a. Kaufinteressenten, Notare und \acp{ÖbVI}.
Danach kann der Interessent eine kostenfreie mündliche oder telefonische Auskunft erhalten, diese Auskunft kann schriftlich bestätigt werden, d.\,h. er erhält einen schriftlichen Auszug  aus dem Baulastenverzeichnis.
Ist eine Baulast auf dem Grundstück vorhanden, so kostet diese Auskunft \EUR{50} bis \EUR{150}.
Falls keine Baulast vorhanden ist, so kann er sich dies auch bescheinigen lassen. Diese schriftliche Bescheinigung kostet \EUR{10} \autocite[vgl.][]{wupp-baulast}.

Der aktuelle Stand der Baulasten in \ac{WuNDa} ist, dass diese Funktionalität in der Entwicklung ist und in naher Zukunft fertiggestellt wird.
Die fertiggestellte Funktionalität wird es den Beamten des Katasteramtes ermöglichen die bestehenden Baulasten in der Karte anzuzeigen, nach Baulasten zu suchen, die einzelnen Baulasten anzeigen zu lassen und zu editieren sowie die Dokumente, die zu den Baulasten gehören, herunterzuladen.

Leicht modifiziert bieten sich diese Funktionalitäten demnach auch für die Projektentwickler an, allerdings wird hierzu zusätzlich eine Abfrage für das berechtigte Interesse benötigt.
Der Ablauf des Beziehens einer Baulast für die externen Benutzer, aber natürlich auch für andere Kundengruppen könnte wie folgt aussehen:
\subsubsection{Suche nach Baulasten}
In einem ersten Schritt muss der Benutzer die Grundstücke angeben, von denen er die Baulasten haben will.
Dies kann über die Baulast-Suche passieren, diese müsste allerdings soweit eingeschränkt sein dass sie keine anderen Angaben als die Grundstücke entgegen nimmt, wie dies der Abbildung \ref{fig:baulast-suche} zu entnehmen ist.
Die dort eingegeben Flurstückskennzeichen kann z.\,B. über die Flurstücks-Suche ermittelt werden. Die Suche kann daraufhin über den Knopf \enquote{Suchen} gestartet werden.

\begin{figure}[htbp]
	\centering
	\incgraph{width=0.5\textwidth}{./rimg/baulast_suche_disabled.png}
	\caption{Eingeschränkte Baulast-Suche}
	\label{fig:baulast-suche}
\end{figure}

\subsubsection{Abfrage des Berechtigten Interesses}
Ehe die eigentliche Suche gestartet wird, muss das berechtigte Interesse des Benutzers abgefragt werden, so dass sichergestellt wird, dass er berechtigt ist um eine Einsicht in das Baulastenverzeichnis zu erhalten.
Die genaue Vorgehensweise wie ein berechtigtes Interesse festgestellt wird entzieht sich dem Wissen des Autors \todo{fehlendes Wissen}.
Auf der Grundlage von den bereits erwähnten Quellen wird aber angenommen, dass:
\begin{itemize}
\item es je nach Kundengruppe einige immer wieder auftretende Gründe gibt,
\item diese können, oder müssen sogar, in einem Kommentarfeld näher spezifiziert werden
\item und in einigen Fällen müssen Dokumente hinterlegt werden.\todo{Quelle}
\end{itemize}
Demnach könnte die Abfrage des berechtigten Interesses ähnlich realisiert werden, wie dies der Fall bei der Abrechnungskomponente realisiert ist, also mit einem vorgeschalteten Dialog.
Demnach wird ein Dialog benötigt, der anzeigt für welche Flurstücke das berechtigte Interesse gelten soll, eine Auswahl für den Grund, ein Kommentarfeld sowie eine Möglichkeit um ein Dokument hochzuladen.
Je nach Benutzergruppe kann sich dabei die Auswahl der Gründe ändern und die anschließend auszufüllende Angaben könnten vom Grund abhängig gemacht werden.
Beim Bestätigen des Dialoges werden die angegebenen Daten protokolliert, so dass später nachvollzogen werden kann, welcher Benutzer für welche Grundstücke die Baulasten angefragt hat.

Bei einer solchen automatischen Abfrage stellt sich die Frage, wie die getätigten Eingaben kontrolliert werden, um einen unberechtigten Zugang zu verhindern, hierzu folgen nun einige Vorschläge.
Eine denkbare Kontrolle wäre das regelmäßige, stichprobenartige Prüfen der Interesse-Protokolle durch einen Beamten.
Durch diese Maßnahme wird zwar ein Beamte an diese Aufgabe gebunden, da jedoch nicht sämtliche Interessen bearbeitet werden müssen, wird dennoch Zeit gespart.
Diese Arbeit könnte von \ac{WuNDa} zusätzlich unterstützt werden, indem es versucht Unregelmäßigkeiten in den getätigten Eingaben zu finden.
So könnte z.\,B. überprüft werden ob ein Benutzer ungewöhnlich viele Anfragen in einem bestimmten Zeitraum gestellt hat.
Weiterhin könnte überprüft werden, ob nicht versucht wird möglichst viele Baulasten mit einer Anfrage zu erhalten.
Zu diesem Zweck könnte überprüft werden, ob die ausgewählten Flurstücke örtlich zusammenhängen oder ob die Fläche der Flurstücke in einer Anfrage ungewöhnlich groß ist.
In den Fällen, bei denen ein Dokument hochgeladen werden muss, besteht jeweils die Möglichkeit eine Plausibiltätsprüfung durchzuführen.
So könnte überprüft werden, ob die Dateien nicht leer sind und ob es sich jeweils um eine PDF-Datei handelt.
Weiterhin könnte über Checksummen geprüft werden, ob nicht immer die gleiche Datei hochgeladen wurde.
Je nach angegeben Grund könnte auch überprüft werden, ob die Datei ein typisches Wort für das erwartete Dokument enthält.

\subsubsection{Einsicht und Beziehen der Dokumente}
Nachdem das berechtigte Interesse abgefragt und protokolliert wurde, kann dem externen Benutzer die gefundenen Baulasten für die angegebenen Grundstücke in \ac{WuNDa} angezeigt werden.
Da die Einsicht auch im Katasteramt kostenlos ist, müssen hier keinen weiteren Änderungen vorgenommen oder bestimmte Maßnahmen getroffen werden.
Über die Renderer der Baulasten wird es dann möglich sein die entsprechenden, kostenpflichtigen Dokumente, über die Abrechnungskomponente, herunterzuladen.

\subsubsection{Zusammenfassung}
Mit der Möglichkeit das berechtigte Interesse über \ac{WuNDa} abzufragen, wird es den Projektentwicklern und anderen externen Benutzern möglich eine kostenlose Einsicht in das Baulastenverzeichnis zu erhalten.
Weiterhin können kostenpflichtige Dokumente zu den Baulasten über, den bereits an anderer Stelle verwendeten, Abrechnungsmechanismus heruntergeladen werden, so dass auch die schriftliche Auskunft ermöglicht wird.
Der einzige Fall der somit noch nicht abgedeckt wäre, ist das Beziehen der schriftliche Bescheinigung, falls keine Baulast auf einem Grundstück vorliegt.

Weiterhin muss vor einer Realisierung abgewogen werden, ob der Datenschutz mit der, vom Autor vorgeschlagenen, Vorsichtsmaßnahme gewährleistet ist und ob dieser nicht gegebenenfalls noch ausgebaut werden muss.

\subsection{Abfrage der Dachhöhen}
\label{subsec:dachhoehen}
Den Beamten des Katasteramtes ist es möglich die Höhenpunkte der Dachlandschaften von Wuppertal über \ac{WuNDa} zu beziehen.
Diese, circa zehn Jahre alten, Daten wurde mittels eines fotogrametrischen\footnote{\enquote{Verfahren zum Herstellen von Messbildern, Grund- und Aufrissen aus fotografischen Bildern} \autocite{duden-foto}}
Verfahrens erstellt und sind aus diesem Grund nur bis auf den Dezimeter genau.
Die Dachhöhen können somit benutzt werden, um neue Immobilien in das bestehende Stadtbild einzupassen und werden während der Planung eines Projektes von verschiedenen Ämtern angefordert.
Da der, in \ac{WuNDa} benutzte, Datenbestand nicht aktualisiert wird, in Wuppertal aber neue Gebäude entstehen, enthält der Bestand mittlerweile Lücken.
Dadurch, dass die Dachhöhen beim Bau eines neuen Gebäudes gemessen werden müssen, wäre eine Rückführung der Daten sinnvoll, da diese von anderen Stellen wiederverwendet werden könnten.
Im Augenblick werden die Dachhöhen für jeden Gebrauch wieder gemessen.
Diese präzisere Daten könnten dann auch für andere Berechnungen herangezogen werden, wie etwa bei der Berechnung von Abstandsflächen.

Das Freigeben der Dachhöhen für die Projektentwickler und ihre Dienstleister sollte keine Probleme darstellen, allerdings müsste in einem ersten Schritt überprüft werden, wie die Daten zurückgeführt werden könnten.

\chapter{Fazit und Ausblick}

\section{Fazit}

\section{Ausblick}
Neben den potentiellen Funktionalitäten für die externen Benutzer, existiert bereits eine konkrete Planung die Abrechnungskomponente für interne Zwecke zu benutzen.
Die Beamten des Geodatenzentrum müssen im Augenblick bei einem Beziehen eines Produktes für einen Kunden am Schalter zwei Systeme bedienen.
In einem ersten Schritt beziehen sie ein oder mehrere Produkte für den Kunden in \ac{WuNDa}.
In einem zweiten Schritt wählen sie die gleichen Produkte ein weiteres Mal in der Geschäftsbuchlösung GEORG der Firma GEOSOFT, die ebenfalls vom R102 benutzt wird, aus, dieses erstellt dann die Quittung für den Kunden.
Der zweite Schritt ist im Grunde unnötig und könnte automatisiert werden.
Dies ist prinzipiell möglich da die Preise der Produkte, aufgrund der Abrechnungskomponente auch in \ac{WuNDa} verwaltet werden und dem System dadurch bekannt sind.
Somit ist geplant die Abrechnungskomponente auch für diese Beamten einzuführen und wie folgt umgebaut werden.
Nachdem bestätigen des Download-Dialogs wird der Preis und die ausgewählten Produkte, z.\,B. in Form des \enquote{Buchungbelegs}, zusätzlich auch an GEORG übermittelt, welcher dann die Rechnung erstellt.
Wie diese Kommunikation zwischen den beiden Systemen genau von Statten geht, muss noch definiert werden, allerdings würde eine solche automatische Kommunikation weitere Zeitersparnisse bringen, da dieser Fall häufig eintritt und zusätzliche Arbeit beim Erstellen der Rechnung wegfällt.




