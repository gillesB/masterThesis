\chapter{Zukünftige Unterstützung der Workflows durch das System}

\begin{itemize}
	\item Oben wurde beschreiben warum die ent. Benutzer Daten anfordern müssen, was der Grunde der gewünschten Änderungen war.
	\item Nun wird ein Schritt weiter gedacht
\end{itemize}

\section{Überprüfung der Workflows}
\begin{itemize}
	\item Überprüfung dieser Workflows auf andere potentielle automatisierbare Teile
	\item Dabei Konzentration auf:
	\begin{itemize}
		\item \ac{ÖbVI}
		\item Polizei
		\item Sparkasse
	\end{itemize}
	\item Durch Aufdecken solcher auto. Teile können Folgeaufträge generiert werden, falls diese relevant sind. (Geld sparen)
	\item Beim Aufdecken wird z.b. geachtet auf
	\begin{itemize}
		\item Beziehen der Daten
		\item Verarbeiten der Daten
		\item Rücksenden der Daten
	\end{itemize}
	\item Kriterien sind u.a. Komfort, Zeitaufwand
	\item Betrachten dieser gefundenen auto. Teile
\end{itemize}
\subsection{\acl{ÖbVI}}
\subsection{Polizei}
\subsection{Sparkasse}

\section{Lösungsansätze zur Unterstützung}
\begin{itemize}
	\item Beschreiben dieser gefundenen Lösungsansätze, für jeden relevanten auto. Teil
\end{itemize}

\section{Bewertung der Lösungsansätze}
\begin{itemize}
	\item Einschätzen dieser Lösungsansätze auf Machbarkeit und Rentabilität
	\item Favorisieren der Lösungsansätze
\end{itemize}