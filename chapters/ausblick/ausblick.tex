\chapter{Gewinnen von neuen externen Benutzern} \label{ch:aussicht}

In diesem Kapitel wird beschrieben wie das R102 neue externe Benutzer gewinnen will. Dabei wird zuerst auf die Motivation des Katasteramtes eingegangen, danach wird eine solche Gruppe von möglichen externen Benutzer beschrieben, die Projektentwickler. Danach werden mögliche neue Funktionalitäten in \ac{WuNDa} kurz beschrieben und anschließend wird eine Auswahl von diesen detaillierter beleuchtet.

Die Informationen zu diesem Kapitel stammen aus einem gemeinsamen Treffen zwischen dem Katasteramt, einem Projektentwickler, dem Unternehmen Colemus \autocite{colemus}. Sowie einem Architekten und einem \ac{ÖbVI} die häufig mit Colemus zusammenarbeiten. Der Autor dieser Arbeit war bei diesem Treffen ebenfalls anwesend.

\section{Motivation des Katasteramtes.}

Durch die bereits beschriebene Rationalisierung innerhalb des Katasteramtes, wird es nötig, dass mehr Kunden des Katasteramtes ihre Informationen über \ac{WuNDa} beziehen und somit die Beamten des Katasteramtes entlasten.
Somit soll die Gesamtanzahl der externen Benutzer erhöht werden.
Dabei hat sich das Katasteramt das Ziel gesetzt für diese Benutzer, trotz des Personalabbaus, die Nutzbarkeit zu erhöhen, wobei der Umsatz, der durch die eingenommenen Gebühren, gemacht wird in etwa gleich bleiben soll.
Im Augenblick benutzen nur die Polizei und die \acp{ÖbVI} \ac{WuNDa} als externe Benutzer.
Dieses hat bei diesen Berufsgruppen bereits eine breite Akzeptanz gefunden, so dass hier nur wenige weitere externe Benutzer erschlossen werden können.
Somit muss das Katasteramtes Anstrengungen unternehmen um \ac{WuNDa} interessant für andere Berufsgruppen zu machen.
Dabei werden Gespräche mit Repräsentanten dieser Gruppen geführt um diesen den aktuellen Stand von \ac{WuNDa} vorzuführen und um eventuelle neue Funktionalitäten ausfindig zu machen.
Falls die Funktionalitäten realisierbar sind, so sollen diese auch umgesetzt werden, so dass schlussendlich die Attraktivität von \ac{WuNDa} erhöht wird und mehr Zuspruch bei den einzelnen Kunden aus diesen Berufsgruppen findet.

Das Katasteramt hat einige Kriterien für die neuen Funktionalitäten festgelegt. 
So soll insgesamt der Zugriff auf die benötigten Informationen, für die Kunden, schneller und direkter werden.
Weiterhin soll möglichst wenig zusätzliche Arbeit für das Katasteramt anfallen.
So sollen z.\,B. keine neuen Datenbestände erstellt werden, die zusätzlich gepflegt werden müssen.
Dies bedeutet im Umkehrschluss, dass lediglich Datenbestände benutzt werden können die das Katasteramt aus anderen, z.\,B. rechtlichen, Gründen pflegen muss.
Weiterhin muss bei den Funktionalitäten darauf geachtet werden dass keine rechtliche Gegebenheiten verletzt werden, so z.\,B. das Datenschutzrecht.

Zu den angesprochen Berufsgruppen gehören die Projektentwickler, welche in dem nächsten Abschnitt beleuchtet werden.


%\begin{itemize}
%	\item Oben wurde beschreiben warum die ent. Benutzer Daten anfordern müssen, was der Grunde der gewünschten Änderungen war.
%	\item Nun wird ein Schritt weiter gedacht
%	\item Quelle: Gespräch mit Projektentwickler
%\end{itemize}

\section{Projektentwickler}

Unter Projektentwickler versteht man eine Berufsgruppe, die größere Immobilienprojekte konzipieren und erstellen.
Dabei organisieren sie die einzelne Schritte des Projektes, von der Suche nach geeigneten Grundstücken, die Finanzierung und Planung des Projektes bis zum Fertigstellen der schlüsselfertigen Immobilie, sowie deren Vermarktung.
Die meisten Teilaufgaben werden an externe Dienstleister weitergeleitet, wie etwa Architekten, Vermessungsingenieuren oder Makler.

Besonders interessant für das Katasteramt sind die kleinen bis mittleren Projektentwickler, da diese standortgebunden sind und somit lediglich Projekte in Wuppertal entwickeln können.
Zu dieser Zielgruppe gehört z.\,B. das Unternehmen Colemus, das in einem durchschnittlichen Projekt circa zehn Wohnhäuser baut.
Diese Gebundenheit der Projektentwickler ist einerseits auf die Beschaffung des Kredites zurückzuführen, da lediglich die lokalen Sparkassen solche Projekte finanzieren.
Ein weiterer Grund sind die regionalen Unterschiede in den Verwaltungswegen, die durchlaufen werden müssen um ein solches Projekt durchzuführen.
Ein dritter Grund sind die Beziehungen zu den Partnern, so arbeitet das Unternehmen Colemus bevorzugt immer mit den gleichen Dienstleistern zusammen.

Die Projektentwickler sind durch ihre verschiedene Dienstleister mit denen sie zusammenarbeiten eine interessante Kundengruppe für das Katasteramt.
Durch sie steigt der Bekanntheitsgrad von \ac{WuNDa} bei ihren Dienstleister die dadurch ebenfalls als externe Benutzer gewonnen werden können.
Dies zeigte sich bereits im Gespräch, da hier solche Dienstleister ebenfalls am Tisch saßen und von \ac{WuNDa} angesprochen wurden, obwohl sie nicht die direkte Zielgruppe waren.

Weiterhin sind die Projektentwickler interessant für die Stadt Wuppertal an sich.
So werden diese von einer guten Unterstützung einer Stadt angezogen, da diese somit mehr Projekte in kürzerer Zeit durchführen können, wovon die Stadt schlussendlich profitiert da in sie investiert wird und der Immobilienmarkt angetrieben wird.
Zu dieser Unterstützung gehören möglichst einfache und schnelle Verwaltungswege sowie eine gute Zusammenarbeit zwischen den verschiedenen Ämtern.
Kurzum das schnelle Beziehen von Informationen über \ac{WuNDa} gehört auch zu dieser Unterstützung, die das Katasteramt anbieten und unter Umständen noch ausbauen kann.

Aus diesem Grund wird, falls die Projektentwickler als externe Benutzer und ihre Dienstleister gewonnen werden können, in Zukunft \ac{WuNDa} dahingehend entwickelt, so dass es attraktiver für diese Kundengruppen wird.
Deshalb wurde in dem gemeinsamen Gespäch auch auf die fehlenden Funktionalitäten eingegangen, die in den nächsten zwei Abschnitten behandelt werden.

\section{Aufgekommene Featurewünsche}
In diesem Abschnitt werden die Features kurz beschrieben, die während dem Gespräch geäußert wurden und die in \ac{WuNDa} noch nicht realisiert wurden.


\todo{Nach oben verschieben}
Es wird versucht die Funktionalitäten in verschiedene Kategorien zu unterteilen.
\section{Erstellen von Dokumenten}
Nachdem ein Projektleiter ein oder mehrere Grundstücke gefunden hat, bei denen er die Meinung ist, dass er dort ein Projekt entwickeln kann, muss er die ersten Schritte der Realisierungsphase einleiten.
Hierbei muss er vor allem Überzeugungsarbeit leisten, um Kredite bei den Kreditinstituten zu erhalten und die benötigten Genehmigungen bei den verschiedenen Ämtern zu erhalten.
Bei dieser Präsentation seines Projektes ist es hilfreich wenn die, von ihm, vorgelegten Dokumente ein möglichst professionelles Aussehen haben und möglichst vollständig sind.





\subsubsection{Plotrahmen anpassen}
Der Benutzer kann zu jedem Zeitpunkt einen Ausschnitt der Cismap, sprich der Karte in \ac{WuNDa}, in eine PDF-Datei speichern.
Eine solche PDF-Datei kann in Abbildung \vref{fig:cismap-rathaus} betrachtet werden.
Ein jeder solcher Abdruck der Cismap besitzt einen sogenannten Plotrahmen. In diesem Rahmen befinden sich einige Informationen zu dem abgebildeten Bild.
Im Augenblick ist es möglich in \ac{WuNDa} einen Text anzugeben, der dann in der dem mittleren Feld des Rahmens abgebildet wird.
Eine neue Funktionalität wäre es den Rahmen individueller gestalten zu können, so dass externe Benutzer z.\,B. ihr Logo in dem Rahmen abbilden können.
Dabei soll das Logo des Katasteramtes erhalten bleiben, da es die Herkunft des Bildes angibt und somit die Richtigkeit der Bildes gewährleistet. 

\subsubsection{Zugriff auf Daten der WSW}
Die \ac{WSW} bietet Interessenten eine kostenlose Auskunft bezüglich den Daten, die das \ac{WSW} verwaltet.
Hierzu gehören u.\,a. Trassen, wie etwa Strom- und Gasleitungen, Kanaldaten aber auch Buslinien.
Für die Projektentwickler und seine Dienstleister wäre es nützlich diese Daten auch in \ac{WuNDa} importieren zu können, um sie z.\,B. in der Cismap anzeigen zu lassen.
Aus technischer Sicht wäre dies machbar, allerdings scheitert eine Umsetzung  im Augenblick an den Verhandlungen zwischen dem Katasteramt und dem \ac{WSW}.
Somit stehen die Daten nicht zur Verfügung.
\subsubsection{Zugriff auf Daten des Grundbuches}
Die Projektentwickler und ihre Dienstleister müssen regelmäßig Dokumente vom Grundbuchamt beziehen.
Theoretisch könnte dies in \ac{WuNDa} ähnlich, wie das Beziehen von Produkten, umgesetzt werden.

Ein solcher einfacher Zugriff auf die Daten des Grundbuchamtes scheitert allerdings an dem starken Datenschutz, dem das Grundbuch und insbesondere sein Online-Zugriff unterliegt.
So dürfen, per Gesetz, z.\,B. nur Gerichte, Notare und \acp{ÖbVI} uneingeschränkt auf die Internet-Grundbucheinsicht zugreifen.
Andere Nutzungsbrechtigte, wie etwa Banken und Rechtsanwälte, können die Internet-Grundbucheinsicht lediglich eingeschränkt nutzen.
Die bedeutet dass sie vor jeder Einsicht ein berechtigtes Interesse angeben müssen \autocite[vgl.][]{justiz-grundbucheinsicht}.

 
\subsubsection{Abfragen eines berechtigten Interesse}
Das Katasteramt besitzt ebenfalls Daten, die erst ausgegeben werden dürfen, nachdem ein berechtigtes Interesse angegeben wurde.
Dies ist z.\,B. der Fall bei Baulasten und ???.
Diese Funktionalität würde dem Katasteramt und den externen Kunden Zeitaufwand sparen und wird im \todo{ref setzen und vervollständigen} näher behandelt.

\subsubsection{Erschließungsbeiträge}
Die Erschließungsbeiträge werden für die Herstellung von Straßen, Wegen, Plätzen etc... erhoben und müssen von den Eigentümer oder Erbbauberechtigte der anliegenden Grundstücken bezahlt werden.
Weiterhin werden die Erschließungsbeiträge von dem Ressort Straßen und Verkehr beantragt \autocite[vgl.][]{wupp-erschliessungsbeitrag}.

In \ac{WuNDa} werden die Erschließungsbeiträge nicht abgebildet, diese Funktionalität wäre jedoch nützlich für die Projektentwickler und ihre Dienstleister.
Zusätzlich wurde die Funktionalität, in einem anderen Gespräch, von der Kundengruppe der Gutachter ebenfalls angefragt.
Somit handelt es sich hierbei um eine Funktionalität deren Umsetzung näher betrachtet werden sollte, insbesondere da die Erschließungsbeiträge von einem Ressort der Stadtverwaltung Wuppertal verwaltet wird.

\subsubsection{Abfrage der Dachhöhen}
In \ac{WuNDa} ist es möglich die Dachhöhen abzufragen, diese Funktionalität wurde allerdings noch für die externen Benutzer freigegeben.
Sie wäre jedoch nützlich für die Projektentwickler und ihre Dienstleister um ein neue Gebäude ins bestehende Stadtbild einzufügen.
Mehr Informationen hierzu befinden sich in \todo{ref setzen}

\subsubsection{3D-Ansicht}
Im Gespräch kam auf, dass eine 3D-Ansicht der Stadt Wuppertal nice-to-have wäre.
Diese Ansicht könnte z.\,B. dafür verwendet werden um sich ein Bild der Lage zu verschaffen, oder um in Dokumenten darzustellen, wie sich die Gebäude eines Projektes, in die aktuelle Umgebung einfügen.

Diese Funktionalität wird allerdings nicht in näher Zukunft realisiert werden, da sie sehr kostspielig ist.
Zum einen ist hier die Technik in \ac{WuNDa} noch nicht ausgreift, insbesondere fehlt ein vollständiges 3D-Modell der Stadt Wuppertal, welches zuerst angefertigt werden müsste.
Weiterhin gibt es für das Katasteramt selbst keinen konkreten Anwendungsfall, welches eine solche Investition rechtfertigen würde.  

\subsubsection{Verwaltung von eigenen Stammdaten}
Im Augenblick haben sämtliche Daten, die von \ac{WuNDa} benutzt werden, einen direkten Bezug zu dem R102.
Durch das Aufkommen der externen Benutzer kann es nützlich werden, dass diese eigene Daten in \ac{WuNDa} anzeigen wollen.
Ein Beispiel wäre das Hinzufügen und Verwalten von Bookmarks für häufig benutzte Objekte oder angezeigte Stellen in der Karte.
Andere Anwendungsfälle mit persönlicheren Daten sind ebenfalls denkbar, hier wurde im Gespräch bereits deutlich dass solche Daten im Besitz der externen Benutzer bleiben muss und nicht, wie vorgeschlagen, von dem R102 verwaltet werden kann.
\subsubsection{Freigeben der Layer für Problemimmobilien und Schrottimmobilien}
Das R102 besitzt einige Layer die für Öffentlichkeit nicht freigegeben sind.
Im Gespräch wurde hier das Beispiel der Problemimmobilien und Schrottimmobilien genannt.
Als Schrottimmobilien werden Immobilien bezeichnet, die in einem minderwertigen oder mangelhaften Zustand deutlich über ihren Wert verkauft wurden.
Hierunter fallen häufig Gebäude, die nicht vollständig oder nur mangelhaft saniert wurden und dann an Privatkunden verkauft wurden \autocite[vgl.][]{iva-schrottimmo}.
Unter Problemimmobilien werden auf Grund von Mängel nur schwer verkäufliche Immobilien verstanden \autocite[vgl.][]{immo-problemimmo}.

Das Wissen um solche Immobilien ist nützlich für Projektentwickler, da diese sich häufig auf Grundstücken befinden, die für eine neues Projekt interessant sind.
So kann z.\,B. das Grundstück einer Problemimmobilie für einen Neubau geeignet sein.
Andererseits ist eine Gegend mit zu vielen solcher Immobilien eher zu meiden.

Die entsprechenden Layer können jedoch nicht ohne weiteres freigegeben werden, da auch hier  der Datenschutz beachtet werden muss.
Nichtsdestotrotz besitzt die Stadt Wuppertal selbst solche Immobilien, so dass der Datenschutz in diesem Fall keine Rolle spielt und die Layer in einem ersten Schritt eingeschränkt zur Verfügung gestellt werden könnten.
  
\subsubsection{Anzeigen von zukünftige Vorhaben}
Informationen zu zukünftigen Vorhaben auf Grundstücken, können das Investitionspotential einer Gegend erhöhen.
So wurde z.\,B. im vorherigen Abschnitt erwähnt dass zu viele Schrottimmobilien in einer Gegend einen Projektentwickler davon abschrecken können in diese Gegend zu investieren.
Dem könnte entgegengewirkt werden in dem die Projektentwickler Informationen dazu hätten, welche Vorhaben bereits auf diesen Grundstücken in Planung sind, wie etwa eine Komplettsanierung von solchen Immobilien oder das Erstellen von Neubauten auf diesen Grundstücken.
Weitere Informationen zu anderen zukünftigen Vorhaben, wären aber auch hilfreich, wie etwa welche Grundstücke in Bauland umgewandelt werden oder entwickelt werden. 
Dadurch hätten die Projektentwickler einen Einblick in die nahe Zukunft und die Entwicklung einer Gegend und könnten somit ein Projekt dort vorbereiten, obwohl diese Gegend zu diesem Zeitpunkt noch nicht interessant ist.
Kurzum das Investitionspotential einer Gegend wäre lediglich durch Informationen gestiegen, die zwar bereits vorhanden, jedoch in solcher Form noch nicht veröffentlicht oder nur schwer zugänglich sind.

Diese Funktionalität wäre aus technischer Sicht umsetzbar, da lediglich zusätzliche Informationen zu einem Grundstück angezeigt werden müssten.
Dies könnte über neue Layer oder neue Objekte in \ac{WuNDa} realisiert werden.
Allerdings gibt es diese Art der Informationen im Augenblick nicht und müssten in einem ersten Schritt zusammengetragen werden und in Zukunft auf einem aktuellen Stand gehalten werden.
Aus diesem Grund müsste deshalb zuerst analysiert werden, welche Daten relevant sind, in welchen Ämtern diese erfasst werden und es müssten Gespräche mit diesen Ämtern für dieses Kooperationsprojekt angestoßen werden.
 
\subsubsection{Luftlinienentfernung von einem Grundstück zu interessanten Orten}
Für die Projektentwickler und die Vertreiber der Immobilien spielt die Qualität der Lage eine wichtige Rolle, da über diese den Endpreis mitbestimmt wird und sich somit auch in deren Rendite widerspiegelt.
Eine Möglichkeit die Qualität der Lage zu bestimmen, ist es die Luftlinienentfernung zu verschiedenen interessanten Orten für die späteren Bewohner zu berechnen.
Diese Gegebenheit dieser Orte unterscheidet sich je nach Zielgruppe, typischerweise handelt es sich um Orte wie Schulen, Supermärkte, Ärzte, Freizeitbeschäftigungen, Bushaltestellen etc...

Über die Cismap kann in \ac{WuNDa} die Entfernung zwischen zwei Punkten berechnet werden.
Trotzdem kann diese Funktionalität nicht sofort umgesetzt werden, da Informationen bezüglich diesen interessanten Orten fehlen.
Weiterhin müssten auch die verschiedenen Zielgruppen bestimmt werden sowie ihr Interessegrad an den verschiedenen Orten.
So sind z.\,B. junge Eltern interessierter an Schulen (Vor-, Grund- und Sekundarschulen) als Senioren.

\subsubsection{Anzeigen des Baujahres}
Ein weiterer Layer der im Gespräch aufkam war das Anzeigen des Baujahres in Zehnjahres-Gruppen.
Dadurch ist sofort ersichtlich wie eine Gegend entstanden ist und welche letzten Investitionen in ihr getätigt wurden.


\section{Beleuchtung einiger Features}

