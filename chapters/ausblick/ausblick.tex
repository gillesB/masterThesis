\chapter{Gewinnen von neuen externen Benutzern} \label{ch:aussicht}

In diesem Kapitel wird beschrieben wie das R102 neue externe Benutzer gewinnen will. Dabei wird zuerst auf die Motivation des Katasteramtes eingegangen, danach wird eine solche Gruppe von möglichen externen Benutzer beschrieben, die Projektentwickler. Danach werden mögliche neue Funktionalitäten in \ac{WuNDa} kurz beschrieben und anschließend wird eine Auswahl von diesen detaillierter beleuchtet.

Die Informationen zu diesem Kapitel stammen aus einem gemeinsamen Treffen zwischen dem Katasteramt, einem Projektentwickler, dem Unternehmen Colemus \autocite{colemus}. Sowie einem Architekten und einem \ac{ÖbVI} die häufig mit Colemus zusammenarbeiten. Der Autor dieser Arbeit war bei diesem Treffen ebenfalls anwesend.

\section{Motivation des Katasteramtes.}

Durch die bereits beschriebene Rationalisierung innerhalb des Katasteramtes, wird es nötig, dass mehr Kunden des Katasteramtes ihre Informationen über \ac{WuNDa} beziehen und somit die Beamten des Katasteramtes entlasten.
Somit soll die Geasmtanzahl der externen Benutzer erhöht werden.
Dabei hat sich das Katasteramt das Ziel gesetzt für diese Benutzer, trotz des Personalabbaus, die Nutzbarkeit zu erhöhen, wobei der Umsatz, der durch die eingenommenen Gebühren, gemacht wird in etwa gleich bleiben soll.
Im Augenblick benutzen nur die Polizei und die \acp{ÖbVI} \ac{WuNDa} als externe Benutzer.
Dieses hat bei diesen Berufsgruppen bereits eine breite Akzeptanz gefunden, so dass hier keine weiteren externen Benutzer erschlossen werden können.
Somit muss das  Katasteramtes dort hingehend Anstrengungen unternehmen um \ac{WuNDa} interessant für andere Berufsgruppen zu machen.
Dabei werden Gespräche mit Repräsentanten dieser Gruppen geführt um diesen den aktuellen Stand von \ac{WuNDa} aufzuzeigen und um eventuelle neue Funktionalitäten ausfindig zu machen.
Falls die Funktionalitäten umsetzbar sind, so sollen diese auch umgesetzt werden, so dass schlussendlich die Attraktivität von \ac{WuNDa} erhöht wird und mehr Zuspruch bei den einzelnen Kunden aus diesen Berufsgruppen findet.

Das Katasteramt hat einige Kriterien für die neuen Funktionalitäten festgelegt. 
So soll insgesamt der Zugriff auf die benötigten Informationen, für die Kunden, schneller und direkter werden.
Weiterhin soll möglichst wenig zusätzliche Arbeit für das Katasteramt anfallen.
So sollen z.\,B. keine neuen Datenbestände erstellt werden, die zusätzlich gepflegt werden müssen.
Dies bedeutet im Umkehrschluss, dass lediglich Datenbestände benutzt werden können die das Katasteramt aus anderen, z.\,B. rechtlichen, Gründen pflegen muss.
Weiterhin muss bei den Funktionalitäten darauf geachtet werden dass keine rechtliche Gegebenheiten verletzt werden, so z.\,B. das Datenschutzrecht.

Zu den angesprochen Berufsgruppen gehören die Projektentwickler, welche in dem nächsten Abschnitt beleuchtet werden.


%\begin{itemize}
%	\item Oben wurde beschreiben warum die ent. Benutzer Daten anfordern müssen, was der Grunde der gewünschten Änderungen war.
%	\item Nun wird ein Schritt weiter gedacht
%	\item Quelle: Gespräch mit Projektentwickler
%\end{itemize}

\section{Projektentwickler}

Unter Projektentwickler versteht man eine Berufsgruppe, die größere Immobilienprojekte konzipieren und erstellen.
Dabei organisieren sie die einzelne Schritte des Projektes, von der Suche nach geeigneten Grundstücken, die Finanzierung und Planung des Projektes bis zum Fertigstellen der schlüsselfertigen Immobilie, sowie dessen Vermarktung.
Die meisten Teilaufgaben werden an externe Dienstleister weitergeleitet, wie etwa Architekten, Vermessungsingenieuren oder Makler.

Besonders interessant für das Katasteramt sind die kleinen bis mittleren Projektentwickler, da diese standortgebunden sind und somit lediglich Projekte in Wuppertal entwickeln können.
So werden in einem durchschnittlichen Projekt des Unternehmens Colemus circa zehn Wohnhäuser gebaut und entspricht somit der Zielgruppe.
Diese Gebundenheit der Projektentwickler ist einerseits auf die Beschaffung des Kredites zurückzuführen, da lediglich die lokalen Sparkassen solche Projekte finanzieren.
Ein weiterer Grund sind die regionalen Unterschiede in den Verwaltungswegen, die durchlaufen werden müssen um ein solches Projekt durchzuführen.
Ein dritter Grund sind die Beziehungen zu den Partnern, so arbeitet das Unternehmen Colemus bevorzugt immer mit den gleichen Dienstleistern zusammen.

Im Gespräch zeigte sich, dass Projektentwickler durch ihre Beziehungen mit anderen Dienstleistern eine besonders interessante Kundengruppe sind.
Dies liegt daran, dass hier Potential vorhanden ist diese Dienstleister ebenfalls als externe Benutzer zu gewinnen.

Weiterhin sind die Projektentwickler interessant für die Stadt Wuppertal an sich.
So werden diese von einer guten Unterstützung einer Stadt angezogen, da diese somit mehr Projekte in kürzerer Zeit durchführen können, wovon die Stadt schlussendlich profitiert da in sie investiert wird und der Immobilienmarkt angetrieben wird.
Zu dieser Unterstützung gehören möglichst einfache und schnelle Verwaltungswege sowie eine gute Zusammenarbeit zwischen den verschiedenen Ämtern.
Kurzum das schnelle Beziehen von Informationen über \ac{WuNDa} gehört auch zu dieser Unterstützung, die das Katasteramt anbieten und unter Umständen noch ausbauen kann.

Aus diesem Grund wird, falls die Projektentwickler als externe Benutzer und ihre Dienstleister gewonnen werden können, in Zukunft \ac{WuNDa} dahingehend entwickelt, dass es attraktiver für diese Kundengruppen wird.
Deshalb wurde in dem gemeinsamen Gespäch auch auf die fehlenden Funktionalitäten eingegangen, die in den nächsten zwei Abschnitten behandelt werden.

\section{Aufgekommene Featurewünsche}
In diesem Abschnitt werden die Features kurz beschrieben, die während dem Gespräch geäußert wurden und die in \ac{WuNDa} noch nicht realisiert wurden.
\subsubsection{Plotrahmen anpassen}
\subsubsection{Exposee erstellen}
\subsubsection{Pläne auf Cismap ziehen}
\subsubsection{Zugriff auf Daten der WSW}
\subsubsection{Zugriff auf Daten des Grundbuches}
\subsubsection{Abfragen eines berechtigten Interesse}
\subsubsection{Erschließungsbeiträge}
\subsubsection{Thematischer Durchstich}
\subsubsection{Export im DXF-Format}
\subsubsection{Korrekturbetrag anzeigen}
\subsubsection{Dachhöhenabfrage}
\subsubsection{3D-Daten}
\subsubsection{Verwaltung von eigenen Stammdaten}
\subsubsection{Verschiedene nicht öffentliche Layer freigeben (Schrottimmo)}
\subsubsection{zukünftige Vorhaben anzeigen. Investitionspotential}
\subsubsection{Luftlinienentfernung von einem Grundstück zu}
\subsubsection{Anzeigen des Baujahres}



\section{Beleuchtung einiger Features}

