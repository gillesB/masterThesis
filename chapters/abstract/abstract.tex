\begin{abstract}
Aufgrund einer Rationaliserung in der Stadtverwaltung Wuppertal müssen einige Arbeittschritte des Ressort 102  Vermessung, Katasteramt und Geodaten automatisiert werden.
Diese Arbeit beschreibt die Automatisierung der bestehenden Abrechnungskomponente, die ein Teilsystem des, im Ressort 102 bereits verwendeten, Infomrationssystems WuNDa darstellt.
Erst durch diese Komponente wird es externen Benutzern von WuNDa, dies sind Benutzer, die keine Beamten des Ressort 102 sind, ermöglicht verschiedene Produkte, dies sind vorallem Dokumente und Berichte, über WuNDa herunterladen zu können.
Dies liegt am durchaus kostenpflichtigen Zugriff, der dadurch protokolliert werden muss, so dass nach einem bestimmten Abrechnungsturnus eine Abrechnung an die jeweiligen Benutzer zu stellen.
Diese Arbeit beschreibt in einem ersten Schritt die fachliche Seite dieser Abrechnungskomponente, deren aktuelle Umsetzung in WuNDa sowie die manuelle Erstellung der Abrechnung.
Anschließend folgt die Spezifikation der, für die Automatisierung erforderlichen, Funktionalitäten und deren Implementierung wird beschrieben.
Zu Schluss der Arbeit wird die, aufgrund der Rationalisierung bestehenden, Motivation des Ressort 102 beschrieben um neue externe Benutzer zu gewinnen, dabei wird die Kundengruppe der Projektentwickler vorgestellt und es werden von diesen gewünschte, potentielle Funktionalitäten in WuNDa vorgestellt.
\end{abstract}