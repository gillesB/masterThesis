\begin{abstract}
	In dem cids basierten System WUNDa, das in der Stadtverwaltung Wuppertal eingesetzt wird, wurde für ein 
	Teilsystem (Abgabe der Produkte des Liegenschaftskatasters) eine Abrechnungskomponente entworfen und implementiert. 
	Für den wichtigen Bereich der Vermessungsunterlagen umfassen diese Produkte zum einen Auszüge aus dem Amtlichen
	Liegenschaftskatasterinformationssystem ALKIS (Flurkarte, Flurstücks- und Eigentümernachweise, Punktlisten, NAS-Daten),
	zum anderen durch Scannen analoger Originale entstandene digitale Dokumente (Fortführungsrisse, Grenzniederschriften, 
	Aufnahmepunktkarten (AP-Karten), Nivellementspunkt-Beschreibungen (NivP-Beschreibungen), AP-Übersichten etc.). 
	Die erforderlichen online-Systeme zur Selbstentnahme dieser Produkte wurden über mehrere Vergaben seit 
	12/2008 in Form von WuNDa-Fachverfahren und -Fachthemen  hergestellt.
			
	Auf Grund fachrechtlicher Vorgaben und wegen datenschutzrechtlicher 
	Belange müssen die von den Endkunden vorgenommenen Datenentnahmen durch die Stadt 
	Wuppertal protokolliert werden. Dabei müssen einige vom externen Nutzer zu liefernde Angaben 
	abgefragt werden, z. B. der Verwendungszweck der Daten - hieraus ergibt sich, ob die 
	Datenentnahme kostenfrei oder kostenpflichtig ist - und die Nummer, unter der er die Datenentnahme in seinem Geschäftsbuch führt.
			
	Um das volle Rationalisierungspotenzial der Selbstentnahmen ausschöpfen zu können, 
	müssen auch gebührenpflichtige Datenentnahmen unterstützt werden. Dafür müssen die WuNDA-Beschreibungsseiten, 
	aus denen heraus Datendownloads ermöglicht werden, um weitere Funktionen und Oberflächenelemente zur Gebührenberechnung, 
	Gebührenanzeige und zum Akzeptieren der Gebührenforderung ergänzt werden. 
	Jedes Herunterladen solcher Daten führt zu einem Protokolleintrag in der sog. „Billing“-Tabelle der WuNDa-Datenbank.
	Die Lösung ermöglicht es, die Prozesse rund um die Selbstentnahme von Vermessungsunterlagen 
	medienbruchfrei zu gestalten und die betroffenen WuNDa-Teilsysteme mit dem Geschäftsbuch 
	des Ressorts 102 zu verknüpfen.
			
	Aus der Billing-Tabelle werden turnusmäßige Abrechnungen erzeugt, wobei je nach Kundengruppe und Verwendungszweck
	der Daten monatliche, quartalsweise und jährliche Abrechnungen möglich sind. Die Abrechnungen enthalten eine Liste 
	aller Downloads des jeweiligen Nutzers mit den zugehörigen Gebühren bzw. den geschuldeten Entgelten. Die Unterstützung dieser Abrechnungen wird vom 
	System genauso unterstützt wie die Anzeige der "Kontoinformationen" für den Endkunden der Datenendnahme.
			
			
	Die Arbeit gibt zunächst einen Überblick über die abzugebenden Produkte und ihre Verwendung in den typischen Prozessen der Kunden der Stadtverwaltung. 
	Anschließend wird die Integration des Billing Mechanismus in die Fachsysteme, die Preisermittlung und die Protokollierung der Produktentnahmen erläutert.
	Der Auswertungsteil der Software wird aus der Sicht der Endkunden und aus der Sicht der Fachabteilungen erläutert. Die Benutzerinteraktion und die 
	betroffenen GUI Elemente werden beschrieben.
			
	Im abschließenden Teil der Arbeit wird auf die wirtschaftlichen Aspekte (Einsparungen) in der Stadtverwaltung eingegangen und diese werden anhand eines Beispiels 
	veranschaulicht. Im Ausblick wird die Integrationsmöglichkeit von cids/WuNDa in weitere Prozesse der Endkunden (Beschränkung auf ÖbVi, Sparkassen und Polizei) untersucht 
	und die notwendigen weiteren Entwicklungsschritte skiziert.
\end{abstract}