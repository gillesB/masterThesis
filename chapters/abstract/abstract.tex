\begin{abstract}
Aufgrund einer Rationalisierung in der Stadtverwaltung Wuppertal müssen einige Arbeitsschritte des Ressorts 102 Vermessung, Katasteramt und Geodaten automatisiert werden.
Diese Arbeit beschreibt die Automatisierung der bestehenden Abrechnungskomponente, die ein Teilsystem des, im Ressort 102 bereits verwendeten, Informationssystems \acs{WuNDa} darstellt.
Erst durch diese Komponente wird es externen Benutzern von \acs{WuNDa}, dies sind Benutzer, die keine Beamten des Ressorts 102 sind, ermöglicht verschiedene Dokumente und Berichte über \acs{WuNDa} herunterzuladen.
Dies liegt am durchaus kostenpflichtigen Zugriff, der dadurch protokolliert werden muss, so dass nach einem bestimmten Abrechnungsturnus eine Abrechnung an die jeweiligen Benutzer gestellt werden kann.
Diese Arbeit beschreibt in einem ersten Schritt die fachliche Seite dieser Abrechnungskomponente, deren aktuelle Umsetzung in \acs{WuNDa} sowie die manuelle Erstellung der Abrechnung.
Anschließend folgt die Spezifikation der, für die Automatisierung erforderlichen, Funktionalitäten und deren Implementierung wird beschrieben.
Zu Schluss der Arbeit wird die, aufgrund der Rationalisierung bestehenden, Motivation des Ressorts 102 zum Gewinnen von neuen externen Benutzern beschrieben, dabei wird die Kundengruppe der Projektentwickler, sowie deren Workflow, vorgestellt und es werden, von dieser, gewünschte, potentielle Funktionalitäten in \acs{WuNDa} vorgestellt.
\end{abstract}