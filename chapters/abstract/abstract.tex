\begin{abstract}
Aufgrund einer Rationalisierung in der Stadtverwaltung Wuppertal müssen einige Arbeitsschritte des Ressorts 102 Vermessung, Katasteramt und Geodaten automatisiert werden.
Diese Arbeit beschreibt die Automatisierung der bestehenden Abrechnungskomponente, die ein Teilsystem des im Ressort 102 bereits verwendeten Informationssystems \acs{WuNDa} darstellt.
Erst durch diese Komponente wird es externen Benutzern ermöglicht verschiedene Dokumente und Berichte über \acs{WuNDa} zu beziehen.
Bei externen Benutzern handelt es sich um Benutzer von \acs{WuNDa}, die keine Beamten des Ressorts 102 sind.
Durch das unter Umständen kostenpflichtige Beziehen der Dokumente, muss der Zugriff auf diese protokolliert werden, so dass nach einem bestimmten Abrechnungsturnus eine Abrechnung an die jeweiligen Benutzer gestellt werden kann.
Diese Arbeit beschreibt in einem ersten Schritt die fachliche Aspekte dieser Abrechnungskomponente, deren aktuelle Umsetzung in \acs{WuNDa} sowie die manuelle Erstellung der Abrechnung.
Anschließend folgt die Spezifikation der für die Automatisierung erforderlichen Funktionalitäten und deren Implementierung wird beschrieben.
Zum Schluss der Arbeit wird beschrieben, wie das Ressort 102 neue externe Benutzer gewinnen will. Die Motivation hierbei ist vor allem die geplante Rationalisierung.
Dabei wird die Kundengruppe der Projektentwickler, sowie deren Workflow, vorgestellt und es werden potentielle Funktionalitäten in \acs{WuNDa} vorgestellt.
\end{abstract}